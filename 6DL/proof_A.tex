If we add the post-smoothing with a symmetric form which means we 
use $[S^\ell \ast ]^\top$ as the smoother, then we can get the V-cycle
version multigrid algorithm.

\begin{breakablealgorithm}%[!htb]
	\caption{$u = {\text{V-MG}}(f; J,\nu_1, \cdots, \nu_J)$}
	\label{alg:V-cycle}
	\begin{algorithmic}
		\State Call Algorithm \ref{alg:L-Slash0},
		$$
		(u^{\ell,\nu_\ell}: ~\ell = 1:J) = {\text{MG0}}(f; J,\nu_1, \cdots, \nu_J).
		$$
		\State Prolongation and restriction from coarse to fine level
		\For{$\ell = J-1:1$}
		\State
		$$
		u^{\ell,0} \leftarrow u^{\ell,\nu_\ell} + R  \ast_2^{\top} u^{\ell+1, \nu_{\ell+1}}.
		$$
		\For{$i = 1:\nu_\ell$}
		\State
		$$
		u^{\ell,i} = u^{\ell,i-1} + \tilde S^\ell \ast (f^\ell - A_\ell \ast u^{\ell,i-1})
		$$
		\EndFor
				%		\ENDIF
		\EndFor
		\State Output
		$$
		u = u^{1,\nu_\ell}.
		$$
	\end{algorithmic}
\end{breakablealgorithm}
Here $\tilde S^\ell $ means the central symmetry of kernel for smoother $S^\ell$ as in the 
definition of \eqref{eq:def_tildeK} in Lemma \ref{lemm:tilde-K}.


\begin{theorem}
If $R$ is consistent with $A_\ell$ which means that $R$ should be linear or bi-linear as $A_\ell$, then we have
the $A_{\ell+1}$ operation in coarse grid defined in \eqref{eq:def_coarse} is the same with $A_\ell$.
\end{theorem}

\begin{proof}
By the definition above, we have that
\begin{equation}
A_{\ell+1} (v) = \mathcal S\left( (R\ast A_{\ell} \ast R )\ast \mathcal S^\top(v) \right),
\end{equation}
because of the properties of convolution we know that 
\begin{equation}\label{eq:K=RAR}
(R\ast A_{\ell} \ast R )\ast = K \ast,
\end{equation}
for some 
$$
K \in \mathbb{R}^{7\times 7}.
$$
Then we have the next computation for $A_{\ell+1}(v)$
\begin{equation}\label{eq:compute_A}
\begin{aligned}
[A_{\ell+1} (v)]_{i,j} &= [\mathcal S\left( (R\ast A_{\ell} \ast R )\ast \mathcal S^\top(v) \right)]_{i,j}, \\
&= [ K\ast  \mathcal S^\top(v)]_{2i,2j}, \\
&= \sum_{p,q=-3}^{3} [\mathcal S^\top (v)]_{2i+p, 2j+q} K_{p,q}, \\
&= \sum_{p,q=-1}^1  [\mathcal S^\top (v)]_{2(i+p), 2(j+q)} K_{2p,2q}, \\
&= \sum_{p,q=-1}^1  v_{i+p, j+q} \hat K_{p,q}, \\
\end{aligned}
\end{equation}
	
Thus to say, we have
\begin{equation}
A_{\ell+1}(v) =  \hat K \ast v,
\end{equation}
with 
$$
\hat K_{p,q} = K_{2p,2q}, \quad p,q = -1,0,1,
$$
with $K$ is defined in \eqref{eq:K=RAR}.

Then by the direct computation of \eqref{eq:K=RAR} as 
$$
(R\ast A_{\ell} \ast R )\ast = K \ast
$$
and take the even index we have that
\begin{equation}
A_{\ell+1} = \hat K = A_\ell,
\end{equation}
if $R$ is consistent with $A_\ell$ which means that $R$ should be linear or bi-linear as $A_\ell$.	
\end{proof}