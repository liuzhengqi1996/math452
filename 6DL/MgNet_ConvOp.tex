\subsection{Convolution in one grid (without stride)}
%Firstly, as mentioned before, an image type
%data with multichannel can be understand as  
%vector valued functions
%$$
%f \in \mathbb{R}^{(2^m+1)\times(2^n+1) \times c}, \quad \text{or}\quad [f]_i \in
%\mathbb{R}^{(2^m+1)\times(2^n+1)}, \quad i = 1:c,
%$$
Here we mention our new set up for all the next notes as all image with
size like 
\begin{equation}\label{eq:size_f}
m = m_1 = 2^{s} + 1,  \quad n = n_1 = 2^t  +1.
\end{equation}
By the linear finite element discretization and bilinear finite element discretization, 
we have \eqref{2d-fe0} and \eqref{2d-fe1} respectively. 
A more general form for the discretization systems  is the so 
called convolution for $ g \in \mathbb{R}^{m\times n}$ with padding as:
%We first consider $\theta$ a convolution operator (with stride $1$) 
%and padding:
\begin{equation}\label{con1}
[K \ast g]_{i,j} = \sum_{p,q=-k}^k K_{k+1+p,k+1+q} g_{i + p, j + q}, \quad i=1:m + 1, j = 1:n + 1.
\end{equation}
The coefficients in \eqref{con1} constitute  a kernel matrix
\begin{equation}
K \in \mathbb{R}^{(2k+1) \times (2k+1)},
\end{equation}
where $k$ is often taken as small integers. 
Here padding means how choose $ g_{i+ p, j + q}$ 
when $(i+ p, j + q)$ is out of $1:m$ or $1:n$. 
Those next three choices are often used
\begin{equation}\label{eq:padding}
f_{i + p, j + q} = \begin{cases}
0,  \quad &\text{zero padding}, \\
f_{(i + p)\pmod{m}, (s + q)\pmod{n}},  \quad &\text{periodic padding}, \\
f_{|i-1 +p|, |j -1  +q|},  \quad &\text{reflected padding}, \\
\end{cases}
\end{equation}
if 
\begin{equation}
i + p \notin \{1, 2, \dots, m\} ~\text{or} ~  j+ q \notin \{1, 2, \dots, n\}.
\end{equation}
Here $ d \pmod{m} \in \{1, \cdots, m\} $  means the remainder when $d$ is divided by $m$.

By the definition of convolution \eqref{con1}, we can rewrite \eqref{2d-fe0} and \eqref{2d-fe1} as follows: 
\begin{equation}
K_A\ast:  \mathbb{R}^{m\times n}\rightarrow \mathbb{R}^{m\times n},~~~K_A\ast u=f, 
\end{equation}
where 
\begin{equation}\label{fe0_Ka}
	K_A=
	\begin{pmatrix}
	0 &-1&0\\
	-1& 4&-1\\
	0 &-1& 0
	\end{pmatrix}
	\end{equation}
for linear finite element discretization or	
	\begin{equation}\label{fe1_Ka}
	K_A=
	\begin{pmatrix}
	-1 &-1&-1\\
	-1& 8&-1\\
	-1 &-1& -1
	\end{pmatrix}.
	\end{equation}
for bilinear finite element discretization.

The gradient descent method for \eqref{laplace} or \eqref{laplace-h} can be derived as following 
\begin{itemize}
	\item $\min_{x\in \mathbb{R}^n} J(x)$ with $ J(x) =
          \frac{1}{2}u^TA u - f^Tu$ 
	\item Gradient:
	$$
	\nabla J= Au - f := - r.
	$$
	\item Given $u^k$, a new approximate solution is computed by two steps:
	$$
	r^k = f-Au^k,\quad u^{k+1} = u^k + \frac{\omega}{4} r^k.
	$$
\end{itemize}
Here we can clearly see that gradient descent method is equivalent to damped Jacobi method. Usually, we call this as 
smoother. A commonly used 
damped Jacobi is the one with damped coefficient $\omega$ with $\omega \in (0,2)$,  
which can be written as $S_{0}:\mathbb R^{m\times n}\mapsto
\mathbb R^{m\times n}$ satisfying
\begin{equation}
\label{jacobi1}
(S_{0}f)_{i,j}={\omega\over 4}f_{i,j},
\end{equation}
for equation \eqref{laplace-h} with initial guess zero.
If we apply the Jacobian iteration twice, then
$$
S_1(f) = S_{0} f + S_0(f - A(S_{0}f)),
$$
with element-wise form
\begin{equation} 
\begin{aligned}
\label{jacobi2}
[S_1(f)]_{i,j} &={1\over 4}\omega(2-\omega)f_{i,j} + {\omega^2\over 16}(f_{i+1,j}+f_{i-1,j}+f_{i,j+1}+f_{i,j-1}).
\end{aligned}
\end{equation}
Then by the definition of convolution \eqref{con1}, we have
 \begin{equation}\label{eq:convS}
S_{0}f = K_{S_{0}} \ast f \quad S_1 f = K_{S_1} \ast f.
\end{equation}
with
\begin{equation}\label{eq:kernel-S}
K_{S_{0}} = {\omega \over 4},
\end{equation}
and 
\begin{equation}\label{eq:kernel-S2}
K_{S_1} = \begin{pmatrix}
0 & \frac{\omega^2}{16} & 0 \\
\frac{\omega^2}{16} & {\omega(2-\omega) \over 4} & \frac{\omega^2}{16}  \\
0 & \frac{\omega^2}{16}  & 0
\end{pmatrix}.
\end{equation}





%Then we remark that, if we take convolution without pooling, there is a 
%shift with 
%$$
%i+ p  \rightarrow i + k+ p,
%$$
%and also for $j$ such that all index $(i + k+ p, j + k+ p)$ is during
%$1:n$. 

%\subsection{Convolution with stride by introducing stride operator}
As discussed above, the restriction mapping is consistent with the 
co-called convolution with stride $s=2$, which is defined as:
\begin{equation}\label{stride_2}
[K \ast_2 f]_{i,j} = \sum_{p,q=-k}^k K_{p,q} f_{2i + p, 2j + q},  
\quad i = 1: \frac{m+1}{2} , j = 1: \frac{n+1}{2}.
\end{equation}


\begin{remark}
More generally in CNN, given an integer $s\ge1$, a convolution with stride $s$ for $f \in \mathbb{R}^{(m)\times
		(2^n+1)}$ can also be defined as:
	\begin{equation}\label{stride}
	[K \ast_s f]_{i,j} = \sum_{p,q=-k}^k K_{p,q} f_{si + p, sj + q},  
	\quad i = 1: \lceil  \frac{m}{s}\rceil , j = 1: \lceil  \frac{n}{s}\rceil.
	\end{equation}
	
	Here $ \lceil  \frac{m}{s}\rceil$ denotes the smallest integer that greater than $\frac{m}{s}$.
\end{remark}

As general in CNN and multigrid method, convolution with stride $s=2$ is the most important case.
To make this operator more clear, now we are going to introduce the stride operator $\mathcal S$ as:
\begin{equation}\label{eq:strideopdim}
\mathcal S: \mathbb{R}^{m \times n} \mapsto \mathbb{R}^{\frac{m+1}{2} \times \frac{n+1}{2}},
\end{equation}
with
\begin{equation}\label{eq:strideop}
[\mathcal S(f)]_{i,j} = f_{2i, 2j}, \quad i  = 1:\frac{m+1}{2}, j = 1:\frac{n+1}{2}.
\end{equation}

Then we have the rewrite form of convolution with stride $2$ as:
\begin{equation}\label{eq:convstride_2_1}
K \ast_2 f = \mathcal S( K\ast f).
\end{equation}


Then the deconvolution in defined by the transposed of convolution 
with stride with respect to the F-inner product defined as:
$$
(u, v)_F := \sum_{i,j} u_{i,j}v_{i,j}.
$$
Then we have the definition of deconvolution as
\begin{equation}\label{eq:def_deconv}
(K \ast_2 u, v)_F = (u, K \ast_2^\top v),
\end{equation}
with
\begin{equation}
u \in \mathbb{R}^{m \times n} \quad \text{and} \quad v \in \mathbb{R}^{\frac{m+1}{2} \times \frac{m+1}{2}}.
\end{equation}

For a more comprehensive notation, let us also define 
$K\ast$ as an operator from $\mathbb{R}^{m\times n} \quad \text{and}$
to $\mathbb{R}^{m\times n}$ as $\mathcal C_K$.
Then we have the ``transpose'' of convolution without stride as:
\begin{equation}\label{eq:def_tran_conv}
(K \ast u, v)_F = (\mathcal C_K (u), v) =  (u,  \mathcal C_K^\top v).
\end{equation}

\begin{lemma}\label{lemm:tilde-K}
For the transposed convolution without stride, we have the next result
\begin{equation}\label{eq:}
\mathcal C_K^\top = \mathcal C_{\tilde K},
\end{equation}
where $\tilde K$ is defined as
\begin{equation}\label{eq:def_tildeK}
\tilde K_{p,q} = K_{-p, -q}, \quad p,q = -k:k.
\end{equation}
Intuitively, if we take $K_{0,0}$ as the center for the convolutional kernel $K$, 
then $\tilde K$ is the central symmetry of $K$. 
In 2D case, it can also be understood as the rotation of $\pi$ with respect to
the center $K_{0,0}$.
\end{lemma}

Then recall the definition of deconvolution in \eqref{eq:def_deconv} we have
\begin{equation}\label{eq:op_deconv}
\begin{aligned}
(u,  K \ast_2^\top v) &= (K \ast_2 u, v)_F = (\mathcal S \mathcal C_K u, v) \\
&= (u,  \mathcal C^\top_K \mathcal S^\top v),
\end{aligned}
\end{equation}
with definition
\begin{equation}\label{eq:de_stride_dim}
\mathcal S^\top:   \mathbb{R}^{\frac{m+1}{2} \times\frac{n+1}{2}} \mapsto \mathbb{R}^{m\times n},
\end{equation}
and 
\begin{equation}\label{eq:de_stride}
[\mathcal S^\top (f)]_{i,j} = 
\begin{cases}
0 \quad &\text{if i or j is even}, \\
f_{i/2, j/2}, \quad &\text{else}.
\end{cases}
\end{equation}

Thus to say, we have the simple version of the deconvolution for $K \ast $ as
\begin{equation}\label{eq:simple_deconv}
K \ast_2^\top v = \mathcal C_K\top \circ \mathcal S^\top (v) = \mathcal C_{\tilde K} \circ \mathcal S^\top (v) = \tilde K \ast \mathcal S^\top (v),
\end{equation}
thus to say
\begin{equation}\label{eq:final}
K \ast_2^\top  = \tilde K \ast \mathcal S^\top.
\end{equation}

In short, we have the next decomposition
\begin{itemize}
	\item convolution with stride = stride $ \circ$ convolution,
	\item deconvolution with stride  = transposed convolution $\circ$ transposed stride = convolution with the central symmetry of original kernel $\circ$ transposed stride.
\end{itemize}

\begin{theorem}\label{thm:deconv_op}
Here let us show a simple example about some details for computing deconvolution. 
Let us consider 
\begin{equation}
K_{p,q} \quad p,q = -1, 0, 1.
\end{equation}
Then we have 
$$
K \ast_2^\top v = \tilde K \ast \mathcal S^\top (v).
$$
As in \eqref{eq:de_stride} and the Lemma \ref{lemm:tilde-K}, we have the 
final version is 
\begin{equation}
\label{eq:7}
[K \ast_2^\top v ]_{2i,2j}=  K_{0,0}v_{i,j},
\end{equation}
with 
\begin{equation}
\label{eq:9}
[K \ast_2^\top v ]_{2i-1, 2j} = K_{0,1}v_{i-1,j} + K_{0,-1}v_{i,j}, \quad 
[K \ast_2^\top v ]_{2i, 2j-1} = K_{1,0}v_{i,j} + K_{-1,0}v_{i,j-1},
\end{equation}
and
\begin{equation}
%\begin{tiny}
%{\scriptsize 
[K \ast_2^\top v ]_{2i-1, 2j-1}  =  
K_{1,1}v_{i,j} + K_{-1,1}v_{i-1,j} + K_{1,-1}v_{i,j-1} + K_{-1,-1}v_{i-1,j-1}.
%\end{tiny}
%}
\end{equation}


\end{theorem}
