\section{Convolution}
For simplicity of exposition, we denote 
\begin{equation}\label{eq:size_f}
m = m_1 = 2^{s} + 1,  \quad n = n_1 = 2^t  +1.
\end{equation}
\begin{definition}
A convolution defined on $\mathbb{R}^{m\times n}$ is a linear mapping 
$K\ast: \mathbb{R}^{m\times n}\mapsto \mathbb{R}^{m\times n}$ defined with padding,  
for any $g \in \mathbb{R}^{m\times n}$ by:
%We first consider $\theta$ a convolution operator (with stride $1$) 
%and padding:
\begin{equation}\label{con1}
[K \ast g]_{i,j} = \sum_{p,q=-k}^k K_{p, q} g_{i + p, j + q}, \quad i=1:m, j = 1:n.
\end{equation}
\end{definition}
Here we note that the indices for the entries in $K$ are given un a special way. 
For example, if $k=1, K\in \mathbb R^{3\times 3}$, and 
$$
K=\begin{pmatrix}
	K_{-1,-1} &K_{-1,0} &K_{-1,1} \\
	K_{0,-1} &K_{0,0} &K_{0,1} \\
	K_{1,-1} &K_{1,0} &K_{1,1} \\
	\end{pmatrix}.
$$
  
The coefficients in \eqref{con1} constitute  a kernel matrix
\begin{equation}
K \in \mathbb{R}^{(2k+1) \times (2k+1)},
\end{equation}
where $k$ is often taken as a small integer. 
Here padding means how $ g_{i+ p, j + q}$ is defined
when $(i+ p, j + q)$ is out of $1:m$ or $1:n$. 
The following three choices are often used
\begin{equation}\label{eq:padding}
f_{i + p, j + q} = \begin{cases}
0,  \quad &\text{zero padding}, \\
f_{(i + p)\pmod{m}, (s + q)\pmod{n}},  \quad &\text{periodic padding}, \\
f_{|i-1 +p|, |j -1  +q|},  \quad &\text{reflected padding}, \\
\end{cases}
\end{equation}
if 
\begin{equation}
i + p \notin \{1, 2, \dots, m\} ~\text{or} ~  j+ q \notin \{1, 2, \dots, n\}.
\end{equation}
Here $ d \pmod{m} \in \{1, \cdots, m\} $  means the remainder when $d$ is divided by $m$.


\section{Convolution with stride}
\begin{definition}
For any given integer $s\ge1$, a convolution with stride $s$ for $g \in \mathbb{R}^{m\times n}$ is defined as:
	\begin{equation}\label{stride}
	[K \ast_s g]_{i,j} = \sum_{p,q=-k}^k K_{p,q} g_{si + p, sj + q},  
	\quad i = 1: \lceil  \frac{m+1}{s}\rceil, j = 1: \lceil  \frac{n+1}{s}\rceil.
	\end{equation}
	Here $ \lceil  \frac{m}{s}\rceil$ denotes the smallest integer that greater than $\frac{m}{s}$.
\end{definition}
Both in CNN and in multigrid method, convolution with stride $s=2$ is the most important case, namely
\begin{equation}\label{stride_2}
[K \ast_2 g]_{i,j} = \sum_{p,q=-k}^k K_{p,q} g_{2i + p, 2j + q},  
\quad i = 1: \frac{m+1}{2} , j = 1: \frac{n+1}{2}.
\end{equation}


\begin{lemma}
The convolution with stride $2$ can be rewritten as:
\begin{equation}\label{eq:convstride_2_1}
K \ast_2 g = \mathcal S( K\ast g),
\end{equation}
where $\mathcal S$ is a stride operator defined by:
\begin{equation}\label{eq:strideopdim}
\mathcal S: \mathbb{R}^{m \times n} \mapsto \mathbb{R}^{\frac{m+1}{2} \times \frac{n+1}{2}},
\end{equation}
with
\begin{equation}\label{eq:strideop}
[\mathcal S(g)]_{i,j} = g_{2i, 2j}, \quad i  = 1: \frac{m+1}{2} j = 1: \frac{n+1}{2}.
\end{equation}
\end{lemma}
