\chapter{Stochastic Differential Equations and Training Algorithms}
%% Jonathan's notes here

\section{Bayesian Statistics}
A Bayesian statistical model approximates a `true' probability distribution over a set $X$
from a collection of independent samples $x_1,...,x_n\in X$ (the data) drawn from the distribution. 
Here we call $X$ the sample space.

The Bayesian model consists of a set of parameters $\Theta$, along with an initial probability distribution $\mathbb{P}$ over 
the product space $\Theta\times X$. To avoid technical issues, we will assume that $X$ and $\Theta$ are either finite
sets or subsets of Euclidean space $\mathbb{R}^d$. This means that $X$ and $\Theta$ have a natural measure
$\mu_X$ and $\mu_\Theta$, respectively, which are either counting measure or Lesbesgue measure.
We will also assume that the distribution $\mathbb{P}$ is absolutely
continuous with respect to the product measure $\mu_\Theta\times \mu_X$, and thus has a density, which we denote by
$\mathbb{P}(\theta,x)$.

Abusing notation, we denote the marginal distributions of $X$ and $\Theta$ as follows
$$\mathbb{P}(x) = \int_{\Theta} \mathbb{P}(\theta,x)d\mu_\Theta$$
$$\mathbb{P}(\theta) = \int_{X} \mathbb{P}(\theta,x)d\mu_x$$
and the conditional distributions as
$$\mathbb{P}(x|\theta) = \frac{\mathbb{P}(\theta,x)}{\mathbb{P}(\theta)}$$
$$\mathbb{P}(\theta|x) = \frac{\mathbb{P}(\theta,x)}{\mathbb{P}(x)}$$

We want to update the joint distribution $\mathbb{P}(\theta,x)$ after seeing the data $D = \{x_1,...,x_n\}$. We denote this
new distribution by $\mathbb{P}(\theta,x|D)$. Additionally, we denote the new marginal and conditional distributions as follows
$$\mathbb{P}(x|D) = \int_{\Theta} \mathbb{P}(\theta,x|D)d\mu_\Theta$$
$$\mathbb{P}(\theta|D) = \int_{X} \mathbb{P}(\theta,x|D)d\mu_x$$
$$\mathbb{P}(x|s,D) = \frac{\mathbb{P}(\theta,x|D)}{\mathbb{P}(\theta)}$$
$$\mathbb{P}(\theta|x,D) = \frac{\mathbb{P}(\theta,x|D)}{\mathbb{P}(x)}$$

The question now becomes how to update this distribution after seeing the data. The key constraint
is that we choose to update $\mathbb{P}(\theta,x)$ in a way which keeps the conditional distributions 
$\mathbb{P}(x|\theta)$ fixed for each $\theta\in \Theta$. This means that we only need to specify how the marginal distribution
$\mathbb{P}(\theta)$ should be updated, as $\mathbb{P}(\theta,x) = \mathbb{P}(x|\theta)\mathbb{P}(\theta)$.

Given a sample $x_i$, we choose to set the marginal distribution $\mathbb{P}(\theta)$ equal to the conditional distribution
$\mathbb{P}(\theta|x_i)$. A formula for this new distribution is provided by Bayes' Law, which
follows easily from the definitions above:
\begin{equation}\label{bayes_law}
  \mathbb{P}(\theta|x_i) = \frac{\mathbb{P}(x_i|\theta)\mathbb{P}(\theta)}{\mathbb{P}(x_i)}
\end{equation}
Often, the normalizing constant $\mathbb{P}(x_i)$ (which is independent of $\theta$) is suppressed and this is written
\begin{equation}
  \mathbb{P}(\theta|x_i) \sim \mathbb{P}(x_i|\theta)\mathbb{P}(\theta)
\end{equation}
where $\sim$ means `is proportional to.'

Given $n$ independent samples $x_1,...,x_n$, this formula is iterated to obtain
\begin{equation}
  \mathbb{P}(\theta|D) \sim \mathbb{P}(x_1|\theta)\mathbb{P}(x_2|\theta)\cdots\mathbb{P}(x_n|\theta)\mathbb{P}(\theta)
\end{equation}
and the predicted (or approximated) distribution over $X$ given the data is obtained by marginalizing out the 
parameter $\theta$ (remember that we choose to keep the conditional distributions $\mathbb{P}(x|\theta)$ fixed independently
of the data $D$)
\begin{equation}
  \mathbb{P}(x|D) = \int_{\Theta} \mathbb{P}(\theta,x|D)d\mu_\Theta = \int_{\Theta} \mathbb{P}(x|\theta)\mathbb{P}(\theta|D)d\mu_\Theta
\end{equation}

This ideal approach is almost always intractable. Consequently, much research in the statistics literature is concerned with
computationally efficient approximations to this approach. Two of the most common approaches are either to approximate
the posterior by a distirbution from a specified class (this is called the variational Bayes' approach) or to design an
algorithm which generates (often approximately) a sample from the posterior distribution.

\subsection{Sample Mean Estimation Example}
\subsection{Application to Image Classification}
For the problem of image classification, we want to estimate more than just a single distribution. We are trying to
estimate a distribution over the labels for every possible image $I$.

The Bayesian setup is now similar, it consists of a parameters space $\Theta$, together with an initial probability
distribution over $\Theta\times X$ for every image $I$ (here $X$ is the set of possible labels). We make the same assumptions
as before, im particular for each image $I$ we obtain a density over $\Theta\times X$, wehich we write $\mathbb{P}(s,x|I)$.

We introduce the same notation as before, this time everything depends upond the image $I$. The marginal distributions
are denoted
$$\mathbb{P}(x|I) = \int_{\Theta} \mathbb{P}(\theta,x|I)d\mu_\Theta$$
$$\mathbb{P}(\theta|I) = \int_{X} \mathbb{P}(\theta,x|I)d\mu_x$$
and the conditional distributions as
$$\mathbb{P}(x|\theta,I) = \frac{\mathbb{P}(\theta,x|I)}{\mathbb{P}(\theta|I)}$$
$$\mathbb{P}(\theta|x,I) = \frac{\mathbb{P}(\theta,x|I)}{\mathbb{P}(x|I)}$$

We make one additional important assumption. We assume that in our model the marginal distribution 
$\mathbb{P}(\theta|I)$ is independent of $I$
and we remove the $I$ in our notation and write $\mathbb{P}(\theta)$. Essentially, we are assuming that our
parameter distribution is fixed independently of the image $I$. Note that this assumption does hold for our
Neural Network models.

As before, the question becomes how to update the distributions $\mathbb{P}(s,x|I)$ when data $D = \{(x_1,I_1),...,(x_N,D_N)\}$
are available. We choose to update this distribution so that $\mathbb{P}(x|\theta,I)$ remains fixed for all
$\theta$ and $I$, and also so that $\mathbb{P}(\theta)$ remains independent of $I$.

Similar to the previous setup, when we see the data point $(x_1,I_1)$ we update the distribution $\mathbb{P}(\theta)$
by setting it equal to $\mathbb{P}(\theta|x_1,I_1)$ and Bayes' formula gives
\begin{equation}
 \mathbb{P}(\theta|x_1,I_1) = \frac{\mathbb{P}(\theta,x_1|I_1)\mathbb{P}(\theta)}{\mathbb{P}(x|I)}
\end{equation}

One often omits the normalizing factor $\mathbb{P}(x|I)$ and simply writes 
\begin{equation}
 \mathbb{P}(\theta|x_1,I_1) \sim \mathbb{P}(\theta,x_1|I_1)\mathbb{P}(\theta)
\end{equation}
Iterating this, we see that the new distribution (called the Bayes' posterior distribution) after seeing the data
is given by
\begin{equation}
 \mathbb{P}(\theta|D) \sim \mathbb{P}(\theta,x_1|I_1)\mathbb{P}(\theta,x_2|I_2)\cdots\mathbb{P}(\theta,x_N|I_N)\mathbb{P}(\theta)
\end{equation}
To make a prediction given an image $I$ we marginalize out the parameters to obtain (recalling again that $\mathbb{P}(x|\theta,I)$
remains fixed)
\begin{equation}
 \mathbb{P}(x|I,D) = \int_{\Theta} \mathbb{P}(x|\theta,I)\mathbb{P}(\theta|D)d\mu_{\Theta}
\end{equation}

\section{Smoluchowski Dynamics}
The Smoluchowski dynamics, or `noisy gradient flow' is given by the SDE
\begin{equation}\label{Smoluchowski}
 dX_t = -\nabla V(x_t)dt + \sqrt{2\beta^{-1}} dW_t
\end{equation}
where $V$ is some potential (in our application the Neural Network loss function).

The Fokker-Planck equation corresponding to this dynamics is
\begin{equation}\label{Smoluchowski_FP}
 \frac{d}{dt}\rho = \nabla \cdot ((\nabla V)\rho + \beta^{-1}\nabla \rho)
\end{equation}
This means that if $X_0$ is distributed according to $\rho_0$, then $X_t$ will be distributed
according to $\rho_t$ which solves the above equation. We have the following convergence result.
\begin{theorem}

 Assume that $V$ is a confining potential, which means that $$e^{-\beta V(x)}\in L^1$$
 for all $\beta > 0$. 
 
 Then
 the solution to the Fokker-Planck equation (\ref{Smoluchowski_FP}) converges exponentially to the
 equilibrium distribution
 \begin{equation}\label{equilibrium_dist}
 \rho_{eq}(x) \sim e^{-\beta V(x)}
 \end{equation}
 which satisfies
 \begin{equation}\label{equilibrium_eq}
 (\nabla V)\rho_{eq} + \beta^{-1}\nabla \rho_{eq} = 0
 \end{equation}
 
 The convergence is in the following sense. If $\rho_0 = f_0\rho_{eq}$ with $f_0\in L^2(\rho_{eq}dx)$,
 then $\rho_t = f_t\rho_{eq}$ for $f_t\in L^2(\rho_{eq}dx)$ and 
 \begin{equation}\label{decay_rate}
  \|f_t - 1\|_{L^2(\rho_{eq}dx)} \leq e^{-\alpha t}\|f_0 - 1\|_{L^2(\rho_{eq}dx)}
 \end{equation}
 for some convergence rate $\alpha > 0$.
\end{theorem}

We won't formally prove this, but let us explain where this theorem comes from and how to determine the decay rate
$\alpha$.

First, a simple calculation shows that $\rho_{eq}$ satisfies equation (\ref{equilibrium_eq}). Namely, the chain rule gives
$$\nabla \rho_{eq} = -\beta (\nabla V)\rho_{eq}$$
regardless of the normalizing factor. This implies that $\rho_{eq}$ is stationary, i.e. that by equation (\ref{Smoluchowski_FP}),
$$\left.\frac{d}{dt}\rho\right|_{\rho_{eq}} = 0$$

To understand the decay rate, let $\rho_0 = f_0\rho_{eq}$ (here the subscript denotes the time $t$) with $f_0\in L^2(\rho_{eq}dx)$.
This space is the Hilbert space generated by the inner product
\begin{equation}
 \langle f,g\rangle_{L^2(\rho_{eq}dx)} = \int f(x)g(x)\rho_{eq}(x)dx
\end{equation}

We now use the equation (\ref{Smoluchowski_FP}) to derive an equation for $f_t$. We obtain (using equation (\ref{equilibrium_eq}))
\begin{equation}
 \frac{d}{dt} f = \nabla V\cdot\nabla f + \beta^{-1}\Delta f = Lf
\end{equation}
where we have set $L = \nabla V\cdot\nabla  + \beta^{-1}\Delta$.

A similar calculation shows that $L$ is self-adjoint with respect to the inner product 
$\langle \cdot, \cdot\rangle_{L^2(\rho_{eq}dx)}$. The decay rate to equilibrium can now be determined from 
the spectrum of $L$. These calculations can be explicitly carried out for a quadratic objective, which we will
do later in this section.

\subsection{Relation to Stochastic Gradient Descent}
Stochastic gradient descent is given by the following iteration
\begin{equation}\label{sgd}
 x_{n+1} = x_n - s_n(\nabla V(x_n) + \nu_n)
\end{equation}
where $V$ is the objective we are trying to minimize, $s_n$ is the step size, and $\nu_n$ is a random variable
representing the noise in our gradient sample. If the random variable $\nu_n$ satisfies
\begin{equation}
 \mathbb{E}(\nu_n) = 0,~\mathbb{E}(\nu_n^2) = \frac{2\beta^{-1}}{\sqrt{s_n}}
\end{equation}
then the iteration (\ref{sgd}) converges to the dynamics (\ref{Smoluchowski}) by Donsker's theorem.

If we set $V$ to be the average log-likelihood loss function
\begin{equation}
 V(\theta) = -\frac{1}{N}\displaystyle\sum_{j = 1}^N \log(\mathbb{P}(x_j|\theta,I_j))
\end{equation}
and set the inverse temperature $\beta = N$, then the equilibrium distribution corresponding to the
dynamics (\ref{Smoluchowski}) is exactly the Bayes Posterior distribution (with improper prior)
$$\mathbb{P}(\theta|D) \sim \mathbb\prod_{j = 1}^N \log(\mathbb{P}(x_j|\theta,I_j))$$

So we see that the relationship between the step size and the noise in our gradient sample which produces
an approximate Bayes posterior sample is
\begin{equation}
 \mathbb{E}(\nu_n^2)\sqrt{s_n} = \frac{2}{N}
\end{equation}
where $N$ is the number of samples. This helps explain many phenomena, for instance why taking the step size too small produces
worse results and why increasing the minibatch size (which decreases the sample variance) allows for a greater step size.

\section{Langevin Dynamics}
\section{Simulated Annealing}
Given a function $\phi(x)$, we want to find the global minimum of this funcion, that is, we need to find $x^*=argmin(\phi(x))$
\subsection{Theorem}
For stochastic process $X^T_t$ which satisfies a stochastic differential equation 
$$dX_t^T=-\frac{1}{\gamma}\nabla\phi(X_t^T)dt+\frac{\sigma}{\gamma} dW_t$$
where $T=\frac{\sigma^2}{2\gamma}$ and $\sigma$, $\gamma$ are constant.
~\\We have 
$$\mathop{lim}\limits_{t\rightarrow\infty, T\rightarrow 0}E(X^T_t)=x^*$$
where $x^*=argmin (\phi(x))$
\subsection{Proof}For a stochastic differential equation
$$dX_t=A(X_t,t)dt+\sigma(X_t,t)dW_t \eqno{(1)}$$
By Kolmogorov forward equation, we have the transition probability $P(x,t;x',0)$ satisfies 
$$\frac{\partial}{\partial t}P(x,t;x',0)=L_t^{+}P(x,t;x',0) \eqno{(2)}$$
where $L_t=A(X_t,t)\frac{\partial}{\partial x}+\frac{1}{2}\sigma^2(X_t,t)\frac{\partial^2}{\partial x^2}$ and $L_t^+$ is the adjoint of $L_t$.
~\\By the definition of probability density function w.r.t. transition probability:
$$f(x,t)=\int P(x,t;x',0)P(x')dx' \eqno{(3)}$$
where $P(x')$ is the density function at $t=0$.
~\\Combine equation (3) and (2), by simple computation, we get Fokker-Planck equation:
$$\frac{\partial}{\partial t}f(x,t)=L_t^{+}f(x,t) \eqno{(4)}$$
Now we set $A(x,t)=-\frac{1}{\gamma}\nabla\phi(x)$ and $\sigma(x,t)=\frac{\sigma}{\gamma}$ (constant)
~\\Then the Fokker-Planck equation [or equation (4)] becomes:
$$\frac{\partial f}{\partial t}=\frac{\partial}{\partial x}[\frac{1}{\gamma}\nabla\phi(x)\cdot f]+\frac{1}{2}\frac{\partial^2}{\partial x^2}(\frac{\sigma^2}{\gamma^2} f) \eqno(5)$$
When it comes to a steady state, that is, the density function would not change according to time:
$$\frac{\partial f}{\partial t}=0 \eqno(6)$$
(In physics, we often set $t\rightarrow \infty$ to represent the steady state)
~\\Combine equation (5) and (6), we get 
$$\frac{\partial}{\partial x}[\frac{1}{\gamma}\nabla\phi(x)\cdot f]+\frac{1}{2}\frac{\partial^2}{\partial x^2}(\frac{\sigma^2}{\gamma^2} f)=0 \eqno(7)$$
Solve this ODE, we have:
$$f=\frac{e^{-\frac{2\gamma\phi(x)}{\sigma^2}}}{\int e^{-\frac{2\gamma\phi(x)}{\sigma^2}}dx} \eqno(8)$$
We define $T=\frac{\sigma^2}{2\gamma}$ as the temperature. We call this steady state distribution equilibrium distribution at temperature T. So just by changing variables, we rewrite equation (8):
$$f^T_{eq}=\frac{e^{-\frac{\phi(x)}{T}}}{\int e^{-\frac{\phi(x)}{T}}dx} \eqno(9)$$
We denote $X_t^T$ as the stochastic process which satisfies
$$dX_t^T=-\frac{1}{\gamma}\nabla\phi(X_t^T)dt+\frac{\sigma}{\gamma} dW_t \eqno(10)$$
where $T=\frac{\sigma^2}{2\gamma}$ [Because we have set $A(x,t)=-\frac{1}{\gamma}\nabla\phi(x)$ and $\sigma(x,t)=\frac{\sigma}{\gamma}$ (constant)]
~\\By previous argument, the density function of random variable $X_{\infty}^T$ is $f^T_{eq}$
~\\By Laplace Method, which says that:
~\\For  $\forall g\in C^{\infty}(\mathbb{R})$, we have 
$$\int g\cdot f^T_{eq}\rightarrow \int g\cdot \delta(x-x^*)=g(x^*)~\ as ~\ T\rightarrow 0 \eqno(11)$$
where $x^*=argmin (\phi(x))$
~\\Because $g(x)=x\in C^{\infty}(\mathbb{R})$, we immediately get:
$$E(X^T_\infty)=\int x\cdot f^T_{eq}\rightarrow\int x\cdot \delta(x-x^*)=x^* ~\ as ~\ T\rightarrow 0 \eqno(12)$$
So we get:
$$\mathop{lim}\limits_{t\rightarrow\infty, T\rightarrow 0}E(X^T_t)=x^* \eqno(13)$$
So the basic idea of simulated annealing is to approximate $\mathop{lim}\limits_{t\rightarrow\infty, T\rightarrow 0}E(X^T_t)$
\subsection{Algorithm of simulated annealing}
Fix T=1, we fix a point $x_0$ and evolve the stochastic differential equation $dX_t^T=-\frac{1}{\gamma}\nabla\phi(X_t^T)dt+\frac{\sigma}{\gamma} dW_t$ [equation(10)]. When it comes to the steady state $(t>>1)$, we start sampling $\{x_1, x_2, x_3, ..., x_n\}$
when n is large enough, we get $E(X^1_\infty)\approx\frac{1}{n}\sum\limits^n_{i=1}x_i$
~\\Fix T=0.1, we set $x_0=\frac{1}{n}\sum\limits^n_{i=1}x_i$ and evolve the stochastic differential equation $dX_t^T=-\frac{1}{\gamma}\nabla\phi(X_t^T)dt+\frac{\sigma}{\gamma} dW_t$ [equation(10)]. When it comes to the steady state $(t>>1)$, we start sampling $\{x_1, x_2, x_3, ..., x_n\}$
when n is large enough, we get $E(X^{0.1}_\infty)\approx\frac{1}{n}\sum\limits^n_{i=1}x_i$.
~\\We keep lowering the temperature.
~\\Finally, we get $T=\epsilon$ ($\epsilon<<1$). After doing sampling $\{x_1, x_2, x_3, ..., x_n\}$, we have $\frac{1}{n}\sum\limits^n_{i=1}x_i\approx E(X^{\epsilon}_\infty)\approx x^*$


\subsection{From SGD (minibatch) to SA}
We first consider the following lost function 
$$ 
F(x) = \frac{1}{N} \sum_{i=1}^N f_i(x),
$$ 
where $N$ is the number of data point. The Minibatch SGD is written
as 
\begin{equation} \label{equ:minibatch}
x^{n+1} = x^n - \Delta t_{n} \nabla F_{B_n}(x^n),
\end{equation}
where 
$$ 
F_{B_n}(x) = \frac{1}{|B_n|} \sum_{j \in B_n} f_j(x).
$$ 
Now, we rewrite \eqref{equ:minibatch} as 
\begin{equation} \label{equ:minibach-SA}
X^{n+1} = X^n - \Delta t_{n} \nabla F(X^n) + \Delta {t}_n
\underbrace{[\nabla F(X^n) - \nabla F_{B_n}(X^n)]}_{Y^n}. 
\end{equation}

\begin{equation} \label{equ:minibach-SA}
X^{n+1} = X^n - a_{n} \nabla F(X^n) + a_n
[\nabla F(X^n) - \nabla F_{B_n}(X^n)-\frac{b_n}{a_n}] + b_nW_n 
\end{equation}
\begin{equation} \label{equ:minibach-SA}
X^{n+1} = X^n - a_{n} [\nabla F(X^n) + \xi_n] + b_nW_n 
\end{equation}
where
$$
a_n=\Delta t_n, \xi_n=\nabla F(X^n) - \nabla F_{B_n}(X^n)-\frac{b_n}{a_n}W_n.
$$

Let $p_i$ is the probability of choosing the $i$-th component to
the minibatch. We note here that each component may be in the
minibatch for more than once. Then, we have 
$$ 
\begin{aligned}
\mathbb{E}(Y^n) &= \nabla F(X^n) - \mathbb{E} [\nabla F_{B_n}(X^n)] \\
&= \frac{1}{N} \sum_{i=1}^N \nabla f_i(X^n) - \frac{1}{|B_n|}
\mathbb{E}[\sum_{j\in B_n} \nabla f_j(X^n)] \\
&= \frac{1}{N} \sum_{i=1}^N \nabla f_i(X^n) - \mathbb{E}[\nabla
f(X^n)] \qquad (\text{independence}) \\
&= \frac{1}{N} \sum_{i=1}^N \nabla f_i(X^n) - \sum_{i=1}^N p_i \nabla
f_i(X^n).
\end{aligned}
$$ 
Therefore, we have the following condition
\begin{equation} \label{equ:mean-condition}
\frac{1}{N}\sum_{i=1}^N \nabla f_i(x^n) = \mathbb{E}[\nabla f(X^n)].
\end{equation}
From the above condition, a simple choice is $p_i = \frac{1}{N}$,
namely uniform distribution. Condition \eqref{equ:mean-condition}
can also be written as $\mathbb{E}(Y^n) = 0$.

Note that ${\rm Var}(Y^n) = \mathbb{E}[Y^n (Y^n)^T] - \mathbb{E}(Y^n)
\mathbb{E}(Y^n)^T = \mathbb{E}[Y^n (Y^n)^T]$. For simplicity,
we remove the dependence of $X^n$ in the following. Then, 
$$ 
\begin{aligned}
{\rm Var}(Y^n) &= \mathbb{E}[ (\nabla F - \nabla F_{B_n})
(\nabla F - \nabla F_{B_n})^T ]
\\
&= \mathbb{E}[\nabla F_{B_n} (\nabla F_{B_n})^T ] - \nabla F (\nabla
F)^T \\
&= \frac{1}{|B_n|^2} \mathbb{E}_{i_1} \mathbb{E}_{i_1} \cdots
\mathbb{E}_{i_{|B_n|}}[ (\sum_{j=1}^{|B_n|} \nabla f_{i_j})
(\sum_{j=1}^{|B_n|} \nabla f_{i_j})^T ] - \nabla F (\nabla F)^T \\
&= \frac{1}{|B_n|^2} \mathbb{E}_{i_1} \mathbb{E}_{i_2} \cdots
\mathbb{E}_{i_{|B_n|}}[ \sum_{j,r=1 , i\neq j}^{|B_n|} \nabla f_{i_j}
(\nabla f_{i_r})^T] \\ 
& ~~ + \frac{1}{|B_n|^2} \mathbb{E}_{i_1} \mathbb{E}_{i_2} \cdots
\mathbb{E}_{i_{|B_n|}} [\sum_{j=1}^{|B_n|} \nabla
f_{i_j} (\nabla f_{i_j})^T]- \nabla F (\nabla F)^T \\ 
& = (1 - \frac{1}{|B_n|}) \mathbb{E}(\nabla f) [\mathbb{E}(\nabla
f)]^T + \frac{1}{|B_n|} \mathbb{E}[ (\nabla f) (\nabla f)^T] -
\nabla F (\nabla F)^T \\
&= \frac{1}{|B_n|} \left( \mathbb{E}[(\nabla f) (\nabla f)^T] -
\mathbb{E}(\nabla f) [\mathbb{E}(\nabla f)^T] 
\right) \\
&= \frac{1}{|B_n|} {\rm Var}(\nabla f).
\end{aligned}
$$ 

\begin{remark}
	Since the index here can be repeated, there is NO need that ${\rm
		Var}(Y^n) = 0$ when $|B_n| = N$.
\end{remark}

In summary, we have 
\begin{equation} \label{equ:summary}
\mathbb{E}(Y^n) = 0, \qquad {\rm Var}(Y^n) = \frac{1}{|B_n|} {\rm
	Var}(\nabla f).
\end{equation}
It is conceivable that the result can be extended to the expectation
version of the lost function, namely, 
$$ 
F(x) = \mathbb{E}_{\xi} f(x, \xi).
$$ 


\subsection{Convergence result}
In the SA scheme, 
\begin{equation} \label{equ:SA}
X^{n+1} = X^n - \Delta t_n \nabla F(X^n) + \sigma_n W_{\Delta t_n},
\end{equation}
the convergence is proven under the following condition: 
$$ 
\Delta t_n = \mathcal{O}(n), \qquad \sigma_n^2 =
\mathcal{O}(\frac{1}{n \log\log n})
$$ 
Compare the variance of $\sigma_n W_{\Delta t_n}$ and $\Delta t_n
Y^n$, we have 
$$ 
\sigma_n^2 \Delta t_n \approx \Delta t_n^2 {\rm Var}(Y^n),
$$ 
which implies that 
$$ 
\frac{1}{|B_n|} {\rm Var}(\nabla f) =
\mathcal{O}(\frac{\sigma_n^2}{\Delta t_n}) =
\mathcal{O}(\frac{1}{\log\log n}).
$$ 
We conclude that, if the SGD is stagnated, namely ${\rm Var}(\nabla
f)$ stays the almost the same, we need to slowly increase the size of
minibatch in the way that 
\begin{equation} \label{equ:batch-size}
|B_n| = \mathcal{O}(\log\log n).
\end{equation}


\begin{remark}
	The above argument requires that $\nabla f(x, \xi)$ is Gaussian on
	$\xi$ for any $x$. This assumption holds for linear Gaussian
	regression. 
\end{remark}
