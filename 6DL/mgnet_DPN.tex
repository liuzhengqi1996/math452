\subsection{Dual Path Network (DPN)}
The general idea of DPN model \cite{chen2017dual} can be described as
\begin{equation}\label{eq:DPN}
\begin{cases}
f^{1,0} &=R_{\rm max}\circ \sigma \circ \theta^0(f), \\
\text{\bf For} &\ell = 1:J \\
\quad &\text{\bf For} \quad i = 1:\nu_\ell \\
&\tilde{f}^{\ell,i} = \tilde{\mathcal{H}}^{\ell,i} \left(  [f^{\ell,0}, \cdots, f^{\ell,i-1}]  \right), 
\quad  \bar{f}^{\ell,i} = \bar{\mathcal{H}}^{\ell,i} \left([f^{\ell,0}, \cdots, f^{\ell,i-1}] \right),\\ 
&f^{\ell,i} =  \tilde f^{\ell,i} +  \bar f^{\ell,i} \\
\quad &\text{\bf EndFor} \\
&\tilde f^{\ell+1,0} =  \tilde R_\ell^{\ell+1} ( f^{\ell, \nu_\ell} ) ,  \quad \bar f^{\ell+1,0} =  \bar R_\ell^{\ell+1} ( f^{\ell, \nu_\ell} ) ,\\
&f^{\ell,0} =  \tilde f^{\ell,0} +  \bar f^{\ell,0} \\
\text{\bf End} &\\
H_0(f) &=  R_{\rm ave}(f^{L,\nu_\ell}). \\
\end{cases}
\end{equation}

Here, if we take 
\begin{equation}
\tilde{\mathcal{H}}^{\ell,i} =  \bar{\mathcal{H}}^{\ell,i},
\end{equation}
with some special forms such as classical CNN or ResNet, this is
similar to the trick in AlexNet.

In DPN, the trick is to take different forms for $\tilde{\mathcal{H}}^{\ell,i} $ and $ \bar{\mathcal{H}}^{\ell,i}$.
For example, they take
\begin{equation}\label{eq:DPN-1}
\tilde{\mathcal{H}}^{\ell,i} \left(  [f^{\ell,0}, \cdots, f^{\ell,i-1}]  \right) = \sigma( f^{\ell,i-1} + \xi^{\ell,i} \circ \sigma \circ \eta^{\ell,i}(f^{\ell,i-1})),
\end{equation}
and 
\begin{equation}\label{eq:DPN-2}
\bar{\mathcal{H}}^{\ell,i} \left(  [f^{\ell,0}, \cdots, f^{\ell,i-1}]  \right) = \sigma \left( \sum_{j=0}^{i-1} [\theta^{\ell,i}]_{j} f^{\ell,j} \right)
\end{equation}
in original DPN paper.
Here we need to mention that, equation \eqref{eq:DPN-1} mens that $\tilde f^{\ell,i}$ follows the
process of ResNet structure and equation \eqref{eq:DPN-2} mens that $\bar f^{\ell,i}$ follows the
process of DenseNet structure. That is to say, DPN is some special combination of these two models.

\paragraph{The connection to MgNet}
First, we can make a extended version for DPN as:
\begin{equation}\label{eq:DPN}
\begin{cases}
f^{1,0} &=R_{\rm max}\circ \sigma \circ \theta^0(f), \\
\text{\bf For} &\ell = 1:J \\
\quad &\text{\bf For} \quad i = 1:\nu_\ell \\
&\tilde{f}^{\ell,i} = \tilde{\mathcal{H}}^{\ell,i} \left(  f^{\ell,i-1} \right), 
\quad  \bar{f}^{\ell,i} = \bar{\mathcal{H}}^{\ell,i} \left( f^{\ell,i-1} \right),\\ 
&f^{\ell,i} =  \red{[\tilde f^{\ell,i} ,\bar f^{\ell,i}]}\\
\quad &\text{\bf EndFor} \\
&\tilde f^{\ell+1,0} =  \tilde R_\ell^{\ell+1} ( f^{\ell, \nu_\ell} ) ,  \quad \bar f^{\ell+1,0} =  \bar R_\ell^{\ell+1} ( f^{\ell, \nu_\ell} ) ,\\
&f^{\ell,0} =  \tilde f^{\ell,0} +  \bar f^{\ell,0} \\
\text{\bf End} &\\
H_0(f) &=  R_{\rm ave}(f^{L,\nu_\ell}). \\
\end{cases}
\end{equation}
Then, we can take some special form of $ \tilde{\mathcal{H}}$,
$\bar{\mathcal{H}}^{\ell,i}$, $\tilde R_\ell^{\ell+1}$ and $ \bar R_\ell^{\ell+1}$
to connect our MgNet. 
\begin{itemize}
\item recover $u^{\ell,i}$ for $i = 1:\nu_\ell$
\begin{equation}
\blue{\tilde f^{\ell,i}} = \tilde{\mathcal{H}}^{\ell,i} \left(  f^{\ell,i-1} \right) = \tilde{\mathcal{H}}^{\ell,i} \left(  [\tilde f^{\ell,i} ,\bar f^{\ell,i}] \right) = 
\blue{\tilde f^{\ell,i-1} + B_{\ell,i}  ({\bar f^{\ell,i} -  A^{\ell} (\tilde f^{\ell,i-1})})},
\end{equation}
\item recover $f^{\ell}$
\begin{equation}
\blue{\bar f^{\ell,i}} = \bar{\mathcal{H}}^{\ell,i} \left(  f^{\ell,i-1} \right) = \bar{\mathcal{H}}^{\ell,i} \left(  [\tilde f^{\ell,i} ,\bar f^{\ell,i}] \right) = 
\blue{\bar f^{\ell,i-1}} ,
\end{equation}
\item recover $u^{\ell,0}$
\begin{equation}
\blue{\tilde f^{\ell+1,0}} = \tilde f^{\ell+1,0} =  \tilde R_\ell^{\ell+1} ( f^{\ell, \nu_\ell} ) = \tilde f^{\ell+1,0} =  
\tilde R_\ell^{\ell+1} ( [\tilde f^{\ell,\nu_\ell} ,\bar f^{\ell,\nu_\ell}]  )  = \blue{\Pi_\ell^{\ell+1} \tilde f^{\ell,\nu_\ell}},
\end{equation}
\item recover $f^{\ell+1}$
\begin{equation}
\begin{aligned}
\blue{\bar f^{\ell+1,0}} =  \bar R_\ell^{\ell+1} ( f^{\ell, \nu_\ell} ) 
&=  \bar R_\ell^{\ell+1} ( [\tilde f^{\ell,\nu_\ell} ,\bar f^{\ell,\nu_\ell}]  ),  \\
&= \blue{R^{\ell+1}_\ell( \bar f^{\ell,\nu_\ell} - A^\ell(\tilde f^{\ell,\nu_\ell})) + A^{\ell+1} (\Pi_\ell^{\ell+1} \tilde f^{\ell,\nu_\ell})},\\
&= \blue{R^{\ell+1}_\ell( \bar f^{\ell,\nu_\ell} - A^\ell(\tilde f^{\ell,\nu_\ell})) + A^{\ell+1} (\tilde f^{\ell+1,0})}. \\
\end{aligned}
\end{equation}
\end{itemize}
This means that, we can connect the MgNet and DPN with the next relationship:
\begin{equation}
u^{\ell,i} = \tilde f^{\ell,i}, \quad i = 0:\nu_\ell, \quad \ell = 1:J,
\end{equation}
and
\begin{equation}
f^{\ell} = \bar f^{\ell,i},  \quad i = 0:\nu_\ell, \quad \ell = 1:J.
\end{equation}

Thus to say, we may give a better explanation for DPN from the viewpoint of MgNet.
Furthermore, we give some further developments version of DPN by involving more
structure from MgNet like what we are doing for ResNet and iResNet.

