\section{Convergence rates in Sobolev norms}
\subsection{Compactly supported activation function}
Given a bounded domain $\Omega\subset\mathbb R^d$, we consider the function
$$
f: \Omega\mapsto \mathbb R
$$
Let 
$$
f^e: \mathbb R^d\mapsto \mathbb R
$$ 
be any extension of $f$ so that
$$
f^e|_\Omega=f(x), \quad x\in \Omega. 
$$
Most time, we will drop the superscript $``e"$ to still use $f$ to
denote an extension of $f$. 

Consider the Fourier transform:
\begin{equation}
\label{Fourier}
\hat f(\omega)=\frac{1}{(2\pi)^d}\int_{\mathbb{R}^d}e^{-i\omega\cdot x}f(x)dx
\quad \forall \omega \in \mathbb R^d.
\end{equation}
Let $x_\Omega\in\Omega$ is such that
\begin{equation}
\label{xOmega}
x_\Omega\in \arg\min_{y\in\bar\Omega}(\max_{x\in\bar\Omega}|x-y|)
\end{equation}
We may call
\begin{equation}
\label{rOmega}
r_\Omega=\max_{x\in\bar\Omega}|x-x_\Omega|
\end{equation}
the radius of $\Omega$.


Using the Fourier inversion formula, we can have a Fourier
representation of $f(x)$ as follows
\begin{equation}
\label{eqn1}
f(x)=\int_{\mathbb{R}^d}e^{i\omega\cdot x}\hat{f}(\omega)d\omega
\quad \forall x \in \Omega,
\end{equation}
Let us write
\begin{equation}
\label{theta-omega}
\hat{f}(\omega)=e^{i\beta_1(\omega)}|\hat{f}(\omega)|.   
\end{equation}

%Define $\mu_f$ as a complex measure on $\mathbb{R}^d$ such that 
%\begin{equation}
%d\mu_f = e^{i\alpha(\omega)}|\hat{f}(\omega)|d\omega
%\end{equation}
%and
%\begin{equation}
%\label{norm}
%\|\mu_f\| = \int_{\mathbb{R}^d} (1+|\omega|)^2d|\mu_f|(\omega) 
%\end{equation}
%Also we have:
%\begin{equation}
%f(x)=\int_{\mathbb{R}^d}e^{i\omega\cdot x}d\mu_f
%\end{equation}


Let $\sigma$ be the activation function with compact support and bounded derivatives up to order $m$, that is
\begin{equation}
\|\sigma\|_{m,\infty} := \max_{0\le\alpha\le m}\sup_{x\in\mathbb{R}}|\sigma^{(\alpha)}(x)|\le \infty
\end{equation}
Set
\begin{equation}
\label{eq:3}
\tilde x=x-x_\Omega.  
\end{equation}
Choose $a\ne0$ such that $\hat{\sigma}(a)\ne 0$, by Fourier transform, we have:
\begin{equation*}
\hat{\sigma}(a)=\frac{1}{2\pi}\int_{\mathbb{R}}\sigma(y)e^{-ia
	y}dy=\frac{1}{2\pi}\int_{\mathbb{R}}\sigma(\omega\cdot \tilde
x+b)e^{-ia(\omega\cdot \tilde x+\theta)}db
\end{equation*}
and we can also write
\begin{equation*}
\hat{\sigma}(a)=e^{i\beta_2(a)}|\hat{\sigma}(a)|.   
\end{equation*}

Thus
\begin{equation}
|\hat{\sigma}(a)|=\frac{1}{2\pi}\int_{\mathbb{R}}\sigma((\omega\cdot \tilde{x}+b ) )e^{-ia(\omega\cdot \tilde{x}+b )-i\beta_2(a)}db.   
\end{equation}
%be the hat function defined as:
%\begin{equation*}
%\sigma(x)=\left\{
%\begin{aligned}
%&x,\qquad\ \  0\le x\le 1 \\
%&2-x,\quad 1\le x\le 2\\
%&0,\qquad\quad \mathrm{otherwise} 
%\end{aligned}
%\right.
%\end{equation*}

Then we can write
\begin{equation}
\begin{aligned}
f(x)=& \int_{\mathbb{R}^d}e^{i\omega\cdot x}e^{i\beta_1(\omega)}|\hat{f}(\omega)|d\omega\\
=&\int_{\mathbb{R}^d}|a|^d e^{ia(\omega\cdot x)}e^{i\beta_1(a\omega)}|\hat{f}(a\omega)|d\omega \\
=&\int_{\mathbb{R}^d}\frac{|a|^d }{ |\hat{\sigma}(a)|} |\hat{\sigma}(a)|e^{ia(\omega\cdot x)+i\beta_1(a\omega)}|\hat{f}(a\omega)|d\omega \\
=&\int_{\mathbb{R}^d}\int_{\mathbb{R}}\frac{|a|^d}{2\pi|\hat{\sigma}(a)|}\sigma(\omega\cdot \tilde{x}+b  )e^{-i\beta_2(a)}
e^{ia\omega\cdot x_\Omega-iab}db e^{i\beta_1(a\omega)}|\hat{f}(a\omega)|d\omega \\
=&\int_{\mathbb{R}^d}\int_{\mathbb{R}}\sigma(\omega\cdot x+b )\frac{|a|^d}{2\pi|\hat{\sigma}(a)|}e^{i\beta_3(a,\omega,b)}|\hat{f}(a\omega)|db d\omega
\end{aligned}
\end{equation}
where $\beta_3(a,\omega,b)=a\omega\cdot x_\Omega-ab-\beta_2(a)+\beta_1(a\omega)$.

Let
$$
\theta=(\omega, b).
$$
Since $f$ is real, we have
$$
f(x)={\rm Re}\; f(x) 
%\int_{\mathbb{R}^d}\int_{\mathbb{R}}\frac{\cos(\beta_1(a\omega)-ab-\beta_2(a))}{2\pi|\hat{\sigma}(a)|}\sigma(\omega\cdot
%\tilde x+b )|a|^d|\hat{f}(a\omega)|db d\omega
=\int_{\mathbb{R}^{d+1}}\kappa(\theta)\sigma(\omega\cdot\tilde x+b )|a|^d|\hat{f}(a\omega)|d\theta.
$$
where
\begin{equation}
\kappa(\theta)\equiv
\kappa(\omega,b)=\frac{\cos\beta_3(a,\omega,b)}{2\pi|\hat{\sigma}(a)|},
\end{equation}



Notice that $\sigma$ is compactly supported, assume that $\mathrm{supp}(\sigma)\subset[-M_1,M_1]$. Then we have:
\begin{equation}
\begin{aligned}
f(x) =\int_{\mathbb{R}^{d+1}}\kappa(\theta)\mathbf{1}_{D_M}\sigma(\omega\cdot \tilde{x}+b)|a|^d|\hat{f}(a\omega)|d\theta 
\end{aligned}
\end{equation}
where
\begin{equation}
M=\max(M_1,r_\Omega)
\end{equation}
\begin{equation}
D_M=\{\theta=(\omega,b):|b|\le(1+|\omega|)M  \}
\end{equation}

Define 
\begin{equation}
\gamma(f)=\int_{\mathbb{R}^{d+1}}\mathbf{1}_{D_M}(1+|\omega|)^m|a|^d|\hat{f}(a\omega)|d\theta 
\end{equation}

Now that
\begin{equation}
\begin{aligned}
f(x) 	&=\int_{\mathbb{R}^{d+1}}\kappa(\theta)\mathbf{1}_{D_M}\sigma(\omega\cdot \tilde{x}+b)|a|^d|\hat{f}(a\omega)|d\theta \\
&= \int_{\mathbb{R}^{d+1}}\frac{\kappa(\theta)\gamma(f)}{(1+|\omega|)^m}\sigma(\omega\cdot \tilde{x}+b)\frac{(1+|\omega|)^m\mathbf{1}_{D_M}|a|^d|\hat{f}(a\omega)|}{\gamma(f)}d\theta\\
&:=\int_{\mathbb{R}^{d+1}}\frac{\kappa(\theta)\gamma(f)}{(1+|\omega|)^m}\sigma(\omega\cdot \tilde{x}+b) d\lambda
\end{aligned}
\end{equation}
where

$$
d\lambda= \frac{(1+|\omega|)^m\mathbf{1}_{D_M}|a|^d|\hat{f}(a\omega)|}{\gamma(f)}d\theta ,\quad \int_{\mathbb{R}^{d+1}}d\lambda=1,
$$

and we can write
\begin{equation}
\tilde f(x)\equiv\frac{f(x)}{\gamma(f)}
=\mathbb{E}(g(\theta; x))
\end{equation}
where
\begin{equation}
\label{gx}
g(\theta; x))
=\frac{\kappa(\theta)}{(1+|\omega|)^m}\sigma(\omega\cdot\tilde x+b)) 
\end{equation}
And 
\begin{equation}
D^\alpha \tilde f(x)=\mathbb{E}(D^\alpha g).
\end{equation}
%\mathbb{E}(\kappa(\omega,b)\frac{D^\alpha\sigma(\omega\cdot\tilde x+b)}{(1+|\omega|)^m})
Let $\theta_i=(\omega_i,b_i)_{i=1}^n$ be independently drawn from the same distribution $\lambda$ and let 
$$
\bar g(\theta_1,\ldots,\theta_n)
=\frac{1}{n} \sum_{i=1}^ng(\theta_i, x).
$$
Then
$$
\bar{\mathbb E} \left(\sum_{|\alpha|\le m}(D^\alpha \tilde f-D^\alpha \bar
g)^2\right) 
=\sum_{|\alpha|\le m}\bar{\mathbb E} (D^\alpha \tilde f-D^\alpha \bar g)^2
=\sum_{|\alpha|\le m}\bar{\mathbb E} (\mathbb E(D^\alpha g)-D^\alpha \bar g)^2 
\le{1\over n}\sum_{|\alpha|\le m}\|D^\alpha g\|_\infty^2.
$$
Taking the $L^2$ norm (with a probability measure) on both side of the above inequality
and using Fubini's theorem,
\begin{equation}
\mathbb{\bar{\mathbb{E}}}(\|\tilde f(\cdot)-g(\theta,\cdot)\|^2_{H^m(\Omega)})
\le{1\over n}\sum_{|\alpha|\le m}\|D^\alpha g\|_\infty^2.
\end{equation}
Thus there exists $\theta_i=(\omega_i^*,b_i^*)_{i=1}^n$ and $\beta_i$ such that 
$$
\|\tilde f-\frac{1}{n}\sum_{i=1}^{n} g(\theta_i; \cdot)\|_m^2
\le{1\over n}\sum_{|\alpha|\le m}\|D^\alpha g\|_\infty^2.
$$
$$
\|f-f_n\|_m^2
\le{\gamma(f)^2\over n}
\sum_{|\alpha|\le m}\|D^\alpha g\|_\infty^2.
$$
where
$$
f_n(x)=\frac{\gamma(f)}{n}\sum_{i=1}^{n} g(\theta_i; \cdot)=\frac{1}{n} \sum_{i=1}^n a_i\sigma(\omega_i^*\cdot\tilde{x}+b_i^*).
$$
where 
$$
a_i= \frac{\gamma(f)\kappa(\theta_i^*)}{(1+|\omega_i^*|)^m} \le \frac{\gamma(f)}{2\pi|\hat{\sigma}(a)|}
$$

Noting that $\frac{1}{1+|\omega|}\le1$, we have
$$
\|D^\alpha g\|_{\infty}=
\max_{0\le|\alpha|\le m}\max_{\in\mathbb{R}^d,\theta\in\mathbb{R}^{d+1}}|\kappa(\theta)|\frac{|D^\alpha\sigma(\omega\cdot\tilde x+b)|}{(1+|\omega|)^m}\le \frac{\|\sigma\|_{m,\infty}}{2\pi|\hat{\sigma}(a)|}
$$
Then we have:

\begin{equation}
\begin{aligned}
\gamma(f)&=|a|^d\int_{\mathbb{R}^d}(1+|\omega|)^m \int_{-(1+|\omega|)M}^{(1+|\omega|)M}db |\hat{f}(a\omega)|d\omega \\
&= |a|^d\int_{\mathbb{R}^d}2M(1+|\omega|)^{m+1} |\hat{f}(a\omega)|d\omega\\
&\le 2M \int_{\mathbb{R}^d}(1+|\omega/a|)^{m+1} |\hat{f}(\omega)|d(\omega)\\
\end{aligned}
\end{equation}

Denote
$$
\|f\|_{m+1} = \int_{\mathbb{R}^d}(1+|\omega/a|)^{m+1} |\hat{f}(\omega)|d(\omega)
$$
Then there exists $(\omega_i^*,b_i^*)_{i=1}^n$ and $\beta_i$ such that 
$$
\|f-f_n\|_m
\le\frac{\|\sigma\|_{m,\infty}}{\sqrt{n}}\|f\|_{m+1} 
$$
where $C=\frac{M\sqrt{C_{m+d}^m}}{\pi|\hat{\sigma}(a)|} $.

\subsection{Periodic activation function}

Now $\sigma$ is periodic with period $T$. Then we can write the Fourier series for $\sigma$:
$$
\sigma(x) =  \sum_{i=-\infty}^{\infty}c_n e^{i\frac{2\pi nx}{T}}
$$
choose $n_1$ such that $c_{n_1}\ne 0$, we also have:
$$
c_{n_1} = \frac{1}{T}\int_{0}^{T}\sigma(x) e^{-i\frac{2\pi n_1x}{T}}dx= |c_{n_1}|e^{i\beta_2(c_{n_1})}
$$
Then for $f$:
\begin{equation}
\begin{aligned}
f(x)=& \int_{\mathbb{R}^d}e^{i\omega\cdot x}e^{i\beta_1(\omega)}|\hat{f}(\omega)|d\omega\\
=&\int_{\mathbb{R}^d}\frac{1}{c_{n_1}}\frac{1}{T}\int_{x_0}^{x_0+T}\sigma(t) e^{-i\frac{2\pi n_1t}{T}}dte^{i(\omega\cdot x)+i\beta_1(\omega)}|\hat{f}(\omega)|d\omega \\
=&\int_{\mathbb{R}^d}\frac{1}{|c_{n_1}|e^{i\beta_2(c_{n_1})}}\frac{1}{2\pi n_1}\int_{0}^{2\pi n_1}\sigma(\frac{T}{2\pi n_1}(\omega\cdot x+b)) e^{-i(\omega\cdot x+b)}dbe^{i(\omega\cdot x)+i\beta_1(\omega)}|\hat{f}(\omega)|d\omega \\
=&\int_{\mathbb{R}^d}\frac{1}{2|c_{n_1}|\pi n_1}\int_{0}^{2\pi n_1}\sigma(\frac{T}{2\pi n_1}(\omega\cdot x+b)) e^{i(\beta_1(\omega)-b-\beta_2(c_{n_1})}db|\hat{f}(\omega)|d\omega \\
=&\int_{\mathbb{R}^d}\int_{0}^{2\pi n_1}\kappa(b,\omega)\sigma(\frac{T}{2\pi n_1}(\omega\cdot x+b)) db|\hat{f}(\omega)|d\omega \\
\end{aligned}
\end{equation}
here we do a substitution $t=\frac{T}{2\pi n_1}(\omega\cdot x+b)$ and 
\begin{equation}
\kappa = \frac{\cos(\beta_1(\omega)-b-\beta_2(c_{n_1}))}{2|c_{n_1}|\pi n_1}
\end{equation}
which is then the same as what we did for compactly supported activation functions.
\begin{equation}
\begin{aligned}
f(x)=&\int_{\mathbb{R}^d}\int_{0}^{2\pi n_1}\kappa(b,\omega)\sigma(\frac{T}{2\pi n_1}(\omega\cdot x+b)) db|\hat{f}(\omega)|d\omega \\
=&\int_{\mathbb{R}^d\times[0,2\pi n_1]}\frac{\kappa(\theta)}{(1+|\omega|)^m}\sigma(\frac{T}{2\pi n_1}(\omega\cdot x+b))(1+|\omega|)^m|\hat{f}(\omega)|d\theta \\
=&\int_{\mathbb{R}^d\times[0,2\pi n_1]}\frac{\kappa(\theta)\gamma(f)}{(1+|\omega|)^m}\sigma(\frac{T}{2\pi n_1}(\omega\cdot x+b))\frac{(1+|\omega|)^m|\hat{f}(\omega)|}{\gamma(f)}d\theta \\
:=&\int_{\mathbb{R}^{d+1}}\frac{\kappa(\theta)\gamma(f)}{(1+|\omega|)^m}\sigma(\frac{T}{2\pi n_1}(\omega\cdot x+b))d\lambda
\end{aligned}
\end{equation}
where
$$
\gamma(f)=\int_{\mathbb{R}^{d+1}}\mathbf{1}_{0\le b\le 2\pi n_1}(1+|\omega|)^m|\hat{f}(\omega)|d\theta =2\pi n_1\int_{\mathbb{R}^d}(1+|\omega|)^m|\hat{f}(\omega)|d\omega
$$
and
$$
d\lambda= \frac{(1+|\omega|)^m|\hat{f}(\omega)|\mathbf{1}_{0\le b\le 2\pi n_1}}{\gamma(f)}d\theta,\quad \int_{\mathbb{R}^{d+1}}d\lambda=1,
$$


We can write
\begin{equation}
\tilde f(x)\equiv\frac{f(x)}{\gamma(f)}
=\mathbb{E}(g(\theta; x))
\end{equation}
where
\begin{equation}
\label{gx}
g(\theta; x))
=\frac{\kappa(\theta)}{(1+|\omega|)^m}\sigma(\frac{T}{2\pi n_1}(\omega\cdot x+b))
\end{equation}
And 
\begin{equation}
D^\alpha \tilde f(x)=\mathbb{E}(D^\alpha g).
\end{equation}
%\mathbb{E}(\kappa(\omega,b)\frac{D^\alpha\sigma(\omega\cdot\tilde x+b)}{(1+|\omega|)^m})
Let $\theta_i=(\omega_i,b_i)_{i=1}^n$ be independently drawn from the same distribution $\lambda$ and let 
$$
\bar g(\theta_1,\ldots,\theta_n)
=\frac{1}{n} \sum_{i=1}^ng(\theta_i, x).
$$
Then
$$
\bar{\mathbb E} \left(\sum_{|\alpha|\le m}(D^\alpha \tilde f-D^\alpha \bar
g)^2\right) 
=\sum_{|\alpha|\le m}\bar{\mathbb E} (D^\alpha \tilde f-D^\alpha \bar g)^2
=\sum_{|\alpha|\le m}\bar{\mathbb E} (\mathbb E(D^\alpha g)-D^\alpha \bar g)^2 
\le{1\over n}\sum_{|\alpha|\le m}\|D^\alpha g\|_\infty^2.
$$
Taking the $L^2$ norm (with a probability measure) on both side of the above inequality
and using Fubini's theorem,
\begin{equation}
\mathbb{\bar{\mathbb{E}}}(\|\tilde f(\cdot)-g(\theta,\cdot)\|^2_{H^m(\Omega)})
\le{1\over n}\sum_{|\alpha|\le m}\|D^\alpha g\|_\infty^2.
\end{equation}
Thus there exists $\theta_i=(\omega_i^*,b_i^*)_{i=1}^n$ and $\beta_i$ such that 
$$
\|\tilde f-\frac{1}{n}\sum_{i=1}^{n} g(\theta_i; \cdot)\|_m^2
\le{1\over n}\sum_{|\alpha|\le m}\|D^\alpha g\|_\infty^2.
$$
$$
\|f-f_n\|_m^2
\le{\gamma(f)^2\over n}
\sum_{|\alpha|\le m}\|D^\alpha g\|_\infty^2.
$$
where
$$
f_n(x)=\frac{\gamma(f)}{n}\sum_{i=1}^{n} g(\theta_i; \cdot)=\frac{1}{n} \sum_{i=1}^n a_i\sigma(\frac{T}{2\pi n_1}(\omega_i^*\cdot x+b_i^*)).
$$
where 
$$
a_i= \frac{\gamma(f)\kappa(\theta_i^*)}{(1+|\omega_i^*|)^m} \le \frac{\gamma(f)}{2|c_{n_1}|\pi n_1}
$$

Noting that $\frac{1}{1+|\omega|}\le1$, we have
$$
\|D^\alpha g\|_{\infty}=
\max_{0\le|\alpha|\le m}\max_{\in\mathbb{R}^d,\theta\in\mathbb{R}^{d+1}}|\kappa(\theta)|\frac{|D^\alpha\sigma( \frac{T}{2\pi n_1}(\omega\cdot x+b) )|}{(1+|\omega|)^m}\le \frac{k\|\sigma\|_{m,\infty}}{2|c_{n_1}|\pi n_1}
$$
where
$$
k = \max(\frac{T}{2\pi n_1},(\frac{T}{2\pi n_1})^\alpha)
$$


Denote
$$
\||f\||_{m} = \int_{\mathbb{R}^d}(1+|\omega|)^{m} |\hat{f}(\omega)|d(\omega)
$$
Then there exists $(\omega_i^*,b_i^*)_{i=1}^n$ and $\beta_i$ such that 
$$
\|f-f_n\|_m
\le\frac{\|\sigma\|_{m,\infty}}{\sqrt{n}}|\|f\||_{m} 
$$
where $C=\frac{k\sqrt{C_{m+d}^m}}{2|c_{n_1}|\pi n_1} $.


\subsubsection{Exponential Decay}
If there exists $\eta>0$ such that for all $0\le k \le m$
$$
|e^{\eta|t|}D^{k}\sigma(t)|_{L^1(t)}\le \zeta<\infty,
$$
which means that $|e^{\eta|t|}D^{k}\sigma(t)|$ is finite for all $t$ and all $0\le k \le m$, with abuse of notation here, we still use $\zeta$ to denote the bound.
$$
e^{\eta|t|}|D^{k}\sigma(t)|\le \zeta,
$$

We have
\begin{equation}
	\begin{aligned}
		&f(x)=\int_{\mathbb{R}^{d}}\int_{\mathbb{R}}\kappa(\omega,b)\sigma(\omega\cdot x+b)|a|^d|\hat{f}(a\omega)|dbd\omega.\\
		=&\int_{\mathbb{R}^{d}}\int_{\mathbb{R}}\frac{\kappa(\omega,b)}{(1+|\omega|)^m}\sigma(\omega\cdot x+b)(1+|\omega|)^m|a|^d|\hat{f}(a\omega)|dbd\omega.\\
		=&\int_{\mathbb{R}^{d}}\int_{\mathbb{R}}\frac{\kappa(\omega,b)e^{\eta\max\{0,|b|-M|\omega|\}}}{(1+|\omega|)^m}\sigma(\omega\cdot x+b)\frac{(1+|\omega|)^m|a|^d|\hat{f}(a\omega)|}{e^{\max\{0,\eta(|b|-M|\omega|)\}}}dbd\omega.\\
		=&\int_{\mathbb{R}^{d+1}}\frac{\kappa(\theta)e^{\max\{0,\eta(|b|-M|\omega|)\}}\gamma(f)}{(1+|\omega|)^m}\sigma(\omega\cdot x+b) d\lambda
	\end{aligned}
\end{equation}

where
$$
\gamma(f)=\int_{\mathbb{R}^{d+1}}\frac{(1+|\omega|)^m|a|^d|\hat{f}(a\omega)|}{e^{\max\{0,\eta(|b|-M|\omega|)\}}}d\theta
$$
and
$$
d\lambda= \frac{(1+|\omega|)^m|a|^d|\hat{f}(a\omega)|}{e^{\max\{0,\eta(|b|-M|\omega|)\}}\gamma(f)}d\theta,\quad \int_{\mathbb{R}^{d+1}}d\lambda=1,
$$


We can write
\begin{equation}
	\tilde f(x)\equiv\frac{f(x)}{\gamma(f)}
	=\mathbb{E}(g(\theta; x))
\end{equation}
where
\begin{equation}
	\label{gx}
	g(\theta; x))
	=\frac{\kappa(\theta)e^{\max\{0,\eta(|b|-M|\omega|)\}}}{(1+|\omega|)^m}\sigma(\omega\cdot x+b)
\end{equation}
And 
\begin{equation}
	D^\alpha \tilde f(x)=\mathbb{E}(D^\alpha g).
\end{equation}
%\mathbb{E}(\kappa(\omega,b)\frac{D^\alpha\sigma(\omega\cdot\tilde x+b)}{(1+|\omega|)^m})

Also we should have:
\begin{equation*}
	\begin{aligned}
		\|D^\alpha g\|_{\infty}&\le \frac{1}{2\pi|\hat{\sigma}(a)|}e^{\max\{0,\eta(|b|-M|\omega|)\}}|D^\alpha\sigma(\omega\cdot x+b)|\\
		&\le\frac{1}{2\pi|\hat{\sigma}(a)|}e^{\eta|b+\omega\cdot x|}|D^\alpha\sigma(\omega\cdot x+b)|\\
		&\le\frac{\zeta}{2\pi|\hat{\sigma}(a)|}
	\end{aligned}
\end{equation*}


Let $\theta_i=(\omega_i,b_i)_{i=1}^n$ be independently drawn from the same distribution $\lambda$ and let 
$$
\bar g(\theta_1,\ldots,\theta_n)
=\frac{1}{n} \sum_{i=1}^ng(\theta_i, x).
$$
Then
$$
\bar{\mathbb E} \left(\sum_{|\alpha|\le m}(D^\alpha \tilde f-D^\alpha \bar
g)^2\right) 
=\sum_{|\alpha|\le m}\bar{\mathbb E} (D^\alpha \tilde f-D^\alpha \bar g)^2
=\sum_{|\alpha|\le m}\bar{\mathbb E} (\mathbb E(D^\alpha g)-D^\alpha \bar g)^2 
\le{1\over n}\sum_{|\alpha|\le m}\|D^\alpha g\|_\infty^2.
$$
Taking the $L^2$ norm (with a probability measure) on both side of the above inequality
and using Fubini's theorem,
\begin{equation}
	\mathbb{\bar{\mathbb{E}}}(\|\tilde f(\cdot)-g(\theta,\cdot)\|^2_{H^m(\Omega)})
	\le{1\over n}\sum_{|\alpha|\le m}\|D^\alpha g\|_\infty^2.
\end{equation}
Thus there exists $\theta_i=(\omega_i^*,b_i^*)_{i=1}^n$ and $\beta_i$ such that 
$$
\|\tilde f-\frac{1}{n}\sum_{i=1}^{n} g(\theta_i; \cdot)\|_m^2
\le{1\over n}\sum_{|\alpha|\le m}\|D^\alpha g\|_\infty^2.
$$
$$
\|f-f_n\|_m^2
\le{\gamma(f)^2\over n}
\sum_{|\alpha|\le m}\|D^\alpha g\|_\infty^2.
$$
where
$$
f_n(x)=\frac{\gamma(f)}{n}\sum_{i=1}^{n} g(\theta_i; \cdot)=\frac{1}{n} \sum_{i=1}^n a_i\sigma(\omega_i^*\cdot\tilde{x}+b_i^*).
$$
where 
$$
a_i= \frac{\gamma(f)\kappa(\theta_i^*)e^{\max\{0,\eta(|b_i^*|-M|\omega_i^*|)\}}}{(1+|\omega_i^*|)^m}
$$

Now with
$$
\||f\||_{m+1} = \int_{\mathbb{R}^d}(1+|\omega/a|)^{m+1} |\hat{f}(\omega)|d(\omega)
$$
we need to give an estimation on $\gamma(f)$:
\begin{equation}
	\begin{aligned}
		&\gamma(f)=\int_{\mathbb{R}^{d+1}}\frac{(1+|\omega|)^m|a|^d|\hat{f}(a\omega)|}{e^{\max\{0,\eta(|b|-M|\omega|)\}}}dbd\omega\\
		=&\int_{\mathbb{R}^d}|a|^d(1+|\omega|)^m \int_{-|\omega|M}^{|\omega|M}db |\hat{f}(a\omega)|d\omega + \int_{\mathbb{R}^d}|a|^d(1+|\omega|)^m \int_{|b|\ge M|\omega|}\frac{1}{e^{\eta(|b|-M|\omega|)}}db |\hat{f}(a\omega)|d\omega  \\
		\le&\||f\||_{m+1}+C\||f\||_{m}\\
		\le&C\||f\||_{m+1}
	\end{aligned}
\end{equation}











%\subsection{Noncompactly supported activation function}
%
%Now $\sigma$ is not compactly supported, however, we can still write $f$ as:
%$$
%f(x)=\int_{\mathbb{R}^{d+1}}\kappa(\theta)\sigma(\omega\cdot\tilde x+b )|a|^d|\hat{f}(a\omega)|d\theta.
%$$
%
%denote 
%$$
%f_M(x)=\int_{\mathbb{R}^{d+1}}\kappa(\theta)\mathbf{1}_{D_M}\sigma(\omega\cdot \tilde{x}+b)|a|^d|\hat{f}(a\omega)|d\theta
%$$
%where for $M >0$, 
%\begin{equation}
%D_M=\{\theta=(\omega,b):|b|\le(1+|\omega|)M  \}
%\end{equation}
%by the conclusion in last subsection, we have :
%$$
%\|f_M-f_n\|_m
%\le C\frac{\|\sigma\|_{m,\infty}}{\sqrt{n}}\|f\|_{m+1} 
%$$
%where $C=\frac{M\sqrt{C_{m+d}^m}}{\pi|\hat{\sigma}(a)|} $.
%
%Now it suffices to estimate $\|f-f_M\|_m$:
%\begin{equation}
%f_r:=f-f_M=\int_{\mathbb{R}^{d}}\int_{|b|>(1+|\omega|)M}\kappa(\omega,b)\sigma(\omega\cdot \tilde{x}+b)|a|^d|\hat{f}(a\omega)|d\omega db
%\end{equation}
%
%Since we are in a bounded domain $\Omega$, we can choose $M>0$ such that $x\in\Omega\subset B_M$ where $B_M$ is the ball of radius $M$ in $\mathbb{R}^d$.
%If $x\in\Omega$ and $|b|>(1+|\omega|)M$, then:
%$$
%|\omega\cdot x+b|\ge |b|-|\omega\cdot x| \ge M
%$$
%
%Hence for all $x\in\Omega$ and $0\le|\alpha|\le m$, since $(1+|\omega|)^k$ is non decreasing w.r.t k, we have:
%\begin{equation}
%\begin{aligned}
%|D^{\alpha}f_r|&\le\frac{1}{2\pi|\hat{\sigma}(a)|}\|D^{|\alpha|}\sigma(t)\|_{L^1(|t|\ge M)}\int_{\mathbb{R}^d}(1+|\omega/a|)^{m} |\hat{f}(\omega)|d(\omega)\\
%&= \frac{1}{2\pi|\hat{\sigma}(a)|}\|D^{|\alpha|}\sigma(t)\|_{L^1(|t|\ge M)}\|f\|_{m}\\
%&\le\frac{1}{2\pi|\hat{\sigma}(a)|}\|D^{|\alpha|}\sigma(t)\|_{L^1(|t|\ge M)}\|f\|_{m+1}
%\end{aligned}
%\end{equation}
%Thus take the $L^2$ norm w.r.t a probability measure on $\Omega$, we can have:
%\begin{equation}
%\int_{\mathbb{R}^{d}}|D^{\alpha}f_r|^2\le \frac{1}{(2\pi|\hat{\sigma}(a)|)^2}\|D^{|\alpha|}\sigma(t)\|^2_{L^1(|t|\ge M)}\|f\|_{m+1}^2
%\end{equation}
%
%Summing over $0\le|\alpha|\le m$, and let
%$$
%R_{M}(\sigma)=\max_{0\le k\le m}\|D^{k}\sigma(t)\|^2_{L^1(|t|\ge M)}
%$$
%we have:
%\begin{equation}
%\|f_r\|_{H^m}^2\le \frac{C_{m+d}^m}{(2\pi|\hat{\sigma}(a)|)^2}R_{M}^2(\sigma)\|f\|_{m+1}^2
%\end{equation}
%
%Thus in the end we have:
%\begin{equation}
%\begin{aligned}
%\|f-f_n\|_{H^m}&\le\|f_r\|_{H^m}^2+\|f_M-f_n\|_m\\
%&\le \frac{\sqrt{C_{m+d}^m}}{2\pi|\hat{\sigma}(a)|}R_{M}(\sigma)\|f\|_{m+1}+\frac{M\sqrt{C_{m+d}^m}}{\pi|\hat{\sigma}(a)|} \frac{\|\sigma\|_{m,\infty}}{\sqrt{n}}\|f\|_{m+1} \\
%&\le  \frac{\sqrt{C_{m+d}^m}}{\pi|\hat{\sigma}(a)|}\|f\|_{m+1}(\frac{R_{M}(\sigma)}{2}+\frac{M\|\sigma\|_{m,\infty}}{\sqrt{n}})
%\end{aligned}
%\end{equation}