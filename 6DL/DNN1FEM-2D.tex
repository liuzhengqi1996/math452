\subsection{Linear finite element cannot be recovered by ${\rm DNN}_1$ for $d\ge2$}
In view of  Theorem~\ref{thm:1dLFEMDNN} and the fact that ${\rm{DNN}_J}
\subseteq {\rm{DNN}_{J+1}} $, it is natural to ask that how many
layers are needed at least to recover all linear finite element
functions in $\mathbb{R}^d$ for $d\ge2$.  In this section, we will show that 
\begin{equation}\label{key}
J_d \ge 2, \quad \text{if} \quad d\ge 2,
\end{equation}
where $J_d$ is the minimal $J$ such that all linear finite element
functions in $\mathbb R^d$ can be recovered by ${\rm DNN}_J$.

In particular, we will show the following theorem~\cite{he2020relu}.
\begin{theorem}\label{lowerbound}
	If $\Omega\subset \mathbb R^d$ is either 
	a bounded domain or $\Omega=\mathbb{R}^d$,  
	${\rm DNN}_1$ can not be used to recover all linear finite element
	functions on $\Omega$. 
\end{theorem}
\begin{proof}
	We prove it by contradiction. Let us assume that for any continuous
	piecewise linear function $f: \Omega \to \mathbb{R} $, we can find
	finite $N \in \mathbb{N}$, $w_i \in \mathbb{R}^{1,d}$ as row vector
	and $\alpha_i, b_i, \beta \in \mathbb{R}$ such that
	$$
	f =  \sum_{i=1}^N \alpha_i {\rm ReLU}(w_i\cdot  x +b_i) + \beta,
	$$
	with $f_i = \alpha_i {\rm ReLU}(w_i\cdot  x +b_i)$, $\alpha_i \neq 0$ and $w_i
	\neq 0$.  Consider the finite element functions, if this one hidden
	layer ReLU DNN can recover any basis function of FEM, then it can
	recover the finite element space.  Thus let us assume $f$ is a locally
	supported basis function for FEM.
	Furthermore, if $\Omega$ is a bounded domain, we assume that 
	\begin{equation}\label{distcondi}
	d({\rm supp}(f), \partial \Omega) > 0,
	\end{equation}with 
	$$
	d(A, B) = \inf_{x\in A, y\in B} \|x-y\|,
	$$ 
	as the distance of two closed sets. 
	
	A more important observation is that $\nabla f: \Omega \to
	\mathbb{R}^d$ is a piecewise constant vector function. The key
	point is to consider the discontinuous points for 
	$$g := \nabla
	f = \sum_{i=1}^N \nabla f_i.$$
	For more general case, we can define the set of discontinuous points of a function by
	$$
	D_{g} := \{x \in \Omega~|~ x ~ \text{is a discontinuous point of} ~ g\}.
	$$
	Because of the property that 
	\begin{equation}\label{eq:disfun}
	D_{f+g} \supseteq D_{f} \cup D_{g} \backslash (D_{f} \cap D_{g}),
	\end{equation}
	we have
	\begin{equation}\label{eq:dis_fn}
	D_{\sum_{i=1}^N g_i} \supseteq \bigcup_{i=1}^N D_{g_i} \backslash \bigcup_{i\neq j}\left( D_{g_i}\cap D_{g_j} \right).
	\end{equation}
	Note that
	\begin{equation}\label{eq:def_gi}
	g_i = \nabla f_i(x) =  \nabla \left( \alpha_i {\rm ReLU}(w_i\cdot   x +b_i)  \right) =\left(\alpha_iH(w_i \cdot  x +b_i)\right)w_i \in \mathbb{R}^d,
	\end{equation}
	for $i=1:N$ with $H$ be the Heaviside function defined as: 
	$$
	H(x) = \begin{cases}
	0 &\text{if} ~ x \le 0, \\
	1 &\text{if} ~ x > 0.
	\end{cases}
	$$ 
	This means that 
	\begin{equation}\label{eq: D_gi}
	D_{g_i} = \{ x ~|~ w_i\cdot   x + b_i = 0\}
	\end{equation}
	is a $d-1$ dimensional affine space in $\mathbb{R}^d$.  
	
	
	Without loss of generality, we can assume that 
	\begin{equation}\label{eq:assumD_gi}
	D_{g_i} \neq D_{g_j}.
	\end{equation}
	When the other case occurs, i.e. $D_{g_{\ell_1}} = D_{g_{\ell_2}} = \cdots= D_{g_{\ell_k}}$, by the definition of $g_i$ in \eqref{eq:def_gi} and $D_{g_i}$ in \eqref{eq: D_gi} , 
	this happens if and only if there is a row vector $(w, b)$ such that
	\begin{equation}\label{eq:Dfcondition}
	c_{\ell_i}\begin{pmatrix}
	w &
	b
	\end{pmatrix} =  
	\begin{pmatrix}
	w_{\ell_i} &
	b_{\ell_i}
	\end{pmatrix},
	\end{equation}
	with some $c_{\ell_i} \neq 0$ for $i = 1:k$.  We combine those $g_{\ell_i}$ as
	\begin{equation*}
	\begin{aligned}\label{mergeH}
	%	\begin{split}
	\tilde g_{\ell} &= \sum_{i=1}^k g_{\ell_i} = \sum_{i=1}^k \alpha_{\ell_i} H(w_{\ell_i} \cdot  x + b_{\ell_i}) w_{\ell_i}, \\
	&= \sum_{i=1}^k \left( c_{\ell_i}\alpha_{\ell_i} H\left(c_{\ell_i}(w\cdot   x + b)\right) \right) w, \\
	&=\begin{cases}
	\displaystyle \left(\sum_{i=1}^k  c_{\ell_i}\alpha_{\ell_i} H(c_{\ell_i}) \right) w  \quad &\text{if} \quad w x + b > 0,\\
	\displaystyle \left(\sum_{i=1}^k  c_{\ell_i}\alpha_{\ell_i} H(-c_{\ell_i}) \right) w  \quad &\text{if} \quad w x + b \le 0.\\
	\end{cases}
	%	\end{split}
	\end{aligned}
	\end{equation*}	
	Thus, if 
	$$
	\left(\sum_{i=1}^k  c_{\ell_i}\alpha_{\ell_i} H(c_{\ell_i}) \right)  = \left(\sum_{i=1}^k  c_{\ell_i}\alpha_{\ell_i} H(-c_{\ell_i}) \right),
	$$
	$\tilde g_\ell$ is a constant vector function, that is to say $D_{\sum_{i=1}^k g_{\ell_i}} = D_{\tilde g_\ell} = \emptyset$. 
	Otherwise, $\tilde g_\ell$ is a piecewise constant vector function with the property that 
	$$
	D_{\sum_{i=1}^k g_{\ell_i}} = D_{\tilde g_\ell} = D_{g_{\ell_i}} = \{ x ~|~ w\cdot  x + b = 0\}.
	$$
	This means that we can use condition \eqref{eq:Dfcondition} as an equivalence relation and split $\{g_i\}_{i=1}^N$ into some groups, and we can combine those $g_{\ell_i}$ in each group as what we do above. After that, we have
	$$
	\sum_{i=1}^N g_i = \sum_{\ell=1}^{\tilde N} \tilde g_{\ell},
	$$
	with $D_{\tilde g_s} \neq D_{\tilde g_t}$.
	Finally, we can have that $D_{\tilde g_s} \cap D_{\tilde g_t}$ is an empty set or a $d-2$ dimensional affine space in $\mathbb{R}^d$.
	Since
	$\tilde N \le N$ is a finite number, 
	$$
	D := \bigcup_{i=1}^N D_{\tilde g_\ell} \backslash \bigcup_{s\neq t}\left( D_{\tilde g_s}\cap D_{\tilde g_t} \right)
	$$
	is an unbounded set. 
	\begin{itemize}
		\item If $\Omega = \mathbb{R}^d$,
		$$
		{\rm supp(f)} \supseteq D_{g} = D_{\sum_{i=1}^N g_i} = D_{ \sum_{\ell=1}^{\tilde N} \tilde g_{\ell}} \supseteq D,
		$$ is contradictory to the assumption that $f$ is locally supported.
		\item If $\Omega$ is a bounded domain, 
		$$
		d(D, \partial \Omega) = 
		\begin{cases}
		s > 0 \quad &\text{if}\quad  D_{\tilde g_i} \cap \Omega = \emptyset, \forall i\\
		0 \quad &\text{otherwise}.
		\end{cases}
		$$
		Note again that all $D_{\tilde g_i}$'s are $d-1$ dimensional affine spaces, while $D_{\tilde g_i} \cap D_{\tilde g_j}$ is either an empty set or a d-2 dimensional affine space. 
		If $d(D, \partial \Omega) > 0$, this implies that $\nabla f$ is continuous in $\Omega$, which contradicts the  assumption that $f$ is a basis function in FEM.
		If $d(D, \partial \Omega) = 0$, this contradicts the previous assumption in \eqref{distcondi}.
	\end{itemize}
	Hence ${\rm DNN}_1$ cannot recover any piecewise linear function in $\Omega$ for $d \ge 2$.
\end{proof}

Following the proof above, we have the following theorem~\cite{he2020relu}.
\begin{theorem}\label{linearindep}
	$\{{\rm ReLU}(w_i\cdot x+b_i)\}_{i=1}^m$ are linearly independent if $(w_i,
	b_i)$ and $(w_j, b_j)$ are linearly independent in
	$\mathbb{R}^{1\times (d+1)} $ for any $i \neq j$.
\end{theorem}
