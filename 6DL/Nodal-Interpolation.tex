\paragraph{Nodal value interpolant}
\Label{sc-seio}

For any continuous function $u$, we define its linear finite element
interpolation, $(I_h u)(x)\in V_{h,0}$,  as follows:
\begin{equation}
  \label{u-interp}
(I_h u)(x)= \sum_{i=1}^{n_h}u(x_i)\varphi_i(x).
\end{equation}
Usually, we also denote $(I_h u)(x)$ as $u_I(x)$. Using interpolation, we can obtain the following approximation property of 
linear finite element space. 
%For any $v\in\Shz$, we can obviously write
%$$
%        v(x)=\sum_{i=1}^{n_h}v(x_i)\phi_i(x).
%$$
%The nodal value interpolation\index{interpolation} operator $I_h: C(\bar\Om)\mapsto V_h$ is defined as follows \index{$I_h$}
%$$
%        (I_h u)(x_i)=u(x_i),\qall x_i\in {\cal N}_h,
%$$
%where ${\cal N}_h$ is the set of the vertexes for the partition $\mathcal T_h$. 
\begin{figure}[hpt]
\begin{center}
\includegraphics*[height=2.5in, width=3in]{figures/fdsolutions.pdf}
\caption{Approximation of finite element space.} 
\label{Interpolation}
\end{center}
\end{figure}


\begin{theorem}\label{interp00}
Assume that $\mathcal T_h$ is quasi-uniform and $V_h$ is the linear finite element space associated with $\mathcal T_h$, then
\begin{equation}
\label{error0}
\inf_{v_h\in V_h} \|v-v_h\|+h |v-v_h|_{1}\lc h^2 |v|_2
        \qall v\in H^2(\Om).
\end{equation}
 \end{theorem}
 \begin{proof}  
Let us first prove Theorem \ref{interp00} for $d=1, 2, 3$.
This proof presented here follows from Xu~\cite{xu1982estimate} (see also Xu~\cite{xu2013estimate}).
Let $x=(x^1,\ldots, x^d)$ and $a_i=(a^1_{i}, \ldots, a^d_{i})$. Introducing
the auxiliary functions
$$
g_i(t)=v(a_i(t)),\mbox{  with  }  a_i(t)=a_i+t(x-a_i),
$$
we have
$$
g_i'(t)=(\nabla v)(a_i(t))\cdot (x-a_i)
=\sum_{l=1}^d(\partial_lv)(a_i(t))(x^l-a_i^l)
$$
and
\begin{equation}\label{gpp}
g_i''(t)=\sum_{k,l=1}^d\partial^2_{kl}v)(a_i(t))(x^k-a_i^k)(x^l-a_i^l).
\end{equation}
Note Taylor expansion
$$
        g_i(0)=g_i(1)-g_i'(1)+\int_0^1tg''_i(t)dt,
$$
namely
\begin{equation}\label{Taylor_vi}
v(a_i)=v(x)-(\nabla v)(x)\cdot (x-a_i)+\int_0^1tg''_i(t)dt,
\end{equation}
and note that
$$
(I_hv)(x)=\sum_{i=1}^{d+1}v(a_i)\lambda_i(x), \quad \sum_{i=1}^{d+1}\lambda_i(x)=1,
$$
and
$$
\sum_{i=1}^{d+1}(x-a_i)\lambda_i(x)=0.
$$
It follows that
\begin{equation}\label{Ihvv}
(I_hv-v)(x)=\sum_{i=1}^{d+1}\lambda_i(x)\int_0^1tg''_i(t)dt.
\end{equation}
Using \rf{gpp} and the trivial fact that $|x^l-a_i^l|\le h$,
we obtain
\begin{eqnarray*}
\|g''_i(t)\|_{L^2(\tau)}\le h^2
\sum_{k,l=1}^d\|(\partial^2_{kl}v)(a_i(t))\|_{L^2(\tau_i^t)}
\le h^2t^{-d/2}\sum_{k,l=1}^d\|\partial^2_{kl}v\|_{L^2(\tau)},
\end{eqnarray*}
where we have used the following change of variable
$$
y=a_i+t(x-a_i): \tau\mapsto \tau_i^t\subset\tau \mbox{ with } dy=t^ddx.
$$
Now taking the $L^2(\tau)$ norm on both hand of sides of
\rf{Ihvv}, we get
\begin{eqnarray*}
\|I_hv-v\|_{L^2(\tau)}
&\le& h^2\sum_{i=1}^{d+1}\max_{x\in\tau}|\lambda_i(x)|
\int_0^1t\|g''_i(t)\|_{L^2(\tau)}\;dt\\
&\le& (d+1)\int_0^1t^{-d/2}dt\;h^2\;
\sum_{k,l=1}^d\|\partial^2_{kl}v\|_{L^2(\tau)}\\
&\le&\frac{2(d+1)}{4-d}h^2
\sum_{k,l=1}^d\|\partial^2_{kl}v\|_{L^2(\tau)}\\
&\le&\frac{4d(d+1)}{4-d}h^2|v|_{H^2(\tau)}.
\end{eqnarray*}
Now we prove the $H^1$ error estimate. Notice that
$$
[\partial_{j}( I_{h} v - v)](x) = \sum_{i} (\partial_{j} \lambda_{i} )(x) \int_{0}^{1} t g''_{i}(t) dt + \sum_{i} \lambda_{i}(x) \partial_{j} \int_{0}^{1} t g''_{i}(t) dt.
$$ 
By \rf{Taylor_vi},
$$
\int_0^1tg''_i(t)dt = v(a_i) - v(x) + (\nabla v)(x)\cdot (x-a_i)
$$
therefore,
\begin{eqnarray*}
\lefteqn{\partial_{j} \int_0^1tg''_i(t)dt} \\
& = & - \partial_{j} v + (\nabla \partial_{j} v )(x) (x - a_{i}) + \nabla v \cdot e_{j} 
\hcomment{$e_{j}$ is the $j$-th standard basis}  \\
& = & (\nabla \partial_{j} v )(x) (x - a_{i}).
\end{eqnarray*}
Noting that $\sum_{i} \lambda_{i}( \nabla \partial_{j} v )(x) (x - a_{i}) = 0$, we have
$$
[\partial_{j}( I_{h} v - v)](x) = \sum_{i} (\partial_{j} \lambda_{i} )(x) \int_{0}^{1} t g''_{i}(t) dt.
$$
Then the estimate for $|\nabla(I_hv-v)|_{L^2(\tau)}$
follows by a similar argument and the following obvious
estimate
$$
|(\nabla\lambda_i)(x)|\lc\frac{1}{h}.
$$
 
On the proof of Theorem \ref{interp0} for $d\ge 4$, the above proof does not
apply for $d \ge 4$. This is because when $d \ge 4$, the embedding
relation between $H^{2}(\Om) \hookrightarrow C(\bar{\Om})$ is no longer true.  Only
continuous functions can have interpolations. In this case, one approach is to use the 
so-called Scott-Zhang interpolation \cite{scott1990finite}, the
details can be found in \cite{Xu.J2015a}.
\end{proof}
As a result of Theorem \ref{interp00}, we have
\begin{theorem}\label{interp0}
Let $V_N$ be linear finite element space on a quasi-uniform
triangulation consisting of $N$ element.  Then 
\begin{equation}
\label{error0N}
\inf_{v_h\in V_N} \|v-v_h\|+N^{-{1\over d}} |v-v_h|_{1}\lc N^{-{2\over d}} |v|_2
        \qall v\in H^2(\Om).
\end{equation}
\end{theorem}



