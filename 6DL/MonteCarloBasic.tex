\subsection{Monte Carlo sampling and analysis}
Let $\lambda\ge 0$ be a probability density function on a domain $G \subset
\mathbb R^D (D\ge 1)$ such that
\begin{equation}\label{density}
\int_{G}\lambda(\theta)d\theta=1.
\end{equation}
We define the expectation and variance as follows
\begin{equation}
\label{E}
\mathbb{E}g:=\int_{G}
g(\theta)\lambda(\theta)d\theta,\qquad
\mathbb{V} g: = \mathbb{E}((g - \mathbb{E} g)^2)= \mathbb{E}(g^2) - (\mathbb{E} g)^2.
\end{equation}
We note that 
$$\displaystyle \mathbb{V} g\le \max_{\theta, \theta'\in G}(g(\theta) - g(\theta'))^2.
$$
For any subset $G_i\subset G$, let
$$
\lambda(G_i)=\int_{G_i}\lambda(\theta)d\theta, \qquad \lambda_i(\theta) = {\lambda(\theta)\over \lambda(G_i)}.
$$
It holds that
$$
\mathbb{E}_Gg= \sum_{i=1}^M \lambda(G_i) \mathbb{E}_{G_i}g.
$$ 
%For each $i$, sample $\theta_i$ is chosen randomly from the probability distribution in $G_i$ with probability $\lambda_i(\theta)$. 
For any function $h(\theta_1,\cdots, \theta_N) : G_1\times G_2\cdots G_N \mapsto \mathbb{R}$, define 
$$\mathbb{E}_{G_i}g=\int_{G_i}g(\theta)\lambda_i(\theta)d\theta$$
 and
\begin{equation}
\label{En}
\mathbb{E}_Nh:=\int_{G_1\times G_2\times\ldots\times G_N}
h(\theta_1,\cdots,\theta_N) \lambda_1(\theta_1) \lambda_2(\theta_2)\ldots \lambda_N(\theta_N)
d\theta_1d\theta_2\ldots d\theta_N.
\end{equation}
For the Monte Carlo method, let $G_i=G$ for all $1\le i\le n$, namely,
\begin{equation} 
\mathbb{E}_Nh:=\int_{G\times G\times\ldots\times G}
h(\theta_1,\cdots,\theta_N) \lambda(\theta_1) \lambda(\theta_2)\ldots \lambda(\theta_N)
d\theta_1d\theta_2\ldots d\theta_N.
\end{equation}


\begin{lemma}   \label{MC}
For any $g\in L^\infty(G)$, we have
  \begin{equation}
  \begin{split}
    \mathbb{E}_n\Big(\mathbb{E}g-\frac1n\sum_{i=1}^n
    g(\omega_i)\Big)^2
    &=\frac{1}{n}\mathbb{V}(g)=\frac{1}{n}\Big(\mathbb{E}(g^2) - \big (\mathbb{E}(g)\big )^2\Big)
    \\
    &=
    \left\{
             \begin{aligned}
    \frac{1}{n}\mathbb{V}(g)
   & \le\frac{1}{n} \sup_{\omega, \omega'\in G} |g(\omega) - g(\omega')|^2
    \\
\frac{1}{n}\Big(\mathbb{E}(g^2) - \big (\mathbb{E}(g)\big )^2\Big)
&\le\frac{1}{n} \mathbb E(g^2)\le \frac{1}{n}\|g\|^2_{L^\infty},
\end{aligned}
\right.
\end{split}
  \end{equation} 
\end{lemma}

\begin{proof}%[Proof of Lemma \ref{MC}]
First note that
 \begin{equation}
    \label{eqn}
    \begin{aligned}
\left(\mathbb{E} g-\frac1N\sum_{i=1}^Ng(\omega_i)\right)^2 
  & 
=\frac{1}{N^2} \left(\sum_{i=1}^N(\mathbb{E} g-g(\omega_i))\right)^2 
  \\
  &=\frac{1}{N^2} \sum_{i,j=1}^N(\mathbb{E} g-g(\omega_i))(\mathbb{E} g-g(\omega_j))
  \\
  &=\frac{I_1}{N^2} +\frac{I_2}{N^2}.
    \end{aligned}
  \end{equation}
with 
\begin{equation}
I_1= \sum_{i=1}^N(\mathbb{E} g-g(\omega_i))^2,\quad I_2=\sum_{i\neq  j}^N\left ((\mathbb{E}g)^2-\mathbb {E}(g)(g(\omega_i)+
g(\omega_j))+g(\omega_i)g(\omega_j))\right).
\end{equation}
Consider $I_1$, for any $i$,
 $$
 \mathbb{E}_N(\mathbb{E} g-g(\omega_i))^2
 =\mathbb{E}(\mathbb{E} g-g)^2 = \mathbb{V}(g).
 $$ 
Thus,
$$
 \mathbb{E}_N (I_1) = n\mathbb{V}(g).
$$
For $I_2$, note that
$$
\mathbb E_N g(\omega_i)=\mathbb E_N g(\omega_j) =\mathbb E(g)
$$
and, for $i\neq j$,
\begin{equation}\label{key}
\begin{aligned}
\mathbb {E}_N ( g(\omega_i)g(\omega_j)) &= 
\int_{G\times G\times\ldots\times G}
g(\omega_i) g(\omega_j) \lambda(\omega_1) \lambda(\omega_2)\ldots \lambda(\omega_N)
d\omega_1d\omega_2\cdots d\omega_N \\
&= \int_{G\times G} g(\omega_i) g(\omega_j) \lambda(\omega_1) 
\lambda(\omega_1) \lambda(\omega_2)
d\omega_1d\omega_2 \\
&= \mathbb {E}_N (
g(\omega_i))\mathbb {E}_n(g(\omega_j))
=[\mathbb E(g)]^2.
\end{aligned}
\end{equation}
Thus
 \begin{equation}
\mathbb{E}_N (I_2) = \mathbb{E}_N \left( \sum_{i\neq j}^N((\mathbb{E}g)^2-\mathbb
  E(g)(\mathbb E(g(\omega_i))+ \mathbb E(g(\omega_j)))+\mathbb E(g(\omega_i)g(\omega_j))) \right)=0.
  \end{equation}
Consequently, there exist the following two formulas for $\displaystyle \mathbb{E}_N\left(\mathbb{E} g-
      \frac1N\sum_{i=1}^Ng(\omega_i)\right)^2$:
 \begin{equation} 
 \mathbb{E}_N\left(\mathbb{E} g-
      \frac1N\sum_{i=1}^Ng(\omega_i)\right)^2 = \frac{1}{N^2}\mathbb{E}_N (I_1)
      =
           \left\{
             \begin{aligned}
            \frac{1}{N}\mathbb{E}\big ((\mathbb{E} g-g)^2\big )\\
            \frac{1}{N}(\mathbb{E}(g^2) - (\mathbb{E} g)^2).
            \end{aligned}
    \right.
  \end{equation}
Based on the first formula above, since
$$
|g(\omega) - \mathbb{E} g|=|\int_G \big (g(\omega) - g(\tilde \omega) \big )\lambda(\tilde \omega)d\tilde \omega|\le \sup_{\omega, \omega'\in G} |g(\omega) - g(\omega')|,
$$
it holds that
 \begin{equation} 
 \mathbb{E}_N\left(\mathbb{E} g-
      {1\over N }\sum_{i=1}^Ng(\omega_i)\right)^2 
            \le\frac{1}{N} \sup_{\omega, \omega\in G} |g(\omega) - g(\omega')|^2.
  \end{equation}
Due to the second formula above,  
 \begin{equation}
    \label{eqn}
 \mathbb{E}_N\left(\mathbb{E} g-
      \frac1N\sum_{i=1}^Ng(\omega_i)\right)^2  
            \le\frac{1}{N} \mathbb E(g^2)\le\frac{1}{N}\|g\|^2_{L^\infty}
  \end{equation}
which completes the proof.
\end{proof}

We can also generalize Lemma \ref{MC} to Hilbert spaces following a similar analysis.

\begin{lemma}    
For any $g: G\rightarrow H$ where $H$ is a Hilbert space, we have
  \begin{equation}
  \begin{split}
    \mathbb{E}_n\Big(\|\mathbb{E}g-\frac1n\sum_{i=1}^n
    g(\omega_i)\|_H^2\Big)
    &=\frac{1}{n}\mathbb{V}(g)=\frac{1}{n}\Big(\mathbb{E}(\|g\|_H^2) - \big (\mathbb{E}(\|g\|_H)\big )^2\Big)
    \\
    &=
    \left\{
             \begin{aligned}
    \frac{1}{n}\mathbb{V}(g)
   & \le\frac{1}{n} \sup_{\omega, \omega'\in G} \|g(\omega) - g(\omega')\|_H^2
    \\
\frac{1}{n}\Big(\mathbb{E}(\|g\|_H^2) - \big (\mathbb{E}(\|g\|_H)\big )^2\Big)
&\le\frac{1}{n} \mathbb E(\|g\|_H^2),
\end{aligned}
\right.
\end{split}
  \end{equation} 
\end{lemma}

\begin{proof}%[Proof of Lemma \ref{MC}]
First note that
 \begin{equation}
    \label{eqn}
    \begin{aligned}
\left\|\mathbb{E} g-\frac1N\sum_{i=1}^Ng(\omega_i)\right\|_H^2 
  & 
=\frac{1}{N^2} \left\|\sum_{i=1}^N(\mathbb{E} g-g(\omega_i))\right\|_H^2 
  \\
  &=\frac{1}{N^2} \sum_{i,j=1}^N\left(\mathbb{E} g-g(\omega_i), \mathbb{E} g-g(\omega_j)\right)
  \\
  &=\frac{I_1}{N^2} +\frac{I_2}{N^2}.
    \end{aligned}
  \end{equation}
with 
\begin{equation}
I_1= \sum_{i=1}^N\|\mathbb{E} g-g(\omega_i)\|_H^2,\quad I_2=\sum_{i\neq  j}^N\left(\mathbb{E} g-g(\omega_i), \mathbb{E} g-g(\omega_j)\right).
\end{equation}
Consider $I_1$, for any $i$,
 $$
 \mathbb{E}_N(\|\mathbb{E} g-g(\omega_i)\|_H^2)
 =\mathbb{E}(\|\mathbb{E} g-g\|_H^2) = \mathbb{V}(g).
 $$ 
Thus,
$$
 \mathbb{E}_N (I_1) = n\mathbb{V}(g).
$$
For $I_2$, note that
$$
\mathbb E_N \|g(\omega_i)\|_H=\mathbb E_N \|g(\omega_j)\|_H =\mathbb E(\|g\|_H)
$$
and, for $i\neq j$,
\begin{equation}\label{key}
\begin{aligned}
\mathbb {E}_N ( g(\omega_i), g(\omega_j)) &= 
\int_{G\times G\times\ldots\times G}
g(\omega_j) g(\omega_j) \lambda(\omega_1) \lambda(\omega_2)\ldots \lambda(\omega_N)
d\omega_1d\omega_2\cdots d\omega_N \\
&= \int_{G\times G} (g(\omega_j) , g(\omega_j) ) 
\lambda(\omega_i) \lambda(\omega_j)
d\omega_id\omega_j \\
&=\left( \mathbb {E}_N (
g(\omega)) , \mathbb {E}_N(g(\omega))\right)
=\|\mathbb E(g)\|_H^2.
\end{aligned}
\end{equation}
Thus
 \begin{equation}
\mathbb{E}_N (I_2) = \mathbb{E}_N \left( \sum_{i\neq j}^N\big(\|\mathbb{E}g\|_H^2-
(\mathbb  E(g), \mathbb E(g(\omega_i))+ \mathbb E(g(\omega_j)))
+ (g(\omega_i), g(\omega_j))\big) \right)=0.
  \end{equation}
Consequently, there exist the following two formulas for $\displaystyle \mathbb{E}_N\left\|\mathbb{E} g-
      \frac1N\sum_{i=1}^Ng(\omega_i)\right\|_H^2$:
 \begin{equation} 
 \mathbb{E}_N\left\|\mathbb{E} g-
      \frac1N\sum_{i=1}^Ng(\omega_i)\right\|_H^2 = \frac{1}{N^2}\mathbb{E}_N (I_1)
      =
           \left\{
             \begin{aligned}
            \frac{1}{N}\mathbb{E}\big (\|\mathbb{E} g-g\|_H^2\big )\\
            \frac{1}{N}(\mathbb{E}(\|g\|_H^2) - \|\mathbb{E} g\|_H^2).
            \end{aligned}
    \right.
  \end{equation}
Based on the first formula above, since
$$
|g(\omega) - \mathbb{E} g|=|\int_G \big (g(\omega) - g(\tilde \omega) \big )\lambda(\tilde \omega)d\tilde \omega|\le \sup_{\omega, \omega'\in G} |g(\omega) - g(\omega')|,
$$
it holds that
 \begin{equation} 
 \mathbb{E}_N\left\|\mathbb{E} g-
      {1\over N }\sum_{i=1}^Ng(\omega_i)\right\|_H^2 
            \le\frac{1}{N} \sup_{\omega, \omega'\in G} \|g(\omega) - g(\omega')\|_H^2.
  \end{equation}
Due to the second formula above,  
 \begin{equation}
    \label{eqn}
 \mathbb{E}_N\left\|\mathbb{E} g-
      \frac1N\sum_{i=1}^Ng(\omega_i)\right\|_H^2  
            \le\frac{1}{N} \mathbb E(\|g\|_H^2),
  \end{equation}
which completes the proof.
\end{proof}


%
Of course, we can use this to prove a high probability result.
\begin{corollary}
 Under the assumptions of the preceding lemma, we have
 \begin{equation}
\mathbb{\bar P}\left [(\mathbb{E}g-\frac1n\sum_{i=1}^n
    g(\omega_i))^2 >\epsilon\right ] 
\le  \frac{1}{n\epsilon}\|g\|^2_{L^\infty}
 \end{equation}
\end{corollary}
\begin{proof}
 \begin{equation}
  \mathbb{\bar P}\left [(\mathbb{E}g-\frac1n\sum_{i=1}^n
    g(\omega_i))^2 >\epsilon\right ] 
\le \epsilon^{-1}
    \mathbb{\bar E}(\mathbb{E}g-\frac1n\sum_{i=1}^n
    g(\omega_i))^2
\le \frac{1}{n\epsilon}\|g\|^2_{L^\infty}. 
 \end{equation}
 
\end{proof}

This corollary implies that the set of $\omega_i$ where the estimate
$n^{-1}\sum_{i=1}^n g(\omega_i)$ is far from the desired value $\mathbb{E}g$
is small.

The practical usefulness of this algorithm depends upon the existence
of a \textit{repeatable} process (for instance some physical process)
which \textit{generates $\omega$ according to a desired distribution
  $\mu$}.

The precise meaning of this last statement is essentially that the
strong law of large numbers holds. Specifically, if
$\omega_1,...,\omega_n,...$ is a infinite sequence generated by the
process, and $A\subset \Omega$ is any a measurable set, then
\begin{equation}
 \lim_{n\rightarrow\infty} \frac{1}{n}\displaystyle\sum_{i=1}^n\chi_A(\omega_i) = \mu(A).
\end{equation}

Generating $n$ independent samples means generating
$\omega_1,...,\omega_n$ from $\mu^n$ according to the above notion.
The existence of a realizable process generating samples from a
probability distribution, and the practical use of such processes is
an interesting topic in the intersection of statistics, physics, and
computer science. In addition, statistics/probability theory studies
how to take samples from one probability distribution and transform
them to samples from another distribution.

\begin{lemma}\label{lem:sample}
\textup{[Monte Carlo Sampling]}
	Consider 
	\begin{equation}
	\label{uv}
	u(x)=\int_{G}g(x,\theta)\rho(\theta)d\theta   = \mathbb E (g)
	\end{equation}
	with $0\le \rho(\theta)\in L^1(G)$. For any $N\ge 1$, there exist $\theta_i^*\in G$ such that
	$$
	\|u-u_N\|_{L^2(\Omega)}^2 
	\le\frac{1}{N}
	\int_G \|g(\cdot,\theta)\|_{L^2(\Omega)}^2\rho(\theta)d\theta = {\|\rho\|_{L^1(G)}\over N}\mathbb E (\|g(\cdot,\theta)\|_{L^2(\Omega)}^2)
	$$
	where  
	$
	\|g(\cdot,\theta)\|_{L^2(\Omega)}^2 = \int_{\Omega} [g(x,\theta)]^2 d\mu(x),
	$
	\begin{equation}\label{fndef} 
	u_N(x)=\frac{\|\rho\|_{L^1(G)}}{N}\sum_{i=1}^N g(x,\theta_i^*).
	\end{equation}

Similarly, if $g(\cdot, \theta)\in H^m(\Omega)$, for any $N\ge 1$, there exist $\theta_i^*\in G_i$ with $f_N$ given in \eqref{fndef} such that
	\begin{equation}\label{eq:hm}
	\|u-u_N\|_{H^m(\Omega)}^2 
	\le 
	\int_G  \|g(\cdot,\theta)\|_{H^m(\Omega)}^2\rho(\theta)d\theta
	=\frac{\|\rho\|_{L^1(G)}}{N} \mathbb E (\|g(\cdot,\theta)\|_{H^m(\Omega)}^2).
	\end{equation}
\end{lemma}

\iffalse
\begin{proof} 
Note that
\begin{equation}
\label{uv}
u(x) = \|\rho\|_{L^1(G)}\mathbb E (g).
\end{equation}
By Lemma \ref{MC},
$$
\mathbb {E}_n\left(\bigg(\mathbb E(g(x,\cdot))
-\frac{1}{N}\sum_{i=1}^N g(x,\theta_i))\bigg)^2
\right)\le {1\over N} \mathbb E (g^2).
$$
By taking integration w.r.t. $x$ on both sides, we get
$$
\mathbb {E}_n\left(h(\theta_1,\theta_2, \cdots, \theta_N)
\right)\le {1\over N} \mathbb {E} \Big(\int_{\Omega} g^2 d\mu(x)\Big),
$$
where 
$$
h(\theta_1,\theta_2, \cdots, \theta_N) =  \int_{\Omega} \bigg(\mathbb E(g(x,\cdot))
-\frac{1}{N}\sum_{i=1}^N g(x,\theta_i))\bigg)^2 d\mu(x).
$$
Since $\mathbb {E}_N (1) = 1$ and $\mathbb {E}_N (h) \le {1\over N} \mathbb {E} \Big(\int_{\Omega} g^2 d\mu(x)\Big)$, there exist $\theta_i^* \in G$ such that
$$
h(\theta_1^*, \theta_2^*, \cdots, \theta_N^*) \le  {1\over N}  \int_{\Omega} \mathbb {E} (g^2) d\mu(x).
$$
%Otherwise, $\mathbb {E}_n\left(h \right) >  {1\over n} \mathbb {E} \Big(\int_{\Omega} g^2 d\mu(x)\Big)$   if $h(\theta_1,\theta_2, \cdots, \theta_n) > {1\over n}  \int_{\Omega} \mathbb {E} (g^2)) d\mu(x)$. 
This implies that
$$
	\mathbb{E}_n\|u-u_N\|_{L^2(\Omega)}^2 
	\le\frac{\|\rho\|_{L^1(G)}}{N}
	\int_G \|g(\cdot,\theta)\|_{L^2(\Omega)}^2\lambda(\theta)d\theta.
	$$ 
	The proof for \eqref{eq:hm} is similar to the above analysis for the $L^2$-error analysis, which completes the proof.
\end{proof}
\fi



We also have a more general version of the above lemma.
\begin{lemma}\label{lem:sampleHk}
	Let 
	\begin{equation} \label{uint}
	u(x)=\int_{G}g(x,\theta)\lambda(\theta)d\theta  = \mathbb E (g)
	\end{equation}
	with $\|\lambda(\theta)\|_{L^1(\Theta)}=1$.
	For any $N\ge 1$, there exist $\theta_i^*\in G$ such that
	$$
	\|u-u_N\|_{H^m(\Omega)}^2 
	\le 
	\int_G  \|g(\cdot,\theta)\|_{H^m(\Omega)}^2\lambda(\theta)d\theta
	=\frac{1}{N} \mathbb E (\|g(\cdot,\theta)\|_{H^m(\Omega)}^2)
	$$
	where 
	$$
	u_N(x)=\frac{1}{N}\sum_{i=1}^N g(x,\theta_i^*)
	$$
	In particular, if 
	\begin{equation}
	\label{eq:4}
	|D^\alpha g(x,\theta)|\le C, \quad\forall x, \theta, |\alpha|\le m
	\end{equation}
	Then
	$$
	\|u-u_N\|_{H^m(\Omega)}
	\le 
	\begin{pmatrix}
	m+d\\
	m
	\end{pmatrix}^{1/2}
	|\Omega|^{1/2}
	N^{-1/2}.
	$$
\end{lemma}
