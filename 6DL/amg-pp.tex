%This is wonderful
\newpage
\section{Train Interpolation for AMG}
Here what we want do is to use the basic idea in Prod. Xu's big paper in ActaNumerica about the AMG method. 


\subsection{Basic Methods}
The original problem is set as: 
%\begin{problem}
	Find $P \in \mathbb{R}^{n\times n_c}$ as 
	\begin{equation}\label{min-tra}
	\min_{P \in \mathcal{X}_eta} \rm{Tr} (P^\top \bar{R}^{\frac{1}{2}} A \bar R^{\frac{1}{2}} P)
	\end{equation}
	where 
	\begin{equation}\label{min-con}
	\mathcal{X}_{\eta} = \{P \in \mathbb{R}^{n\times n_c} ~:~ P \bm{1} =  \sqrt{n_c\eta}\hat \xi_1, \quad (Pv, Pv) \ge \eta (v,v) \}.
	\end{equation}
%\end{problem}

But we want to train some GMG interpolation from AMG. So we can change those model little.
\subsection{Train GMG from AMG}
To verify that GMG can be obtained from this abstract AMG theory, we may ask the next problem:
\subsubsection{Original form}
%\begin{problem}
	Find $P \in \mathbb{R}^{n\times n_c}$ as 
	\begin{equation}\label{min-tra}
	\min_{P \in \mathcal{X}} \rm{Tr} (P^\top \bar{R}^{\frac{1}{2}} A \bar R^{\frac{1}{2}} P)
	\end{equation}
	where 
	\begin{equation}\label{min-con}
	\mathcal{X}_{} = \{P \in \mathbb{R}^{n\times n_c} ~:~ P \bm{1} =  \bm{1}, \quad \text{Sparse}(P) = \text{Sparse}(P^{GMG}) \}.
	\end{equation}
%\end{problem}
\subsubsection{Approximated form}
	Find $P \in \mathbb{R}^{n\times n_c}$ as 
\begin{equation}\label{min-tra}
\min_{P \in \mathcal{X}} \rm{Tr} (P^\top X A \bar X^\top)
\end{equation}
where 
\begin{equation}\label{min-con}
\mathcal{X}_{} = \{P \in \mathbb{R}^{n\times n_c} ~:~ P \bm{1} =  \bm{1}, \quad \text{Sparse}(P) = \text{Sparse}(P^{GMG}) \}.
\end{equation}
Here $\bar R = XX\top$ which we can adopt the exact formula as 
\begin{itemize}
	\item G-S
	\begin{equation}\label{XXT:G-S}
	\bar R^{-1} = (D-L)D(D-U) \Rightarrow X = (D-L)D^{1 \over 2}.
	\end{equation}
	\item ILU
		\begin{equation}\label{XXT:ILU}
		\bar R^{-1} = XX^T \Rightarrow X = ILU.
	\end{equation}
\end{itemize}

\subsubsection{Few parameters form}
We can both solve the above two problems with $P$ only have d.o.f like 
\begin{itemize}
	\item FDM:
\begin{itemize}
	\item Same sparsity w.r.t the ``graph": $5$
	\item Same sparsity w.r.t the GMG: $9$
	\item More connection w.r.t GMG: $16$
\end{itemize}
	\item FEM
\begin{itemize}
		\item Same sparsity w.r.t the ``graph": $7$
		\item Same sparsity w.r.t the GMG: $7$
		\item More connection w.r.t GMG: $9$ (add the symmetry property)
	\end{itemize}
\end{itemize}



\subsection{Implement Detail}
We use online PCA method to solve \eqref{min-tra}. Write $\tilde{A} = \bar{R}^{\frac 12}A\bar{R}^{\frac 12}$, and $P^S = (p^S_{ij}), p^S_{ij}=1 \text{ iff } p^{GMG}_{ij}\neq 0$. Let $\odot$ is entry-wise multiply, i.e.$(A\odot B)_{ij}=(a_{ij}b_{ij})$. Let $\|M\|_*=\sum_{i,j} |M_{ij}|$.
The algorithm is 
\begin{algorithm}  
	\caption{Online PCA for Prolongation of GMG}  
	\label{Online PCA for Prolongation of GMG}  
	\begin{algorithmic}  
		\STATE choose a series of learning rate $\eta_t$, max iteration steps $M_{\text{step}}$ and convergence threshold $\epsilon$ and batchsize $s$.
		\STATE $t \leftarrow 0$ and choose a random initial $P^0$ from distribution $U(0,1)$.
		\FOR{ each row $\mathbf p$ of $P^{0}$} 
		\STATE $\mathbf p \leftarrow \dfrac 1{\|\mathbf p\|_1}\mathbf{p}$
		\ENDFOR
		\STATE $P^0\leftarrow P^S\odot P^0$ 
		\REPEAT  
		\STATE Sample vectors $V=(\mathbf v_1,...,\mathbf v_s)$ from distribution $N(0,\tilde{A})$
		\STATE $P^{t+1} \leftarrow P^t - \dfrac{\eta_t}{s} V V^T P$
		\STATE $P^{t+1}\leftarrow P^S\odot \hat{P}^{t+1}$
		\FOR{ each row $\mathbf p$ of $P^{t+1}$} 
		\STATE $\mathbf p \leftarrow \dfrac 1{\|\mathbf p\|_1}\mathbf{p}$
		\ENDFOR
		\STATE $t\leftarrow t+1$
		\UNTIL{$t>M_{\text{step}}$ or $\|P^{t+1}-P^{t}\|_*<\epsilon \|P^{t+1}\|_* $ }  
	\end{algorithmic}  
\end{algorithm}  
\begin{remark}
	We can use any $X$ such that $\bar{R} =XX^T$ to replace $\bar{R}^\frac{1}{2}$. For GS smoother, because $\bar{R}=(D-U)^{-1}D(D-L)^{-T}$, we can easily set $X=(D-L)^{-1}D^\frac 12$. For general $\bar{R}$, we can choose $X$ is the incomplete Cholesky factorization to make $X$ sparse.
\end{remark}
\newpage
\subsection{Numerical Experiment}
We choose $A$ is the five points  difference form of Possion equation and smoother is GS.

Let $P^*$ is the result of using \eqref{Online PCA for Prolongation of GMG} to solve \eqref{min-tra}. The numerical experiment shows $\dfrac{\|P^{GMG}-P^*\|_*}{\|P^{GMG}\|_*}<8\%$ and if we use $P^*$ in two level multigrid method to solve $Ax=b$, the stop iteration step of $P^*$ is the same as $P^{GMG}$. If we do more iteration, we can get a better  $P^*$ than $P^{GMG}$.
%which means $\rm{Tr} ((P^*)^\T \tilde{A} P^*) <\rm{Tr} ((P^{GMG})^\T \tilde{A} P^{GMG})$ and the stop iteration step of multigrid method is less than $P^{GMG}$.



\section{A Proof for Training $\lambda$ smoother for AMG}
Give some distribution for $v$, and some $v_i$ as $i = 1:m$, we can compute 
\begin{equation}
	\frac{\sum_i \|Av_i\|^2}{\sum_i \|Av_i\|_A^2}.
\end{equation}

Now we sample $v_i$ form the distribution as:
\begin{equation}\label{sample}
v_i^{(j)} \sim N(0,1),  \quad \text{or}\quad  v_i^{(j)}  \sim U[-1,1],
\end{equation}
then we will have the next 