
\subsection{Stratified sampling}
Stratified sampling refers to a type of sampling method . With stratified sampling, the population is divided into separate groups, called strata. Then, a probability sample (often a simple random sample) is drawn from each group.

Consider the political polling, it's impractical to poll an entire population. So pollsters select a sample of individuals that represents the whole population. Understanding how respondents come to be selected to be in a poll is a big step toward determining how well their views and opinions mirror those of the voting population. A simple example is that choose $n$ individuals randomly,
$$
\mathbb{E}_p({\# Republicans \over n})=p, \quad Var({\# Republicans \over n})=p(1-p).
$$
The accuracy of the sampling is approximate to $\sqrt{p(1-p)\over n}$. If the size of the sample is 100 and $p\approx 0.5$, then $\sqrt{p(1-p)\over n}\approx 5\%$. This implies that the size of the sample should be large enough to get a good estimate.

Another way to select a sample is to break the population into several groups, say A and B. Suppose the percentage of individuals belong to group A is $p_A=70\%$, and the percentage for group B is $p_B=30\%$. And $p_{A, R}=80\%$ individuals in group A will vote for Republicans and only $p_{B, R}=10\%$ in group B will vote for Republicans. Instead of calling $n=100$ individuals randomly, we call $np_A=70$ people in group A and $np_B=30$ people in group B. Then the variance becomes
\begin{equation*}
\begin{split}
\mathbb{V}(0.7{\# Republicans \over 70} + 0.3{\# Republicans \over 30})&=0.7^2{p_{A, R}(1-p_{A, R})\over 70} + 0.3^2{p_{B, R}(1-p_{B, R})\over 30}\\
&=100(0.7*0.8*0.2 + 0.3*0.1*0.9).
\end{split}
\end{equation*}

This is an application of the stratified sampling. The population is divided into separate groups. There are two stratified sampling strategies: One is proportionate allocation, which  uses a sampling fraction in each of the strata that is proportional to that of the total population. The above example belongs to this case. The other strategy is 
optimum allocation, where the sampling fraction of each stratum is proportionate to both the proportion (as above) and the standard deviation of the distribution of the variable. 


Stratified sampling has several advantages over simple random sampling. For example, using stratified sampling, it may improve the precision of the sample by reducing sampling error. It can produce a weighted mean that has less variability than the arithmetic mean of a simple random sample of the population.

Let $\theta$ be a random variable representing the groups mentioned above and $P(\theta)$ be the distribution. Define
$$
f(x)=\mathbb{E}_{P} (f_{\theta}(x))=\mathbb{E}_{\theta} (q(x, \theta))=\int_G q(x,\theta) dP(\theta).
$$ 
A sample from the distribution $P(\theta)$ gives us $\theta_1, \theta_2, \cdots , \theta_n$. Let
\begin{equation}
f_n(x)={1\over n}\sum_{k=1}^n f_{ \theta_k}(x)={1\over n}\sum_{k=1}^n q(x, \theta_k)
\end{equation}
Then, the variance is 
\begin{equation}
\mathbb{V}(f_n(x))={1\over n}\mathbb{V}(f)
\end{equation}
with error ${1\over \sqrt{n}}$.

Stratified sampling is to break $G$ into groups $G_1, G_2, \cdots, G_M$ so that the variation of $q(x,\theta)$ is small on each $G_i$. Instead of sampling uniformly from $G$ and taking the average, we sample from each $G_i$ and get $\theta_{i,1}$, $\theta_{i,2}, \cdots, \theta_{i,n_i}$ where $n_i=\lceil nP(G_i)\rceil$ is the size of samples in $G_i$. Then there exists the estimate
\begin{equation}
\sum_{i=1}^M P(G_i)[{1\over n_i}\sum_{j=1}^{n_i} q(x,\theta_{ij})]
\end{equation}
where ${1\over n_i}\sum_{j=1}^{n_i} q(x,\theta_{ij})$ is an estimate of the conditional expectation $\mathbb{E}_p(q(x,\theta|\theta\in G_i))$. Note that $n_i\leq np_i$ with $p_i=P(G_i)=\int_{G_i} dP(\theta)$. The variance of the estimate is 
\begin{equation}
\sum_{i=1}^M p_i^2{\mathbb{V}[q(x,\theta)|\theta\in G_i)]\over n_i}
\leq \sum_{i=1}^M {p_i^2\over np_i}\mathbb{V}[q(x,\theta)|\theta\in G_i]
= {1\over n}\sum_{i=1}^M p_i\mathbb{V}[q(x,\theta)|\theta\in G_i].
\end{equation} 
This implies that we need to choose $G_i$ so that $\mathbb{V}[q(x,\theta)|\theta\in G_i]$ is small. 

If $G$ is bounded in $\mathbb{R}^d$ and $q(x,\theta)$ is smooth with respect to $\theta$,
$$
|\nabla_\theta q|\leq C.
$$
We can consider a particular partition of $G$ into  $M$ sets $G_1, G_2, \cdots, G_M$ with diameter $\mathcal{O}(n^{-1/d})$. We choose one element from each set. For each $G_i$, sample $n_i=\lceil nP(G_i)\rceil$ samples in $G_i$.  Let
\begin{equation}
f_n(x)=\sum_{k=1}^n p_i[{1\over n_i}\sum_{j=1}^{n_i}q(x,\theta_{ij})]\in NN_{2n}.
\end{equation}
The variance is 
\begin{equation}
Var(f_n)= {1\over n}\sum_{i=1}^M p_i\mathbb{V}[q(x,\theta)|\theta\in G_i]\leq (n^{-1/d}C)^2
\end{equation}
According to , there exists the following modified estimate 
\begin{equation}
Var(f_n)\leq n^{-1-2/d} 
\end{equation}
Thus, for any $f$, there exist $\theta_{ij}, 1\leq i\leq n, 1\leq j\leq n_i$ such that
\begin{equation}
|f_n-f|\leq  n^{-1/2-1/d}.
\end{equation}

 