\section{Data space, feature space and relevant mappings}\label{sec:spaces}
Given a data
\begin{equation}
	\label{data-f}
	f \in \mathbb{R}^{m \times n \times c}, 
	\quad \text{or}\quad [f]_i \in
	\mathbb{R}^{m\times n}, \quad i = 1:c,
\end{equation}
where $m\times n$ is called the spatial dimension and $c$ is the 
channel dimension.

For the given data $f$ in \eqref{data-f}, we look for some
feature vector, denoted by $u$,  associated with $f$:
\begin{equation}
	\label{u}
	u \in \mathbb{R}^{m \times n \times h}.
\end{equation}
We make an assumption that the data $f$ and feature $u$ are related by a mapping 
(which can be either linear or nonlinear)
\begin{equation}
	\label{u}
	A:  \mathbb{R}^{m \times n \times h}\mapsto \mathbb{R}^{m \times n \times c}, 
\end{equation}
so that
\begin{equation}
	\label{Auf}
	A(u)=f. 
\end{equation}
A mapping 
\begin{equation*}
	B : \mathbb{R}^{m\times n\times c} \mapsto \mathbb{R}^{m \times n\times h},
\end{equation*}
is called a feature extractor if $B \approx A^{-1}$ and 
\begin{equation}\label{vBf}
	v = B(f),
\end{equation}
is such that $v \approx u$.

The data-feature relationship \eqref{Auf} or \eqref{vBf} is not
unique.   Different relationships give rise to different features. 
We can view the data-feature relationship given in \eqref{Auf}
as a model that we propose.  Here the mapping $A$, which can be either
linear or nonlinear, is unknown and needs to be trained.  


\subsection{Some linear and nonlinear mappings and extractors}
A data-feature map $A$ and feacture extractor $B$ can be either
linear or nonlinear.   The nonlinearity can be obtained from
appropriate application of an activation function
\begin{equation}
	\label{act}
	\sigma: \mathbb{R} \to \mathbb{R} .
\end{equation}
In this paper, we mainly consider a special activation function, known 
as the {\it rectified linear unit} (ReLU), which is defined by
\begin{equation}
	\label{relu}
	\sigma(x)= {\rm ReLu}(x) :=\max(0,x), \quad x\in\mathbb{R}. 
\end{equation}
By applying the function to each component, we can extend this
\begin{equation}
	\label{vector-act}
	\sigma:\mathbb R^{m\times n\times c}\mapsto \mathbb R^{m\times n\times c}.  
\end{equation}


A linear data-feature mapping can simply given by a convolution as in \eqref{con1}:
\begin{equation}
	\label{linearA}
	A(f)=A\ast f
\end{equation}
A nonlinear mapping can be given by compositions of convolution and
activation functions:
\begin{equation}
	\label{nonlinearA}
	A=\xi\circ\sigma\circ\eta ,
\end{equation}

and 
\begin{equation}
	\label{extractor}
	B=\sigma\circ \gamma \circ\sigma  .
\end{equation}
Here $\xi$, $\eta$ and $\gamma$ are all 
appropriate convolution mappings.
