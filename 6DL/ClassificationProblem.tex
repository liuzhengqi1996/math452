\section{A basic machine learning problem: image classification}
Classification problem is a very important part of machine learning. Its goal is to determine which class a new data belongs to based on a set of training data whose class is known. For example, in mail management, classifying an e-mail as ``spam" or ``non-spam" is a typical binary classification problem, and the classification of credit rating of credit card customers by banks belong to multiple classification problems. More specifically, in order to understand image classification problem, let us pose the following question:


\begin{itemize}
	\item Given a set of images of cat, dog and rabbit, how does a machine classify the three different classes?
\end{itemize}

\begin{figure}[H]
	\begin{center}
		\includegraphics[width=.4\textwidth, height=.2\textheight]{figures/cat-dog-1.png}
	\end{center}
\end{figure}

\break

First of all, we introduce how an image is represented on computer. Mathematically, {gray-scale image can be considered as a matrix  in $ \mathbb{R}^{n_0\times n_0}$ shown below, where the left one is the image from human vision and the right one is the matrix represented in computer. Each entry in this $ \mathbb{R}^{n_0\times n_0}$ corresponds to the value of a pixel belong to $[0,255]$.
	\begin{center}
		\includegraphics[width=.4\textwidth, height=.2\textheight]{6DL/figures/gray-1.png}
	\end{center}
	
	A color image can be taken as 3D tensor (matrix with $3$ channels (RGB)) in $ \mathbb{R}^{n_0\times n_0 \times 3}$
	\begin{center}
		\includegraphics[width=.4\textwidth, height=.2\textheight]{6DL/figures/corlor-1.png}
	\end{center}	
	\break
	
	Then, let us think about the image classification problem of cat, dog and rabbit. Each image is a big vector of pixel values, for example
	$$ d=1280\times720\times 3  (\text{width} \times \text{height} \times \text{RGB channel}) \approx 3\text{M},$$
	which can be considered as a point $x \in \mathbb{R}^d$. The question becomes: given 3 different sets of points (cat, dog and rabbit) in $\mathbb{R}^d$, how does a machine classify them?
	\begin{figure}[H]
		\begin{center}
			\includegraphics[width=.3\textwidth, height=.3\textwidth]{figures/cat-dog-2.png}   \quad \quad  \quad \quad
			\includegraphics[width=.3\textwidth, height=.3\textwidth]{figures/cat-dog-3.png}
		\end{center}
	\end{figure}
	
	\noindent To answer this question, we consider a mathematical problem: Find $f(\cdot; \bm \theta): \mathbb{R}^d \to \mathbb{R}^3$ such that:
	$$
	f(\includegraphics[width=.07\textwidth]{figures/cat.png}; \Theta)
	\approx \begin{pmatrix}
	1\\ 0 \\ 0
	\end{pmatrix}
	,\quad
	f(\includegraphics[width=.07\textwidth]{figures/dog.png}; \Theta)
	\approx \begin{pmatrix}
	0\\ 1 \\ 0
	\end{pmatrix}
	,\quad
	f(\includegraphics[width=.07\textwidth]{figures/rabbit.png};
	\Theta) \approx
	\begin{pmatrix}
	0\\ 0 \\ 1
	\end{pmatrix}
	.
	$$
	$f(\cdot; \bm \theta)$ maps a given image to a 3-dimensional vector, which is a probability distribution  $\begin{pmatrix}
	p_1\\ p_2 \\ p_3
	\end{pmatrix}$
	, where $p_1, p_2, p_3$ are probabilities of the given image being cat, dog, rabbit, respectively. For example
	
	$$
	f(\includegraphics[width=.07\textwidth]{figures/cat.png}; \bm \theta)
	=
	\begin{pmatrix}
	0.7\\ 0.2 \\ 0.1
	\end{pmatrix}
	\quad \Longrightarrow \quad \includegraphics[width=.07\textwidth]{figures/cat.png}= {\rm cat}.
	$$
	$f$ is a classifier that can be used to classify what a given image is.
	


\break
\section{Some popular data sets in image classification}
In this subsection, we will introduce some popular and standard data sets
in image classification. The most popular four data sets, MNIST, CIFAR-10, CIFAR-100 and ImageNet are shown in Table \ref{popular_dataset}.

\begin{table}[H]
	\centering
	\begin{tabular}{|c|c|c|c|c|c|}
		\hline
		% after \\: \hline or \cline{col1-col2} \cline{col3-col4} ...
		dataset &  training (N) & test (M)   & classes (k) & channels (c)& input size (d)\\\hline
		MNIST	&	60K	&	10K	&	10	& Greyscale & 28*28  \\\hline
		CIFAR-10	&	50K 	&	10K	&	10 & RGB & 32*32   \\\hline
		CIFAR-100	&	50K 	&	10K	&	100 & RGB & 32*32   \\\hline
		ImageNet	&	1.2M 	&	50K	&	1000 &	RGB & 224*224  \\\hline
	\end{tabular}
	\caption{Basic descriptions about popular datasets }
	\label{popular_dataset}
\end{table}

\break
\subsection{MNIST (Modified National Institute of Standards and Technology Database)}
MNIST\cite{lecun1998mnist} is a database for handwritten digits. It is a simple database for people who want to try learning techniques and pattern recognition methods on real-world data while spending minimal efforts on preprocessing and formatting. In order to use MNIST for the classification problem, the following setup is often used:
\begin{itemize}
	\item Training set : $N = 60,000$;
	\item Test set : $M = 10,000$;
	\item Image size : $d =28*28*1=784$;
	\item Classes: $ k = 10$;
\end{itemize}
\begin{figure}[H]
	\begin{center}
		\includegraphics[height=.2\textheight]{mnist_short.png}
		\caption{Some images in MNIST.}
	\end{center}
\end{figure}
The following example shows  an image of handwritten digits is represented mathematically
\break
$$
x=\adjustbox{valign=c}{\includegraphics[height=.03\textheight]{mnist2_1.jpg}}
=
\begin{pmatrix}
  x_1\\
x_2\\
\vdots\\
x_{784}
\end{pmatrix}
\in \mathbb R^{784}.
$$
MNIST has 10 classes denoted by $\{A_k\}_{k=1}^{10}$ as shown in the following, where $A_{k}$ is the set of handwritten digits $k$ for $k=1,2,3,...,9$ and $A_{10}$ is the set of handwritten digits $0$
\begin{equation}
  \label{A2}
A_2=
\left\{
\adjustbox{valign=c}{\includegraphics[height=.03\textheight]{mnist2_1.jpg}},
\adjustbox{valign=c}{\includegraphics[height=.03\textheight]{mnist2_2.jpg}},\cdots
%\adjustbox{valign=c}{\includegraphics[height=.03\textheight]{mnist2_3.jpg}}, \cdots
\right\},~
%\subset \mathbb R^{784}
A_{9}=
\left\{
\adjustbox{valign=c}{\includegraphics[height=.03\textheight]{mnist9_1.jpg}},
\adjustbox{valign=c}{\includegraphics[height=.03\textheight]{mnist9_2.jpg}},\cdots
%\adjustbox{valign=c}{\includegraphics[height=.03\textheight]{mnist9_3.jpg}}, \cdots
\right\},~
A_{10}=
\left\{
\adjustbox{valign=c}{\includegraphics[height=.03\textheight]{mnist0_1.jpg}},
\adjustbox{valign=c}{\includegraphics[height=.03\textheight]{mnist0_2.jpg}}, \cdots
%\adjustbox{valign=c}{\includegraphics[height=.03\textheight]{mnist0_3.jpg}}, \cdots
\right\}
\subset \mathbb R^{784}.
\end{equation}



\break
\subsection{CIFAR}
\paragraph{CIFAR-10}
CIFAR-10\cite{krizhevsky2009learning} is a set of images that can be used to teach a computer how to recognize objects. It contains 60,000 32x32 color images in 10 different classes,  with 6000 images per class. The 10 different classes represent airplanes, cars, birds, cats, deer, dogs, frogs, horses, ships, and trucks. In order to use CIFAR-10 for the classification problem, people usually use the following setup:
\begin{itemize}
	\item Training set : $N = 50,000$
	\item Test set : $M = 10,000$
	\item Image size : $d  = 32*32*3$
	\item Classes: $k = 10$
\end{itemize}
\begin{figure}[H]
	\begin{center}
		\includegraphics[height=.26\textheight]{cifar10.jpg}
		\includegraphics[height=0.26\textheight]{cifar100.png}
		\label{Fig: CIFAR-10}
		\caption{Left: Some images in CIFAR-10. Right: Some images in CIFAR-100.}
	\end{center}
\end{figure}


\break

\paragraph{CIFAR-100}
CIFAR-100\cite{krizhevsky2009learning} is just like the CIFAR-10, except it has 100 classes containing 600 images each. There are 500 training images and 100 testing images per class. The 100 classes in the CIFAR-100 are grouped into 20 superclasses and each superclass has 5 classes. Each image comes with a ``fine" label (the class to which it belongs) and a ``coarse" label (the superclass to which it belongs). In order to use CIFAR-100 for the classification problem, people usually use the following setup:

\begin{itemize}
	\item Training set : $N = 50,000$
	\item Test set : $M = 10,000$
	\item Image size : $d = 32*32*3$
	\item Classes: $k = 100$
	\end{itemize}

\subsection{ImageNet}
The ImageNet\cite{deng2009imagenet} project is a large visual database designed for use in visual object recognition software research. More than 1 million images have been hand labeled to indicate what objects are pictured. It is organized according to the WordNet hierarchy (currently only the nouns), in which each node of the hierarchy is depicted by hundreds and thousands of images.
Since 2010, the ImageNet runs an annual software contest, the ImageNet Large Scale Visual Recognition Challenge (ILSVRC), where software programs compete to correctly classify and detect objects and scenes. In order to use ImageNet for the classification problem, we list the setup of ILSVRC2012 in the following
\begin{itemize}
	\item Training set : $N = 1,200,000$
	\item Test set : $M = 50,000$
	\item Image size : $d = 224*224*3$
	\item Classes: $k = 1,000$
\end{itemize}
\begin{figure}[H]
	\begin{center}
\includegraphics[height=0.22\textheight]{imagenet-example.png}
		\caption{Some images in ImageNet.}
	\end{center}
\end{figure}
\break

\endinput
