
\section{1D example}
Consider the 1D case of Theorem \ref{lem:stratifiedapprox2}
\begin{theorem}
Suppose $x\in [-T, T]$ and $f\in \mathcal{B}^{m+1, 1}([-T, T])$.
%$$
% \int_{\mathbb{R}^{d}} |\hat{f}(\omega)|\|\omega\|_{\ell_q}^{m+1} d\omega<\infty.
%$$
There exist $\beta_j\in \{-1, 1\}$, $\bar \omega_j\in\{1, -1\}$, $t\in [0,T]$ such that 
\begin{equation}\label{1drepresentation}
f_n(x)= \sum_{k\le m}{1\over k!}f^{(k)}(0) x^k + {\nu\over m!n}\sum_{j=1}^{n}\beta_j (\bar \omega_j x - t_j)_+^m
\end{equation} 
with $\nu=\int_{\{-1,1\}\times [0,T]\times \mathbb{R}^{d}} \rho(\theta)d\theta$ and $\rho(\theta)$  defined in \eqref{eq:straglam} 
satisfies the following estimate
\begin{equation}
\|f - f_n \|_{H^k(\Omega)} \leq C(m,k,\Omega) n^{-{1\over 2}-{1\over d+2}}\|f\|_{\mathcal B^{m+1, q}(\Omega)},\qquad k\le m.
\end{equation} 
Especially,
\begin{equation}
\|f - f_n \|_{L^2(\Omega)} \leq {(2T)^m|\Omega|^{1\over 2}\over (m-1)!} n^{-{1\over 2}-{1\over d+2}}\|f\|_{\mathcal B^{m+1, q}(\Omega)}.
\end{equation} 
%\begin{equation}
%\|D^\beta (f(x)- f_n(x))\|_{L^2(\Omega)}\le \sqrt{2^{m-k-2}(2m-k)\over k!(m-k)!}|\Omega|^{1/2} n^{-{1\over 2}-{1\over d}},\quad |\beta|=k\le m.
%\end{equation}
\end{theorem} 

If $\bar \omega_j=-1$,
\begin{equation} 
\beta_j (\bar \omega_j  x - t_j)_+^m = \beta_j (- (x - (-t_j)))_+^m.
\end{equation} 
If $m$ is odd, 
\begin{equation} 
\beta_j (\bar \omega_j  x - t_j)_+^m = - \beta_j (x - (-t_j))_-^m = \tilde \beta_j(x - \tilde t_j)_-^m
\end{equation} 
with $\tilde \beta_j=-\beta_j\in \{-1, 1\}$ and $\tilde t_j\in[-T, 0]$.
If $m$ is even, 
\begin{equation} 
\beta_j (\bar \omega_j  x - t_j)_+^m = \tilde \beta_j (x - \tilde t_j)_-^m
\end{equation} 
with $\tilde \beta_j=\beta_j\in \{-1, 1\}$ and $\tilde t_j\in[-T, 0]$.


\begin{equation}
\begin{split}
&f(x) - \sum_{k\le m}{1\over k!}f^{(k)}(0) x^k 
\\ 
=&{\rm Re} \bigg ({i^{m+1}\over m!}\int_{\mathbb{R}^d} \int_{0}^T\left[(\bar \omega  x - t)_+^me^{i|\omega|t}
+(-1)^{m-1}(-\bar \omega x - t)_+^me^{-i|\omega|t} \right]\hat{f}(\omega)|\omega|^{m+1}dt d\omega\bigg ) 
\end{split}
\end{equation}
If $m$ is odd,
$$
(\bar \omega  x - t)_+^me^{i|\omega|t}
+(-1)^{m-1}(-\bar \omega x - t)_+^me^{-i|\omega|t}
=
(x - t)_+^me^{i\omega t}
+(- x - t)_+^me^{-i\omega t}.
$$
Thus,
\begin{equation}
\begin{split}
&f(x) - \sum_{k\le m}{1\over k!}f^{(k)}(0) x^k 
\\ 
=&{\rm Re} \bigg ({1\over m!}\int_{\mathbb{R}^d} \int_{0}^T\left[(x - t)_+^me^{i \omega t}
+ (- x - t)_+^me^{-i \omega t} \right]\hat{f}(\omega) (i\omega)^{m+1}dt d\omega\bigg ) 
\end{split}
\end{equation}
Note that
\begin{equation}
f^{(m)}(t)=\int (i\omega)^m\hat{f}(\omega)e^{i\omega t}d\omega 
\end{equation} 
It holds that
\begin{equation}
\begin{split}
&f(x) - \sum_{k\le m}{1\over k!}f^{(k)}(0) x^k 
\\ 
=&{\rm Re} \bigg ({1\over m!}\int_{\mathbb{R}^d} \int_{0}^T\left[(x - t)_+^me^{i \omega t}
+ (- x - t)_+^me^{-i \omega t} \right]\hat{f}(\omega) (i\omega)^{m+1}dt d\omega\bigg ) 
\\
=& {1\over m!} \int_{0}^T\left[(x - t)_+^m f^{(m+1)}(t)
+ (- x - t)_+^mf^{(m+1)}(-t) \right] dt  
\end{split}
\end{equation} 

\noindent\textbf{1D case from Taylor expansion}

Consider the case that $\Omega=[0,1]$, the integral form of Taylor remainder reads
\begin{equation}
\begin{aligned}
f(x) - \sum_{k\le m}{1\over k!}f^{(k)}(0) x^k &={1\over m!}\int_0^x f^{(m+1)}(t)(x-t)^mdt
\\
&={1\over m!}\int_0^1 f^{(m+1)}(t)(x-t)_+^mdt.
\end{aligned}
\end{equation}
Let the probability density be
\begin{equation}
\lambda(t)= {|f^{(m+1)}(t)|\over \|f^{(m+1)}\|_{L^1}}.
\end{equation}
Thus,
\begin{equation}
\begin{aligned}
f(x) - \sum_{k\le m}{1\over k!}f^{(k)}(0) x^k &={ \|f^{(m+1)}\|_{L^1}\over m!}\int_0^1 sgn(f^{(m+1)}(t))(x-t)_+^m\lambda(t)dt.
\end{aligned}
\end{equation}
\begin{lemma}
	 There exists 
	 \begin{equation}
\begin{aligned}
f_n(x) = \sum_{k\le m}{1\over k!}f^{(k)}(0) x^k + { \|f^{(m+1)}\|_{L^1}\over m!n}\sum_{i=1}^n\beta_i  (x-t)_+^m.
\end{aligned}
\end{equation}
with $\beta_i\in [-2,2]$
 such that
	\begin{equation} 
	\|f-f_n\|_{H^k(\Omega)} \leq n^{-3/2}|\Omega|^{1/2}.
	\end{equation}  
	\end{lemma}

\begin{proof}
Note that $\displaystyle f(x)-f_n(x)={ \|f^{(m+1)}\|_{L^1}\over m!}\left(r(x) - r_n(x)\right)$ with
\begin{equation}
r(x) = \int_0^1 sgn(f^{(m+1)}(t))(x-t)_+^m\lambda(t)dt,\quad r_n(x)= {1\over n}\sum_{i=1}^n\beta_i  (x-t)_+^m.
\end{equation}
Consider a $\epsilon$-covering decomposition $G=\cup_{i=1}^M G_i$ of $\Omega$ such that 
\begin{equation}
sgn(f^{(m+1)}(t))= sgn(f^{(m+1)}(t')), \quad |t-t'|<\epsilon, \quad \forall t,\ t'\in G_i.
\end{equation}
Let $n_j=\lceil \lambda(G_j)n\rceil$ and $t_{i,j} \in G_j(1\leq i\leq n_j)$ and 
\begin{equation}
r(x) = \int_0^1 g(x, t)\lambda(t)dt,\quad r_n(x)= \sum_{i=1}^M \lambda(G_i) g_{n_i}^i,\quad \mbox{ with }\ g(x, t)=(x-t)_+^m,\ g_{n_i}^i ={1\over n_i}  \sum_{j=1}^{n_i}(x-t_{ij})_+^m.
\end{equation}
Define
\begin{equation}
\begin{aligned}
f_n(x) = \sum_{k\le m}{1\over k!}f^{(k)}(0) x^k + { \|f^{(m+1)}\|_{L^1}\over m!}r_n(x).
\end{aligned}
\end{equation}
Thus, $\displaystyle f(x)-f_n(x)={ \|f^{(m+1)}\|_{L^1}\over m!}\left(r(x) - r_n(x)\right)$. Note that 
\begin{equation}
\begin{aligned}
r(x) = \mathbb{E}_G g =  \sum_{i=1}^{M}\lambda(G_i) \mathbb{E}_{G_i} g.
\end{aligned}
\end{equation}
By Lemma \ref{MC},
\begin{equation}
\begin{split}
\mathbb{E}_n\| r^{(k)} - r_n^{(k)} \|_{L^2(\Omega)}^2=& 
 \sum_{j=1}^{M}{\lambda^2(G_i) \over n_i}\mathbb{E}_{G_i}\|\mathbb{E}_{G_i} \partial_x g -  \partial_x g^{(k)}\|^2_{L^2(\Omega)}
\\
\le & \sum_{i=1}^{M} {\lambda^2(G_i)\over n_i}\sup_{t,t'\in G_i} \|\partial_x  g(x, t) - \partial_x g(x,t')\|^2_{L^2(\Omega)}.
\end{split}
\end{equation} 
Since ${\lambda(G_i)\over n_i}\le {1\over n}$ and $\displaystyle \sum_{i=1}^M \lambda(G_i)=1$,
\begin{equation}
\mathbb{E}_n\| r^{(k)} - r_n^{(k)} \|_{L^2(\Omega)}^2\leq n^{-1}\max_{1\le i\le M}\sup_{t,t'\in G_i} \| \partial_x g(x,t) - \partial_x g(x,t')\|^2_{L^2(\Omega)}.
\end{equation}
Notice that for any $1\le i\le M$,
\begin{equation}
\sup_{t,t'\in G_i} \| \partial_x g(x,t) - \partial_x g(x,t')\|^2_{L^2(\Omega)}\lesssim |t-t'|^2|\Omega| \le | \Omega| \epsilon^2.
\end{equation}
Since $\epsilon \sim M^{-1}$,
there exist $\{t_{i,j}^\ast\}$ such that $t_{i,j}^\ast\in G_i$ and 
\begin{equation}
\| f^{(k)} -  f_n^{(k)} \|_{L^2(\Omega)}^2\leq (M^2n)^{-1}| \Omega|.
\end{equation}
Suppose $\displaystyle N=\sum_{j=1}^M n_j$, $n\le N\le n+M$,
$$
r_n(x) =  {1\over N}\sum_{i=1}^{M}\frac{N\lambda(G_i)}{n_i}\sum_{j=1}^{n_i} (x - t_{i,j}^\ast)_+^m
 =  {1\over N}\sum_{i=1}^{M}\beta_{i,j}\sum_{j=1}^{n_i} (x - t_{i,j}^\ast)_+^m
$$ 
with
\begin{equation}
\beta_{i,j}= \frac{N\lambda(G_j)}{n_j}\le \frac{\lambda(G_j)(M+n)}{\lambda(G_j)n}\le {M+n\over n}.
\end{equation}
Let $M=n$. There exist $\{t_{i,j}^\ast\}$ such that $t_{i,j}^\ast\in G_i$ and 
\begin{equation}
\| f -  f_n \|_{H^k(\Omega)}^2\leq N^{-3}| \Omega|,
\end{equation}
which completes the proof. 
\end{proof}

\begin{theorem} 
	 There exists 
	 \begin{equation}
\begin{aligned}
f_n(x) = \sum_{k\le m}{1\over k!}f^{(k)}(0) x^k + { \|f^{(m+1)}\|_{L^1}\over m!n}\sum_{i=1}^n (x-t)_+^m.
\end{aligned}
\end{equation}
 such that
	\begin{equation} 
	\|f-f_n\|_{H^k(\Omega)} \leq n^{-{3\over 4}}|\Omega|^{1/2}.
	\end{equation}  
\end{theorem} 

\begin{proof}
Consider the same decomposition $G=G_1\cup \cdots \cup G_M$ of $\Omega$ such that
$$
|t -t'| \leq \epsilon \le M^{-1},\quad t, t'\in G_i.
$$ 
Let  $t_{i,j} \in G_i(1\leq j\leq n_i)$,  $n_i$ equal $\lceil \lambda(G_i)n\rceil$ and $\lfloor \lambda(G_i)n\rfloor$ with probabilities chosen to make its mean equal to $\lambda(G_i)n$ and $m_i=n_i + \mathbb{I}(n_i=0)$. Then
\begin{align} 
\sum_{i=1}^M m_i&=\sum_{i=1}^M n_i\mathbb{I}(n_i>0) + \sum_{i=1}^M n_i\mathbb{I}(n_i=0)
\\
&\le \sum_{i=1}^M (n\lambda(G_i) + 1)\mathbb{I}(n_i>0) + \sum_{i=1}^M \mathbb{I}(n_i=0)
\\
&\le  n\sum_{i=1}^M \lambda(G_i)\mathbb{I}(n_i>0) + M
\le n+M.
\end{align} 
Define $\displaystyle N=\sum_{i=1}^Mn_i$  and
$$
r_{n}(x)= {1\over N}\sum_{i=1}^M {n_i\over m_i} \sum_{j=1}^{m_i} g(x, t_{i,j}).
$$
By the definition of $m_i$, $\displaystyle {n_i\over m_i}=0$ or $1$.  This means that $r_{n}(x) $ is in the form of  $\displaystyle {1\over N}\sum_{i=1}^Ng(x,\theta_i)$.  Define
$$
\bar{r}_{n}(x)= \sum_{i=1}^M {n_i\over N}\mathbb{E}_{G_i}g.
$$
Since
$
\displaystyle r(x)=\mathbb{E}_G g= \sum_{i=1}^M \lambda(G_i) \mathbb{E}_{G_i}g,
$  
\begin{align}  
r_{n} - r  &= r_{n} - \bar{r}_{n} + \bar{r}_{n} - r
\\
&= {1\over N}\sum_{i=1}^M {n_i\over m_i} \sum_{j=1}^{m_i} \big (g(x,t_{i,j}) - \mathbb{E}_{G_i}g \big ) + {1\over N}\sum_{i=1}^M (n_i - \lambda(G_i)N)  \mathbb{E}_{G_i}g.
\end{align}  
It follows that  
\begin{align}  
\|r_{n} - r \|_{L^2(\Omega)}^2 &
\le {1\over N^2}\sum_{i=1}^M {n_i\over m_i} \sum_{j=1}^{m_i} \|g(x,t_{i,j}) - \mathbb{E}_{G_i}g \|_{L^2(\Omega)}^2 
+  {1\over N^2}\sum_{i=1}^M (n_i - \lambda(G_i)N)^2  \|\mathbb{E}_{G_i}g\|_{L^2(\Omega)}^2.
\end{align} 
For the first term on the right-hand side of the above equation,
\begin{align}  
{1\over N^2}\sum_{i=1}^M {n_i\over m_i} \sum_{j=1}^{m_i} \|g(x,t_{i,j}) - \mathbb{E}_{G_i}g \|_{L^2(\Omega)}^2 
\le {|\Omega|\epsilon^2\over N^2}\sum_{i=1}^M n_i ={|\Omega|\epsilon^2\over N}.
\end{align} 
Recall 
$
g(x, t)= (x - t)_+^m {\rm sgn} s(zt,\omega),
$
which is bounded, say $|g(x, t)|\le C$. Note that $|n-N|\le M$. Thus,
\begin{align}
 {1\over N^2}\sum_{i=1}^M (n_i - \lambda(G_i)N)^2  \|\mathbb{E}_{G_i}g\|_{L^2(\Omega)}^2& \le {1 \over N^2}\sum_{i=1}^M  (M^2 + {1\over \lambda^2(G_i)}) \int_\Omega \int_{G_i} g^2(x, t)\lambda^2(t)dt d\mu(x)
\\
&\le  {C^2|\Omega|\over N^2}(M^2 +M)\le {\tilde C\over N^2}M^2.
\end{align}  
It follows that 
\begin{align}  
\mathbb{E}_n \|r_{n} - r \|_{L^2(\Omega)}^2 
\le {\epsilon^2 \over N} + {\tilde C\over N^2}M^2\label{stratify2t}.
\end{align}  
Choose $\displaystyle {\epsilon^2 \over N} = {\tilde C\over N^2}M^2$, then $M^2\sim N\epsilon^2$ and 
$$
\mathbb{E}_n \|r_{n} - r \|_{L^2(\Omega)}^2 \le {2\epsilon^2 \over N} \lesssim N^{-{3\over 2}}.
$$ 
There exist $\{t_{i,j}^\ast\}$ such that $t_{i,j}^\ast\in G_i$ and 
\begin{equation}
\| f -  f_n \|_{L^2(\Omega)}\le {\nu\over m!}\|r_{n} - r\|_{L^2(\Omega)}\lesssim N^{-{3\over 4}}.
\end{equation}
which completes the proof. 

\end{proof}




\hrule

\vspace{1cm}
If $x>0$,
\begin{equation}
\begin{aligned}
f(x) - \sum_{k\le m}{1\over k!}f^{(k)}(0) x^k &={1\over m!}\int_0^T f^{(m+1)}(t)(x-t)_+^mdt;
\end{aligned}
\end{equation}
If $x<0$, since $u_-=-(-u)_+$ ,
\begin{equation}
\begin{aligned}
f(x) - \sum_{k\le m}{1\over k!}f^{(k)}(0) x^k &=-{1\over m!}\int_0^{-x} f^{(m+1)}(-t)(x+t)^mdt
\\
&=-{1\over m!}\int_0^T f^{(m+1)}(-t)(x+t)_-^mdt
\\
&={(-1)^{m+1}\over m!}\int_0^T f^{(m+1)}(-t)(-x-t)_+^mdt
\end{aligned}
\end{equation}
This implies that
\begin{equation}
\begin{aligned}
f(x) - \sum_{k\le m}{1\over k!}f^{(k)}(0) x^k 
&={1\over m!}\int_0^T f^{(m+1)}(t)(x-t)_+^m + (-1)^{m+1}f^{(m+1)}(-t)(-x-t)_+^mdt
\\
&={1\over m!}\int_{\{-1,1\}}\int_0^T z^{m+1}f^{(m+1)}(zt)(zx-t)_+^m dtdz
\\
&={1\over m!}\int_{\{-1,1\}}\int_0^T |f^{(m+1)}(zt)| s(z, t)(zx-t)_+^m dtdz
%\\
%&={1\over m!}\int_0^T |f^{(m+1)}(t)|s_1(t)(x-t)_+^m + |f^{(m+1)}(-t)|s_2(t)(-x-t)_+^mdt
\end{aligned}
\end{equation}
with $s(z, t)= z^{m+1}sgn (f^{(m+1)}(zt))$.
%with $s_1(t)=sgn (f^{(m+1)}(t))$ and $s_2(t)=sgn ((-1)^{m+1}f^{(m+1)}(-t))$. 
It can be written as 
\begin{equation}
\begin{aligned}
f(x) - \sum_{k\le m}{1\over k!}f^{(k)}(0) x^k &={\|f^{(m+1)}\|_{L^1}\over m!}\int_{[0,T]\times \{-1, 1\}}s(z,t)(zx-t)_+^m \lambda(z,t)dtdz
\end{aligned}
\end{equation}
where 
\begin{equation}
\begin{aligned}
\lambda(z,t)={|f^{(m+1)}(zt)|\over \int_{[0,T]\times \{-1, 1\}} |f^{(m+1)}(zt)| dt dz}
\end{aligned}
\end{equation}
By Monte Carlo, there exist $t_j\in [0, T]$ and $\beta_j\in \{-1, 1\}$ such that
\begin{equation}
\begin{aligned}
\|f(x) - f_n(x)\|_{L^2} \le {1\over n^{1/2}}
\end{aligned}
\end{equation}
with
\begin{equation}
\begin{aligned}
f_n(x) = \sum_{k\le m}{1\over k!}f^{(k)}(0) x^k + {\|f^{(m+1)}\|_{L^1}\over m!n}\sum_{j=1}^n\beta_j (z_jx-t_j)_+^m
\end{aligned}
\end{equation}

\noindent\textbf{Higher dimension case from Taylor expansion}

The Taylor expansion reads
\begin{equation}
f(x)=f(0)+\sum_{1\le |\alpha| \leqslant k} \partial^\alpha f(0) x^{\alpha}+\sum_{|\alpha|=k+1} \frac{k+1}{\alpha !} x^{\alpha} \int_{0}^{1}(1-t)^{k} \partial^\alpha f(t x) \quad d t
\end{equation}
Let $y=tx$.
\begin{equation}
\begin{aligned}
x^{\alpha}(1-t)^{k} \partial^{\alpha} f(t x) d t &=(1-t)^{k} \partial^{\alpha} f(y) \prod_{i=1}^{N} x^{\alpha_{i}} d t\\
&=\partial^{\alpha} f(y) (1-t)^{k} \frac{y^{\alpha}}{t^{k+1}} d t \\
&=\partial^{\alpha} f(y)  \frac{1}{t} \prod_{i=1}^{N}\left(\frac{1}{t}-1\right)^{\alpha_{i}} y_{i}^{\alpha_{i}} d t \\
&=\partial^{\alpha} f(y)  \frac{1}{t} \prod_{i=1}^{N}\left(x_{i}-y_{i}\right)^{\alpha_{i}} d t \\
&=\partial^{\alpha} f(y)  (x-y)^{\alpha} \frac{1}{t} d t
\end{aligned}
\end{equation}
Thus 
\begin{equation}
f(x)=f(0)+\sum_{1\le |\alpha| \leqslant k} \partial^\alpha f(0) x^{\alpha}+\sum_{|\alpha|=k+1} \frac{k+1}{\alpha !}  \int_{0}^{\infty} \partial^\alpha f(y) (x-y)_+^\alpha  d s
\end{equation}
with 
\begin{equation}
(x-y)_{+}^{\alpha}=\left\{\begin{array}{ccc}
(x-y)^{\alpha} & \text { if }  
y=sx,\ 0\le s \le 1 
\\
0 &  \text { othemise } 
\end{array}\right.
\end{equation}


\noindent\textbf{Higher dimension case from Taylor expansion 2}









