
\section{Lower bound  of finite element error estimates}
%{Notation and basic results}
In this section, we will indicate that the convergence rate is optimal by showing the lower bound of finite element error estimates. 
%\subsection{Lower bound of finite element approximation in any dimension}
%In this subsection, 
We will provide lower bound for finite element error estimates that are obtained by Lin, Xie and Xu~\cite{lin2014lower}.
\begin{theorem}\label{Lower_Bound_linear}
Assume $u\in H^{3}(\Omega)$ and $u$ is not linear %$u\notin V_h$ 
and for all $v_h\in V_h$ being the linear finite element function.
The following lower bound of the approximation error holds when the family of mesh $\mathcal{T}_h$
is quasi-uniform %the mesh size is $h$ is sufficiently small:
\begin{equation}\label{lower_convergence_0_1}
\inf_{v_h\in V_h}\|u-v_h\|_{r}\geq Ch^{2-r}\approxeq N^{-\frac{2-r}{d}}\quad r=0, 1,
\end{equation}
where $C$ is dependent on $u$.
\end{theorem}
\begin{proof}
Let $\Pi_2$ is the nodal interpolation to quadratic finite element,  and denote $\displaystyle |\cdot|_{2,h}=
\left(\sum_{\tau \in \mathcal T_h} |\cdot|^2_{2,\tau}\right)^{\frac12}$. For any $v_h\in V_h$, since $v_h$ is
linear, by the approximation property of $\Pi_2$ and inverse inequality, we have 
\begin{equation}
\begin{aligned}
|u|_{2,\Omega}&=|u-v_h|_{2,h}\le |u-\Pi_2 u|_{2,h}+|\Pi_2u-v_h|_{2,h}\\
& \le C_1h |u|_{3,\Omega}+C_2h^{r-2}\|\Pi_2u-v_h\|_{r,\Omega}\\
& \le C_1h |u|_{3,\Omega}+C_2h^{r-2}\|\Pi_2u-u\|_{r,\Omega} +C_2h^{r-2}\|u-v_h\|_{r,\Omega}\\
& \le C_1h |u|_{3,\Omega}+C_3h^{r-2}h^{3-r}|u|_{3,\Omega} +C_2h^{r-2}\|u-v_h\|_{r,\Omega}\\
\end{aligned}
\end{equation}
namely 
\begin{equation}
\begin{aligned}
|u|_{2,\Omega}-(C_1+C_3)h|u|_{3,\Omega}\le C_2h^{r-2}\|u-v_h\|_{r,\Omega}.
\end{aligned}
\end{equation}
Let $h_0(u)=\frac{|u|_{2,\Omega}}{2 (C_1+C_3)|u|_{3,\Omega}}$ and we have
$$
|u|_{2,\Omega}-(C_1+C_3)h_0|u|_{3,\Omega}=\frac12|u|_{2,\Omega}.
$$
Noting that $v_h\in V_h$ is arbitrary, then if $h\le h_0$, we have  
\begin{equation}
\inf_{v_h\in V_h}\|u-v_h\|_{r,\Omega}\geq Ch^{2-r}.
\end{equation}
If $\mathcal{T}_h$ is a triangulation with mesh size $h> h_0$, then by the newest vertex bisection strategy shown in \cite{stevenson2008completion}, we 
can refine $\mathcal{T}_h$ such that the new triangulation 
$\mathcal{T}_{h_0}$ with mesh 
size decreasing to the order of $h_0$  and we have $V_h\subset V_{h_0}$, which implies 
\begin{equation}
\inf_{v_h\in V_h}\|u-v_h\|_{r,\Omega}\geq \inf_{v_{h_0}\in V_{h_0}}\|u-v_{h_0}\|_{r,\Omega} \geq Ch_0^{2-r}.
\end{equation}
Therefore \eqref{lower_convergence_0_1} is desired. 
\end{proof}

%\begin{conjecture}
%	Let $V_N$ be  linear finite element spaces associated with 
%	shape-regular simplicial grids $\mathcal T_N$ of $N$-elements (or
%	$N$-grid points) in a polyhedral domain $\Omega\subset \mathbb R^d$,
%	say $\Omega=(0,1)^d$.  Then,  for any reasonable function (which is,  say,
%	not locally linear, $|u|_{2}\neq 0$), then the following lower bound holds:
%	\begin{equation}
%	\label{optimalFEMerror}
%	\inf_{\# \mathcal T_N=\mathcal O(N)} \inf_{v_N\in
%		V_N}\|u-v_N\|_{0,\Omega}\ge c(u) N^{-2/d} 
%	%\le C(u) N^{-2/d}
%	\end{equation}
%\end{conjecture}
%
%Questions:
%\begin{enumerate}
%	\item Is the above conjecture correct in some way?  If yes, what are
%	the more rigorous statements for such results?
%	\item What are the most relevant references that contain such results?
%	\item If the conjecture is incorrect, is there a counter example?
%\end{enumerate}


\begin{conjecture}\label{conjecture1}
For any function $u\in L^2(\Omega)$ % is not smooth (for example $u\in H^{1+\epsilon}$ for any $\epsilon>0$)
 that is not locally linear (namely $u$ is not
linear in any open subset), we  have 
\begin{equation}\label{lower_convergence_0_2}
\inf_{v_h\in V_N}\|u-v_h\| \gtrsim  N^{-\frac{2}{d}}
\end{equation}
for any finite element space $V_N$ of dimension $N$ on a shape-regular grid.
\end{conjecture}
%Questions:
%\begin{enumerate}
%	\item Is the above conjecture correct in some way?  If yes, what are
%	the more rigorous statements for such results?
%	\item What are the most relevant references that contain such results?
%	\item If the conjecture is incorrect, is there a counter example?
%\end{enumerate}
%\begin{conjecture}\label{con2}
%Let $\mathcal  T_N$ be a grid of $\Omega\subset \mathbb R^d$ with $N$ elements and $V_h$ be the finite element space on $\mathcal  T_N$. 
%If $u$ is a smooth function (for example $u\in H^3(\Omega)$) and the optimal partition $\mathcal  T^*_N$ such that 
%\begin{equation}\label{lower_convergence_0_3}
%\inf_{\# \mathcal T_N=N}\inf_{v_h\in V_h}\|u-v_h\|_{r}=\inf_{v_h\in V^*_h}\|u-v_h\|_{r}\geq  CN^{-\frac{2-r}{d}}\quad r=0, 1
%\end{equation}
%where $V_h^*$ is the finite element space on $\mathcal  T^*_N$, then the partition $\mathcal  T^*_N$ can be a quasi-uniform partition.
%\end{conjecture}

\begin{conjecture}\label{con2}
Let %with $N$ elements 
 $V_N$ be the linear finite element space on a grid $\mathcal  T_N$ of $\Omega\subset \mathbb R^d$. 
If $u$ is a smooth function (for example $u\in H^2(\Omega)$) and $u$ is not a linear function,
\begin{equation}\label{lower_convergence_0_3}
\mathop{\inf_{\# \mathcal T_N=N}}_{\mathcal T_N~\hbox{is shape regular}} \inf_{v_h\in V_N}\|u-v_h\|\gtrsim \mathop{\inf_{\# \mathcal T_N=N}}_{\mathcal T_N~\hbox{is quasi-uniform}} \inf_{v_h\in V_N}\|u-v_h\|\gtrsim N^{-\frac{2}{d}}.
\end{equation}
%where $V_N^*$ is the finite element space on $\mathcal  T^*_N$, then the partition $\mathcal  T^*_N$ can be a quasi-uniform partition.
\end{conjecture}
Questions:
\begin{enumerate}
	\item Is the above Conjecture \ref{conjecture1} or Conjecture \ref{con2} correct in some way?  If yes, what are
	the more rigorous statements for such results?
	\item What are the most relevant references that contain such results?
	\item If the Conjecture \ref{conjecture1} or Conjecture \ref{con2} is incorrect, is there a counter example?
\end{enumerate}


