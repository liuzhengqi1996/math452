\chapter{Universal Approximation}\label{ch:approx}

Three categories of approximation theory.
\begin{enumerate}
\item 
In Barron's paper, there is a section on the lower bound of
approximation using linear subspaces. If the basis is fixed, then the
rate $n^{-1/d}$ is not improvable. The DNN uses a basis adapt to the
function.

The adaptive FEM we have studied before is indeed using linear
subspaces. For a given and fixed basis, select the best n term to
approximate a function. The non-linear approximation theory (by
DeVore) is to relax the smoothness of function to achieve the optimal
rate $n^{-1/d}$ but won't improve the rate.

Now the problem is a truly nonlinear approximation problem, even the
basis can be changed according to $f$. The dimension independent rate
$n^{-1/2}$ seems also optimal. What we can improve is the
characterization of the smoothness.
\end{enumerate}

\subsection{Special Case: 1-hidden Layer}
First, let us define the so-called 1-hidden layer (shallow) neural network.
\begin{definition}The 1-hidden layer (shallow) neural network is defined as:
	$$
	{\rm DNN}_1 = \lbrace \phi~:~ \phi(x) = \sum_{i=1}^N a_i \sigma(w_i x + b_i ) + c, ~~ N \in \mathbb{N}^+\rbrace.
	$$
\end{definition}

To consistent with above notation, we can write it as:
$$
\phi = W^1 x^1 + b^1 = (a_1,\cdots,a_N) \sigma(W^0 x + b^0) + c \in {\rm DNN}_1.
$$

The first question about ${\rm DNN}_1$ is about the approximation properties for any continuous functions. Here we have the next theorem:
\begin{theorem}[Universal Approximation Property of Shallow Neural Networks] Let $\Omega$ be bounded, if any $f\in C(\bar \Omega)$, there exists a sequence $\phi_n \in {\rm DNN}_1$ such that 
	$$
	\max_{x\in \bar \Omega} |\phi_n(x) - f(x)| \to 0, \quad n \to \infty.
	$$
	This provide that $\sigma$ is not a polynomial.
	On the other hand, let $\sigma$ be a non-polynomial Riemann integrable function and $\sigma\in L_{loc}^\infty(\mathbb{R})$ then we have
	$$
	\overline{\rm DNN}_1 = C(\bar \Omega).
	$$
\end{theorem}
Before the proof, let us give some notation.
We use $\mathbb{P}_m(\mathbb{R}^d)$ to define the polynomials
of $d$-variables of degree less than $m$. 
Let $\alpha = (\alpha_1,\cdots,\alpha_d)$( $\alpha_i $ non-negative integers), we note $|\alpha| = \sum_{i=1}^d \alpha_i$ and 
$$
x^{\alpha} = x_1^{\alpha_1}x_2^{\alpha_2}\cdots x_d^{\alpha_d}.
$$
\begin{lemma}\label{lemm:dsigma}
	Let $\sigma \in C^{\infty}(\Omega)$ (i.e. $\sigma$ is infinitely differentiable) and is not a polynomial, then for any $k \ge 0$ there exists $t_k \in \mathbb{R}$ such that 
	$$
	\sigma^{(k)}(t_k) \neq 0.
	$$
\end{lemma}

Now we are going to give the proof of 
\begin{proof}
	If $\sigma$ is a polynomial, say $\sigma \in \mathbb{P}_m(\mathbb{R})$, then we have that
	$$
	{\rm DNN}_1 \subset \mathbb{P}_m(\mathbb{R}^d).
	$$
	Thus, ${\rm DNN}_1$ cannot approximate polynomial of degree bigger than $m+1$. This implies that $\sigma$ cannot be polynomial if ${\rm DNN}_1$ has the approximation property.
	
	Now we prove that $\overline{\rm DNN}_1 = C(\bar \Omega)$ if $\sigma$ is not a polynomials.
	
	\begin{description}
		\item[Case 1] First, let us assume that $\sigma \in C^{\infty}(\Omega)$. 
		
		\item[Fact 1:] We have the next relation:
		$$
		\frac{\partial}{\partial [w]_i} \left(\sigma(wx + b) \right) |_{w=0}= \sigma'(wx+b)\frac{\partial}{\partial [w]_i} (wx+b) = \sigma'(wx+b)x_i |_{w=0}.
		$$
		That is to say,
		$$
		\frac{\partial}{\partial [w]_i} \left(\sigma(wx + b) \right) |_{w=0} = \sigma'(b)x_i.
		$$
		By the lemma \ref{lemm:dsigma}, there exists a $b \in \mathbb{R}$ such that 
		$$
		\sigma'(b) \neq 0.
		$$
		\item[Fact 2:] By using the definition of derivative, we have
		$$
		\frac{\partial}{\partial [w]_i} \left(\sigma(wx + b) \right) |_{w=0}=
		\lim_{n\to \infty} \frac{\sigma((0 + \frac{1}{n}e_i )\cdot x + b) - \sigma(b)}{\frac{1}{n}} = \lim_{n\to\infty} \phi_n(x),
		$$
		where 
		$$
		\phi_n(x) = n \left(\sigma(\frac{1}{n}e_i \cdot x + b) - \sigma(b)\right) \in {\rm DNN}_1.
		$$
		This leads to the result that
		$$
		\sigma'(b)x_i = \lim_{n\to\infty} \phi_n(x) \in \overline{\rm DNN}_1,
		$$
		because of the definition that
		$$
		\overline{\rm DNN}_1 = {\rm DNN}_1 \cup \{f ~:~f = \lim_{n\to\infty} \phi_n(x), \phi_n \in {\rm DNN}_1 \}. 
		$$
	\end{description}
	Follow there facts, we know that
	$$
	\sigma'(b)x_i \in {\rm DNN}_1,
	$$
	this leads to 
	$$
	x_i \in 	\overline{\rm DNN}_1,
	$$
	because  $[\sigma'(b)]^{-1} \phi_n(x) \to x_i$.
	
	Thus we have
	$$
	\frac{\partial^2}{\partial [w]_1 \partial [w]_2} \left(\sigma(wx+b)\right) |_{w=0} = \sigma^{(2)}(b)x_1x_2.
	$$
	Using the Lemma \ref{lemm:dsigma} again, there exists a $b \in \mathbb{R}$ such that
	$$
	\sigma^{(2)}(b) \neq 0.
	$$
	This leads to 
	$$
	x_1x_2 \in \overline{\rm DNN}_1 .
	$$
	Similarly, we can prove that
	$$
	x_1^{\alpha_1} x_2^{\alpha_2} \cdots x_d^{\alpha_d} \in \overline{\rm DNN}_1.
	$$
	This proves that ${\rm DNN}_1$ can approximate any polynomials.
	Combine with the Weierstrass theorem, ${\rm DNN}_1$ can approximate any continuous functions.
	
	Then we will finish the proof for any $\sigma$ as a non-polynomial Riemann integrable function and $\sigma\in L_{loc}^\infty(\mathbb{R})$.
\end{proof}

\iffalse
Given an activation function
\begin{equation}
  \label{activation}
\sigma: \mathbb R^1\mapsto \mathbb R^1  
\end{equation}
We consider the following shallow neural network function class
\begin{equation}
  \label{ShallowNN}
	\Sigma_d(\sigma)=\mathrm{span}\left\{\sigma(\omega\cdot x+\theta):\omega\in\mathbb{R}^d,\theta\in\mathbb{R}\right\}.
\end{equation}

In this chapter, we will two theorems as follows. 
\begin{theorem}
Let $\sigma$ be a non-polynomial Riemann integrable function and
$\sigma\in L_{loc}^\infty(\mathbb{R})$. Then $\Sigma_d(\sigma)$ in dense in
$C(\mathbb R^d)$.	
\end{theorem}
\fi 

\begin{theorem}
Let $\Omega\subset \mathbb{R}^d$ be a bounded set, $\sigma\in W^{m,\infty}(\mathbb R)$ that has 
compact support such that 
\begin{equation}
  \label{eq:1}
  \hat\sigma(a)\neq 0, \mbox{ for some } a\neq 0.
\end{equation}
Then
\begin{equation}
  \label{eq:2}
\inf_{f_n\in\Sigma_n}\|f-f_n\|_{H^m( \Omega)}
\le C(d,m)\;n^{-{1\over2}}\int_{\mathbb R^d}(1+|\omega|)^{m+1}|\hat f(\omega)|
\end{equation}
\end{theorem}

\iffalse
	\begin{lemma}
		Let $\sigma\in C^\infty(\mathbb{R})$ and assume $\sigma$ is not a polynomial. Then $\Sigma_n(\sigma)$ is dense in $C(\mathbb{R}^n)$.
	\end{lemma}
	\begin{proof}
		Since $\sigma\in C^\infty(\mathbb{R})$, and $[\sigma((\omega+h e_j)\cdot x+\theta)-\sigma(\omega\cdot x+\theta)]/h\in\Sigma_n(\sigma)$ for every $\omega,\theta$ and $h\ne0$, it follows that 
		$\frac{\partial}{\partial \omega_j}\sigma(\omega\cdot x+\theta)\in\overline{\Sigma}_n(\sigma)$ for all $j=1:n$. By the same argument $\frac{\partial^k}{\partial \omega^k_j}\sigma(\omega\cdot x+\theta)\in\overline{\Sigma}_n(\sigma)$ for all $k\in\mathbb{N}$, $j=1:n$, $\omega\in\mathbb{R}^n$ and $\theta\in\mathbb{R}$.
		
		Now $\frac{\partial^k}{\partial \omega^k_j}\sigma(\omega\cdot x+\theta)=x_j^k\sigma^{(k)}(\omega\cdot x+\theta)$, and since $\sigma$ is not a polynomial there exists a $\theta_k\in\mathbb{R}$ such that $\sigma^{(k)}(\theta_k)\ne0$. Take $\omega=0$ and $\theta=\theta_k$, we then have $x_j^k\in\overline{\Sigma}_n(\sigma)$. Similarly, for all polynomials of the form $x_1^{k_1}\cdots x_n^{k_n}$, we can get them by taking the corresponding partial derivatives.
		
		This implies that $\overline{\Sigma}_n(\sigma)$ contains all polynomials. By Weierstrass's Theorem it follows that $\overline{\Sigma}_n(\sigma)$ contains $C(K)$ for each compact $K\subset\mathbb{R}^n$. That is $\Sigma_n(\sigma)$ is dense in $C(\mathbb{R}^n)$.
		\end{proof}
	
	\fi

The first proof for this lemma above can be found in \cite{leshno1993multilayer} and summarized in \cite{pinkus1999approximation}.
The next theorem plays an important role in the proof of above lemma, which is first proved
in \cite{leshno1993multilayer} with several steps. Here we can present a more direct and simple version.
%
%\begin{proposition}
%	If $\Sigma_1$ is dense in $C(\mathbb{R})$, then $\Sigma_n$ is dense in $C(\mathbb{R}^n)$.
%	\end{proposition}
%
%
%
%From now on, we will focus on $\mathbb{R}$.

\begin{theorem}
	\label{prop:conti}
Let $\sigma$ be a non-polynomial Riemann integrable function and $\sigma\in L_{loc}^\infty(\mathbb{R})$. Then $\Sigma_1(\sigma)$ in dense in $C(\mathbb{R})$.
\end{theorem}
\begin{proof}
	Consider the mollifier $\eta$
	\begin{equation*}\eta(x)=\left\{
	\begin{aligned}
	Cexp(\frac{1}{|x|^2-1}),\quad&|x|<1,\\
 	0,\qquad\qquad\qquad\quad  &|x|\ge1.
	\end{aligned}
	\right.
	\end{equation*}
	here C is selected so that $\int_{\mathbb{R}}\eta dx=1.$
	
	Set $\eta_\epsilon=\frac{1}{\epsilon}\eta(\frac{x}{\epsilon})$. Then consider $\sigma_{\eta_\epsilon}$
\begin{equation}
	\sigma_{\eta_\epsilon}(x):=\sigma\ast{\eta_\epsilon}(x)=\int_{\mathbb{R}}\sigma(x-y){\eta_\epsilon}(y)dy
\end{equation}

It can be seen that $\sigma_{\eta_\epsilon}\in
C^\infty(\mathbb{R})$. Following the proof in the previous
proposition, we want to show that $\overline{\Sigma}_1(\sigma)$
contains all polynomials.  The first step is to show that
$\overline{\Sigma}_1(\sigma_{\eta_\epsilon})\subset\overline{\Sigma}_1(\sigma)$,
which can be done easily by checking the Riemann sum of
$\sigma_{\eta_\epsilon}(x)=\int_{\mathbb{R}}\sigma(x-y){\eta_\epsilon}(y)dy$
is in $\overline{\Sigma}_1(\sigma)$.

Then it suffices to show that there exists $\theta_k$ and
$\sigma_{\eta_\epsilon}$ such that
$\sigma_{\eta_\epsilon}^{(k)}(\theta_k)\ne0$ for each k. If not, then
there must be $k_0$ such that
$\sigma_{\eta_\epsilon}^{(k_0)}(\theta)=0$ for all
$\theta\in\mathbb{R}$ and all $\epsilon>0$.  Thus
$\sigma_{\eta_\epsilon}$'s are all polynomials with degree at most
$k_0-1$.  In particular, It is known that $\eta_\epsilon\in
C_0^\infty(\mathbb{R})$ and $\sigma\ast\eta_\epsilon$ uniformly
converges to $\sigma$ on compact sets in $\mathbb{R}$ and
$\sigma\ast\eta_\epsilon$'s are all polynomials of degree at most
$k_0-1$. Polynomials of a fixed degree form a closed linear subspace,
therefore $\sigma$ is also a polynomial of degree at most $k_0-1$,
which leads to contradiction.
\end{proof}



\section{Semi-constructive Method}
A interesting problem is that how big this special function set $\mathcal{F}$ is? In 1989, there are many papers using some different methods to prove a surprising results that with some simple assumption for active function $g_k$, for K=1, then $\mathcal{F}$ is uniformly dense on compact in $C^0(R^{m})$. To state and proof those result simpler, we have the next definition:
\begin{definition}\label{def:space}
For $\mathcal{F}$ with $K=1$, we note it by the next function set:
\begin{equation}
\mathcal{F} = \Sigma(g) := \{ \sum_{j=1}^{n_1} \beta_j g (\omega_{j}\cdot x + \theta_{j}) ~ | ~  n_1 \in \mathbb{N}^{+}, \beta_i, \theta_j, \in \mathbb{R}, \bm \omega_j \in \mathbb{R}^m \}.
\end{equation}
\end{definition}

\begin{definition}\label{def:algebra-space}
We define the smallest function algebra which contains $\Sigma(g)$ as a subset:
\begin{equation}
\Pi \Sigma(g) := \{ \sum_{j}^{n_1} \beta_j \cdot \prod_{k}^{l_j}G({\omega}_{j}^k \cdot {x} + \theta_j^k) ~|~ {\omega}_{j}^{k} \in \mathbb{R}^m, \beta_j \in \mathbb{R}, l_j \in \mathbb{N}, n_1 \in \mathbb{N}^{+} \},
\end{equation}
\end{definition}

\begin{definition}\label{def:udoc}
A subset $S \subset C^0(\mathbb{R}^m)$ is said to be {\bf uniformly dense on compact in $C^0(\mathbb{R}^m)$ }if for any compact subset $K \subset \mathbb{R}^m$, then $S$ is dense in $C^0(K)$ when restrict $S$ to $K$.
\end{definition}

\begin{definition}\label{def:squashing}
A function $g: \mathbb{R} \to [0,1]$ is {\bf squashing} if it is non-decreasing, $\lim_{x\to \infty}g(x) = 1$ and $\lim_{x\to -\infty} g(x) = 0$.
\end{definition}

\begin{definition}\label{def:cosine-squasher}
We define the next cosine squasher function as: 
\begin{equation}
\phi(x) = 
\begin{cases}
0 & -\infty < x \le -\frac{\pi}{2} \\
\frac{cos(x+\frac{3\pi}{2}) + 1}{2} \quad &-\frac{\pi}{2} < x < \frac{\pi}{2} \\
1 &\frac{\pi}{2} \le x < \infty. 
\end{cases}
\end{equation}
\end{definition}

To prove the next theorem by Hornik et. 1989 we need the Stone-Weierstrass theorem:
\begin{theorem}[Stone-Weierstrass]
Let $\mathcal{A}$ be an algebra of real continuous functions on a compact set $\Omega$, which means that $\mathcal{A} \subset C^0(\Omega)$ and $af(x) + bg(x) \in \mathcal{A}$, $fg \in \mathcal{A}$ for any $a, b \in \mathbb{R}$ and $f, g \in \mathcal{A}$. If $\mathcal{A}$ separates points on $\Omega$ and if $\mathcal{A}$ vanishes at no point of $\Omega$, i.e:
\begin{itemize}
\item for every $x \neq y \in \Omega$, then exists $f \in \mathcal{A}$ s.t $f(x) \neq f(y)$,
\item for every $x \in \Omega$, then exists $f \in \mathcal{A}$ s.t $f(x) \neq 0$.
\end{itemize}
Then the uniform closure of $\mathcal{A}$ w.r.t $L^{\infty}$ norm is equal to $C^{0}(\Omega)$.
\end{theorem}
 
 And we introduce a important lemma which might be the most important results to construct the dense result for $\Sigma(g)$ in Hornik et, 1989. 
 \begin{lemma}\label{lemm:squashing}
 Let $f$ be a continuous squashing function and $g$ an arbitrary squashing function. Then for any $\epsilon > 0$, there is an function $\hat{f} \in \Sigma(g)$ such that 
 $$ \|f - \hat{f} \|_{C^0(\mathbb{R})} < \epsilon .$$
 \end{lemma}
 \begin{proof}
 This is an interesting exercise. Remark: here $m = 1$ for $\Sigma(g)$.
 \end{proof}

Here is the result:
\begin{theorem}[Hornik 1989]
For any squashing function $g$, $\Sigma(g)$ is uniformly dense on compact in $C^0(\mathbb{R}^m)$.
%, if we just take the output layer active function $g_{2}(x) = x$, then there exist $n_1$ and $W_2$ such that:
%\begin{equation}
%\|\bm{f}_2(W_2; \bm{x}) - f(\bm{x})\|_{C^{0}(I^m)} \le \epsilon
%\end{equation}
%where $\bm{f}_2(W_2; \bm{x})$ is defined as
%\begin{equation}
%\bm{f}_2(W_2; \bm{x}) = \sum_{l=1}^{n_1} \omega_{2,l1}\sigma(\sum_{i=1}^m \omega_{1,li}x_i + \theta_{1,l}). 
%= \bm{W}_{2}^{T} \sigma(\bm{W}_{1}\bm{x} + \bm{\theta}_1).
%\end{equation}
\end{theorem}

\begin{proof}
Without loss of generality, we consider a compact set $K$. And choose any $f \in C^{0}(\mathbb{R}^m)$ and any $\epsilon > 0$. We use the next steps to prove the theory.
\begin{itemize}
\item We can use the Stone-Weierstrass theorem which shows that $\Sigma(cos) = \Pi \Sigma(cos)$ is dense in $C^0(K)$. Which means we have $f_1 \in \Sigma(cos)$ such that 
$$
\|f - f_1\|_{C^{0}(K)} \le \frac{\epsilon}{2},
$$
with 
$$f_1 = \sum_{i=1}^P \hat \beta_i cos(\hat\omega_i \cdot x + \hat\theta_i).$$
Now we take $M \in \mathbb{N}^{+}$ such that 
$$
(\hat\omega_i \cdot x + \hat\theta_i) |_K \subset [-2\pi M, 2\pi M] \quad \forall 1\le i \le P.
$$
\item We construct a new function $\hat{\phi}$ by M and $\phi$ defined in \ref{def:cosine-squasher}:
\begin{align}
\hat{\phi}(x) = \sum_{k = -M}^{M} 2(\phi(-x + \frac{\phi}{2} - 2m\phi) + \phi(x - \frac{3\phi}{2} + 2m\phi) - 1) \\=  
\begin{cases}
0 & -\infty < x \le -2\pi M \\
cos(x) - 1 \quad &-2\pi M < x < 2\pi (M+1) \\
0 &  2\pi(M+1)\le x < \infty. 
\end{cases}
\end{align}
So we have 
\begin{equation}
f_1  = \sum_{i=1}^P \hat\beta_i \hat{\phi}(\hat\omega_i \cdot x + \hat\theta_i) + (\sum_{i=1}^P \beta_i )\hat{\phi}(0\cdot x + \frac{\pi}{2}) = \sum_{i=1}^Q\beta_i \phi(\omega_i \cdot x +  \theta_i),
\end{equation}
when restricted into $K$.



\item Then we use the Lemma \ref{lemm:squashing}, we can approximate $\phi$ by using:
$$
\tilde \phi(x) = \sum_{i=1}^{L} \alpha_ig(a_ix + \gamma_i) \quad a_i, \gamma_i \in \mathbb{R},
$$
such that
$$
\|\phi - \tilde \phi \|_{C^{0}(\mathbb{R})} \le \frac{\epsilon}{ 2 \sum_{i=1}^{Q} |\beta_i|}.
$$
So we let
$$
f_2 = \sum_{i=1}^Q \beta_i \tilde \phi(\omega_i \cdot x + \theta_i) \in \Sigma(g),
$$
when restricted to $K$, we have
$$
\|f_1 - f_2\|_{C^0(K)} \le \sum_{i=1}^{Q} |\beta_i| \| \phi - \tilde \phi \|_{C^0(K)} \le \frac{\epsilon}{2}.
$$
\end{itemize}

This finish this proof.
\end{proof}

However, this approximation property cannot tell us anything about how to set the net structure to get better approximation.


\section{Functional Analysis Methods}
\paragraph{Hahn-Banach Theorem: analytic form}
Let $X$ be a normed space over field $K$ of real or complex numbers. Let $f:V\rightarrow K$ be a bounded linear functional defined on a subspace $V\subset X$. Then $f$ can be extended to a bounded linear functional $F:X \rightarrow K$ having the same norm: $norm(F):=\mathop{sup} \limits_{x\in X,norm(x)\le 1}|F(x)|=\mathop{sup} \limits_{x\in V,norm(x)\le 1}|f(x)|=:norm(f)$

\paragraph{Hahn-Banach Theorem: geometric form}

Let $X$ be a normed space over the reals, and let $A,B$ be nonempty, disjoint, convex subsets of $X$:

(i) If $A$ is open, then there exists a bounded linear functional $\phi: X\rightarrow R$ and a number $c\in R$ such that: $\phi(a)<c\le\phi(b)$ for all $a\in A, b\in B$

(ii) If $A$ is compact and $B$ is closed, then there exists a bounded linear functional $\phi: X\rightarrow R$ and numbers $c_1, \ c_2$ such that: $\phi(a)\le c_1<c_2\le \phi(b)$ for all $a\in A, b\in B$

\paragraph{Riesz-Markov Theorem (Representation theorem for the continuous dual of $C_0(X)$)}

Let $X$ be locally compact Hausdorff space. For any continuous linear functional $\phi$ on $C_0(X)$, there is a unique regular countably additive complex Borel measure $\mu$ on $X$ such that $\phi(f)=\int_Xf(x)d\mu(x)$ for all $f\in C_0(X)$.

\paragraph{Some definitions}

Let $I_n$ denote the n-dimensional unit cube, $[0,1]^n$. The space of continuous functions on $I_n$ is denoted by $C(I_n)$ and we use $norm(f)$ to denote the supremum (or uniform) norm of an $f\in C(I_n)$. In general we use $norm(\cdot)$ to denote the maximum of a function on its domain. The space of finite, signed regular Borel measures on $I_n$ is denoted by $M(I_n)$.


Def: We say that $\sigma$ is discriminatory if for a measure $\mu\in M(I_n)$
$$\int_{I_n}\sigma(y^Tx+\theta)d\mu(x)=0$$
for all $y\in R^n$ and $\theta\in R$ implies that $\mu=0$.


Def: We say that $\sigma$ is sigmoidal if
$\sigma(t)\rightarrow \Big\{
\begin{aligned}
1  \ \ \ as \ \ \ t\rightarrow + \infty \\
0  \ \ \ as \ \ \ t\rightarrow - \infty
\end{aligned}$

\paragraph{Theorem 1}

Let $\sigma$ be any continuous discriminatory function. Then finite sums of the form:
$$G(x)=\mathop{\Sigma}\limits_{j=1}^N\alpha_j\sigma(y_j^Tx+\theta_j)$$
are dense in $C(I_n)$. In other words, given any $f\in C(I_n)$ and $\epsilon>0$, there is a sum, $G(x)$, of the above form, for which $|G(x)-f(x)|<\epsilon$ for all $x\in I_n$.
~\\~\\proof: Let $S\subset C(I_n)$ be the set of functions of the form $G(x)$ as in the statement of the theorem 1. Clearly $S$ is a linear subspace of $C(I_n)$. We claim that the closure of $S$ is all of $C(I_n)$.

Assume that the closure of $S$, say $R(=\bar{S})$, is a closed proper subspace of $C(I_n)$ [Notice: the closure of a linear space is also a linear space: $r_1\in R,r_2\in R, r_1=\mathop{lim}\limits_{n\rightarrow \infty}S_n^1, r_2=\mathop{lim}\limits_{n\rightarrow \infty}S_n^2 (S_n^1\in S, S_n^2\in S)$. So, $ar_1+br_2=\mathop{lim}\limits_{n\rightarrow \infty}aS_n^1+bS_n^2$. Because $S$ is a linear space, so $aS_n^1+bS_n^2\in S$. Then $ar_1+br_2\in R$. So $R$ is a linear space.]


Claim: there is a bounded linear functional on $C(I_n)$, call it $L$, with the property that $L\neq 0$ but $L(R)=L(S)=0$.


proof of the claim: Because $R\subsetneqq C(I_n)$, so there exists an $x_0\in C(I_n) \setminus R$. We denote $\tilde{R}=span(R,x_0)$. For $\forall \tilde{r}\in \tilde{R}, \exists! \ r\in R, t\in \mathbb{R}$ s.t. $\tilde{r}=r+tx_0$\ (if $\tilde{r}=r+tx_0=r'+t'x_0$, then $r-r'=(t-t')x_0$. So $(t-t')x_0\in R$. Then we get $t=t', r=r'$)
~\\ We define a funcitonal on $\tilde{R}$, say $\tilde{L}$.
$$\tilde{L}: \tilde{R} \rightarrow \mathbb{R}$$
$$\tilde{r}=r+tx_0\mapsto t$$
That is $\tilde{L}(\tilde{r})=\tilde{L}(r+tx_0)=t$.
~\\It's easy to show $\tilde{L}$ is linear. Now we prove that $\tilde{L}$ is bounded.
~\\$norm(\tilde{L})=\mathop{sup}\limits_{\tilde{r}\neq0}\frac{|L(\tilde{r})|}{norm(\tilde{r})}=\mathop{sup}\limits_{r+tx_0\neq0}\frac{|t|}{norm(r+tx_0)}=\mathop{sup}\limits_{r/t+x_0\neq0}\frac{|t|}{|t|\cdot norm(r/t+x_0)}=\mathop{sup}\limits_{r/t+x_0\neq0}\frac{1}{norm(r/t+x_0)}=\frac{1}{\mathop{inf}\limits_{r/t\in R}norm(r/t+x_0)}=\frac{1}{\mathop{inf}\limits_{v\in R}norm(x_0-v)}$
~\\Because $R,x_0$ are nonempty, disjoint, and convex subsets of $C(I_n)$, and $R$ is closed, $x_0$ is compact.
~\\By Hahn-Banach theorem(geometric form), there exists a bounded linear functional $\phi: C(I_n)\rightarrow \mathbb{R}$ and $c_1, c_2\in \mathbb{R}$ s.t. $\phi(x_0)\le c_1<c_2\le \phi(r)\ \ \forall r\in R$
~\\So $\phi(r)-\phi(x_0)=\phi(r-x_0)\ge c_2-c_1>0$, then we have $c_2-c_1\le\phi(r-x_0)\le norm(\phi)\cdot norm(r-x_0)\Rightarrow norm(r-x_0)\ge\frac{c_2-c_1}{norm(\phi)}>0\ \ \forall r\in R$.
~\\Then, $\mathop{inf}\limits_{v\in R}norm(x_0-v)\ge\frac{c_2-c_1}{norm(\phi)}>0$, so $norm(\tilde{L})<\infty$
~\\Now, we've proved that $\tilde{L}$ is a bounded linear functional on $\tilde{R}$, and $\tilde{L}\neq 0$, and $\tilde{L}(r)=\tilde{L}(r+0\cdot x_0)=0\ \ \forall r\in R$. So $\tilde{L}(R)=\tilde{L}(S)=0$.
~\\if $\tilde{R}=C(I_n)$, then we define $L:=\tilde{L}$, then we're done.
~\\if $\tilde{R}\subsetneqq C(I_n)$. By Hahn-Banach Theorem(analytic form), $\tilde{L}$ can be extended to a bounded linear functional $L: C(I_n)\rightarrow \mathbb{R}$ and $L|_{\tilde{R}}=\tilde{L}$. So $L\neq0$, and $L(R)=\tilde{L}(R)=0$. Then we have $L(R)=L(S)=0$ and $L\neq0$.


So we have proved the claim that there is a bounded linear functional on $C(I_n)$, call it $L$, with the property that $L\neq 0$ but $L(R)=L(S)=0$.

~\\Because $I_n$ is locally compact (actually, it is compact) and Hausdorff space. So $C_0(I_n)=C(I_n)$, and $L$ is a continuous linear functional on $C(I_n)$. By Riesz-Markov Theorem, there is a unique regular countably additive signed Borel measure $\mu$ on $I_n$ such that $L(h)=\int_{I_n}h(x)d\mu(x)$ for all $h\in C(I_n)$
~\\In particular, $\sigma(y^Tx+\theta)\in R$ (by definition), so $\int_{I_n}\sigma(y^Tx+\theta)d\mu(x)=L(\sigma(y^Tx+\theta))=0$ for all $y$ and $\theta$

~\\However, we assumed $\sigma$ wa discriminatory so that $\mu=0$. But $L\neq0$. So $\exists f\in C(I_n)$ such that $L(f)=\int_{I_n}fd\mu\neq0$, so $\mu\neq0$. We get the contradiction.
~\\Hence, $R=\bar{S}=C(I_n)$, that is $S$ is dense in $C(I_n)$.
~\\ Q.E.D.

~\\~\\Now, we specialize this result to show that any continuous sigmoidal $\sigma$ of the form discussed before, namely
$\sigma(t)\rightarrow \Big\{
\begin{aligned}
1  \ \ \ as \ \ \ t\rightarrow + \infty \\
0  \ \ \ as \ \ \ t\rightarrow - \infty
\end{aligned}$
is discriminatory.
\paragraph{Lebesgue Bounded Convergence Theorem}

Let ${f_n}$ be a sequence of measurable functions on a set of finite measure $E$. Suppose ${f_n}$ is uniformly pointwise bounded on $E$, that is, there is a number $M\ge0$ for which $|f_n|\le M$ on $E$ for all $n$.

If $f_n\rightarrow f$ pointwise on $E$, then $\mathop{lim}\limits_{n\rightarrow\infty}\int_Ef_n=\int_Ef$.
\paragraph{Lemma1}

Any bounded, measurable sigmoidal function, $\sigma$, is discriminatory. In particular, any continuous sigmoidal function is discriminatory.
~\\~\\proof:To prove this, we assume $\int_{I_n}\sigma(y^Tx+\theta)d\mu(x)=0$ for $\forall y,\theta$, we want to prove $\mu=0$. To demonstrate this, note that for any $x,y,\theta,\phi$, we have
$$\sigma(\lambda(y^Tx+\theta)+\phi) \Big\{
\begin{aligned}
\rightarrow1 \ \ \ for \ \ \ y^Tx+\theta>0 \ \ \ as \ \ \ \lambda\rightarrow + \infty \\
\rightarrow0 \ \ \ for \ \ \ y^Tx+\theta<0  \ \ \ as \ \ \ \lambda\rightarrow + \infty \\
=\sigma(\phi) \ \ \ for \ \ \ y^Tx+\theta=0 \ \ \ for \ \ \ all \ \ \lambda
\end{aligned}$$
Thus, the functions $\sigma_\lambda(x)=\sigma(\lambda(y^Tx+\theta)+\phi)$ converge pointwise to the function
$$\gamma(x) =\Big\{
\begin{aligned}
1 \ \ \ for \ \ \ y^Tx+\theta>0 \\
0 \ \ \ for \ \ \ y^Tx+\theta<0 \\
\sigma(\phi) \ \ \ for \ \ \ y^Tx+\theta=0
\end{aligned}$$
as $\lambda\rightarrow + \infty$
~\\Furthermore, $\sigma_\lambda(x)=\sigma(\lambda(y^Tx+\theta)+\phi)$ is uniformly bounded. (Because we've assumed that $\sigma$ is bounded).
~\\Let $\Pi_{y,\theta}$ be the hyperplane defined by $\{x|y^Tx+\theta=0, x\in I_n\}$ and let $H_{y,\theta}$ be the open half-space defined by $\{x|y^Tx+\theta>0, x\in I_n\}$. By the Lebesgue Bounded Convergence Theorem, we have:
$$\mathop{lim}\limits_{\lambda\rightarrow+ \infty}\int_{I_n}\sigma_{\lambda}(x)d\mu(x)=\int_{I_n}\gamma(x)d\mu(x)$$
Because $\int_{I_n}\sigma_{\lambda}(x)d\mu(x)=0$ for $\forall \lambda$ by assumption, so $\int_{I_n}\gamma(x)d\mu(x)=\mathop{lim}\limits_{\lambda\rightarrow+ \infty}\int_{I_n}\sigma_{\lambda}(x)d\mu(x)=0$
~\\Then $0=\int_{I_n}\gamma(x)d\mu(x)=\int_{\Pi_{y,\theta}}\gamma(x)d\mu(x)+\int_{H_{y,\theta}}\gamma(x)d\mu(x)+\int_{I_n\setminus (\Pi_{y,\theta}\cup H_{y,\theta})}\gamma(x)d\mu(x)$
~\\Because $$\gamma(x) =\Big\{
\begin{aligned}
1 \ \ \ \ \ \ \ \ \ \ \ \ \ \ \ \ \ \ x\in H_{y,\theta} \\
0 \ \ \ x\in I_n\setminus H_{y,\theta}\cup \Pi_{y,\theta} \\
\sigma(\phi) \ \ \ \ \ \ \ \ \ \ \ \ \ \ x\in \Pi_{y,\theta}
\end{aligned}$$
So, $0=\int_{\Pi_{y,\theta}}\gamma(x)d\mu(x)+\int_{H_{y,\theta}}\gamma(x)d\mu(x)=\sigma(\phi)\cdot\mu(\Pi_{y,\theta})+\mu(H_{y,\theta})$
~\\We now show that the measure of all half-planes being 0 implies that the measure $\mu$ itself must be 0.
~\\Fix y, for a bounded measurable function, h, define the linear functional, F, according to :
$$F(h)=\int_{I_n}h(y^Tx)d\mu(x)$$
$$|F(h)|\le\int_{I_n}|h(y^Tx)|d|\mu|(x)\le norm(h)_{L^{\infty}(\mathbb{R})}\cdot|\mu|(I_n)$$
Because $\mu$ is finite signed measure, so $|\mu|(I_n)<\infty$.
Then $norm(F)=\mathop{sup}\limits_{h\neq0}\frac{|F(h)|}{norm(h}_{L^{\infty}(\mathbb{R})})\le|\mu|(I_n)<\infty$, so $F$ is a bounded functional on $L^{\infty}(\mathbb{R})$.
~\\Let $h$ be the indicator function of the interval $[\theta, \infty)$ (That is, $h(u)=1$ if $u\ge\theta$ and $h(u)=0$ if $u<\theta$). So that
$$F(h)=\int_{I_n}h(y^Tx)d\mu(x)=\int_{\Pi_{y,-\theta}}h(y^Tx)d\mu(x)+\int_{H_{y,-\theta}}h(y^Tx)d\mu(x)+\int_{I_n\setminus (\Pi_{y,-\theta}\cup H_{y,-\theta})}h(y^Tx)d\mu(x)$$
(Recall that $\Pi_{y,\theta}=\{x|y^Tx+\theta=0, x\in I_n\}$ and $H_{y,\theta}=\{x|y^Tx+\theta>0, x\in I_n\}$)
~\\So $F(h)=\mu(\Pi_{y,-\theta})+\mu(H_{y,-\theta})$
~\\Because we've shown that $\sigma(\phi)\mu(\Pi_{y,\theta})+\mu(H_{y,\theta})=0$ for all $\phi, \theta, y$ and $\sigma$ is sigmoidal function. So we take $\phi\rightarrow +\infty$, then $\sigma(\phi)\rightarrow1$. So $\mu(\Pi_{y,\theta})+\mu(H_{y,\theta})=0$ for all $y, \theta$. So
$$F(h)=\mu(\Pi_{y,-\theta})+\mu(H_{y,-\theta})=0$$
If $h$ is the indicator function of the open interval $(\theta, \infty)$.
$$F(h)=\int_{I_n}h(y^Tx)d\mu(x)=\int_{\Pi_{y,-\theta}}h(y^Tx)d\mu(x)+\int_{H_{y,-\theta}}h(y^Tx)d\mu(x)+\int_{I_n\setminus (\Pi_{y,-\theta}\cup H_{y,-\theta})}h(y^Tx)d\mu(x)=\mu(H_{y,-\theta})$$
Similarly, because we have $\sigma(\phi)\mu(\Pi_{y,\theta})+\mu(H_{y,\theta})=0$ for all $\phi, \theta, y$ and $\sigma$ is sigmoidal function. We take $\phi\rightarrow-\infty$, so $\sigma(\phi)\rightarrow0$. Then $\mu(H_{y,\theta})=0$ for all $y,\theta$. So we have
$$F(h)=\mu(H_{y,-\theta})=0$$
By linearity, $F(h)=0$ for the indicator function of any open intervals. ($F(\chi_{(a,b)})=F(\chi_{(a,\infty)}-\chi_{[b,\infty)})=F(\chi_{(a,\infty)})-F(\chi_{[b,\infty)})=0-0=0$)
~\\So it is true for any step functions($\mathop{\Sigma}\limits_{n=1}^Na_n\chi_{c_n}, \ \{c_n\}$ are disjoint open intervals).
~\\Because $sin(z)$ and $cos(z)$ can be uniformly approximated by step functions on a closed bounded interval. So we have $\int_{I_n}sin(y^Tx)d\mu(x)=0$ and $\int_{I_n}cos(y^Tx)d\mu(x)=0$ for any $y$. Then we have:
$$\int_{I_n}cos(y^Tx)+i\cdot sin(y^Tx)d\mu(x)=\int_{I_n}exp(iy^Tx)d\mu(x)=0$$
for all $y$. Thus, the Fourier transform of $\mu$ is 0 and so $\mu$ must be zero as well (trigonometric polynomials are dense in $C(I_n)$(stone-weierstrass), we can approximate a simple function by step functions and approximate a step function by continuous functions. So $\mu(E)=0$ for any measurable set $E$). Hence, $\sigma$ is discriminatory.
~\\Q.E.D.

~\\~\\Now, we apply the previous results to the case of most interest in neural network theory. A straightforward combination of Theorem 1 and Lemma 1 shows that networks with one internal layer and an arbitrary continuous sigmoidal function can approximate continuous functions with arbitrary precision providing that no constraints are placed on the number of nodes or the size of the weights. This is Theorem 2 below.

\paragraph{Theorem 2}

Let $\sigma$ be any continuous sigmoidal function. Then finite sums of the form
$$G(x)=\mathop{\Sigma}\limits_{j=1}^N\alpha_j\sigma(y_j^Tx+\theta_j)$$
are dense in $C(I_n)$. In other words, given any $f\in C(I_n)$ and $\epsilon>0$, there is a sum, $G(x)$, of the above form, for which
$$|G(x)-f(x)|<\epsilon$$
for all $x\in I_n$

~\\proof: Combine Theorem 1 and Lemma 1.
~\\Q.E.D.



