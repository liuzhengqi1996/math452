\section{Fourier transform and Fourier series}
We make use of the theory of tempered distributions (see
\cite{strichartz2003guide} for an introduction)
and we begin by collecting some results of independent interest, which
will also be important later. 
\subsection{Fourier transform}
Before studying the Fourier transform, we first consider Schwartz space which is defined below.
\begin{definition} \label{def:schwarz}
The Schwartz space $\mathcal{S}\left(\mathbb{R}^{n}\right)$ is the topological vector space of functions $f: \mathbb{R}^{n} \rightarrow \mathbb{C}$ such that $f \in C^{\infty}\left(\mathbb{R}^{n}\right)$ and
$$
x^{\alpha} \partial^{\beta} f(x) \rightarrow 0 \quad \text { as }|x| \rightarrow \infty
$$
for every pair of multi-indices $\alpha, \beta \in \mathbb{N}_{0}^{n} .$ For $\alpha, \beta \in \mathbb{N}_{0}^{n}$ and $f \in \mathcal{S}\left(\mathbb{R}^{n}\right)$ let
(5.10)
$$
\|f\|_{\alpha, \beta}=\sup _{\mathbb{R}^{n}}\left|x^{\alpha} \partial^{\beta} f\right|
$$
A sequence of functions $\left\{f_{k}: k \in \mathbb{N}\right\}$ converges to a function $f$ in $\mathcal{S}\left(\mathbb{R}^{n}\right)$ if
$$
\left\|f_{n}-f\right\|_{\alpha, \beta} \rightarrow 0 \quad \text { as } k \rightarrow \infty
$$
for every $\alpha, \beta \in \mathbb{N}_{0}^{n}$.
\end{definition}
The Schwartz space consists of smooth functions whose derivatives and the function itself decay at infinity faster than any power. Schwartz functions are rapidly decreasing. When there is no ambiguity, we will write $\mathcal{S}\left(\mathbb{R}^{n}\right)$ as $\mathcal{S}$.
Roughly speaking, tempered distributions grow no faster than a polynomial at infinity.

\begin{definition}
A tempered distribution $T$ on $\mathbb{R}^{n}$ is a continuous linear functional $T: \mathcal{S}\left(\mathbb{R}^{n}\right) \rightarrow \mathbb{C} .$ The topological vector space of tempered distributions is denoted by $\mathcal{S}^{\prime}\left(\mathbb{R}^{n}\right)$ or $\mathcal{S}^{\prime} .$ If $\langle T, f\rangle$ denotes the value of $T \in \mathcal{S}^{\prime}$ acting on $f \in \mathcal{S}$
then a sequence $\left\{T_{k}\right\}$ converges to $T$ in $\mathcal{S}^{\prime}$, written $T_{k} \rightarrow T$, if
$$
\left\langle T_{k}, f\right\rangle \rightarrow\langle T, f\rangle
$$
for every $f \in \mathcal{S}$.
\end{definition}
Since $\mathcal{D} \subset \mathcal{S}$ is densely and continuously imbedded, we have $\mathcal{S}^{\prime} \subset \mathcal{D}^{\prime} .$ Moreover, a distribution $T \in \mathcal{D}^{\prime}$ extends uniquely to a tempered distribution $T \in \mathcal{S}^{\prime}$ if and only if it is continuous on $\mathcal{D}$ with respect to the topology on $\mathcal{S}$. Every function $f \in L_{\text {loc }}^{1}$ defines a regular distribution $T_{f} \in \mathcal{D}^{\prime}$ by
$$
\left\langle T_{f}, \phi\right\rangle=\int f \phi d x \quad \text { for all } \phi \in \mathcal{D}.
$$
If $|f| \leq p$ is bounded by some polynomial $p,$ then $T_{f}$ extends to a tempered distribution $T_{f} \in \mathcal{S}^{\prime}$, but this is not the case for functions $f$ that grow too rapidly at infinity.

The Schwartz space is a natural one to use for the Fourier transform. Differentiation and multiplication exchange roles under the Fourier transform and therefore so do the properties of smoothness and rapid decrease. As a result, the Fourier transform is an automorphism of the Schwartz space. By duality, the Fourier transform is also an automorphism of the space of tempered distributions.

\begin{definition}\label{def:fourier1}
The Fourier transform of a function $f \in \mathcal{S}\left(\mathbb{R}^{n}\right)$ is the function $\hat{f}: \mathbb{R}^{n} \rightarrow \mathbb{C}$ defined by 
$$
\hat{f}(\omega)= \int f(x) e^{-2 \pi i\omega \cdot x} d x.
$$
The inverse Fourier transform of $f$ is the function $\check{f}: \mathbb{R}^{n} \rightarrow \mathbb{C}$ defined by
$$
\check{f}(x)=\int f(\omega) e^{2 \pi i\omega \cdot x} d k.
$$
\end{definition}

\begin{definition}\label{def:fourier2}
The Fourier transform of a tempered distribution $f \in \mathcal{S}'$ is  defined by 
$$
\langle \hat{f}, \phi\rangle = \langle f, \hat \phi\rangle,\quad \forall \phi\in \mathcal{S}.
$$ 
\end{definition}

The support of a continuous function $f$ is the closure  of the set $\{x\in \mathbb{R}: f(x)\neq 0\}$.
\begin{properties}
The Fourier transform has the following properties
\begin{enumerate}
\item If $f\in \mathcal{S}'$ and the support of $\hat f$ is $\{0\}$, then $f$ is a polynomial.
\item If $f\in \mathcal{S}'$ and the support of $\hat f$ is a single point $\{a\}$, then $f(x)=e^{2\pi iax}P(x)$, where $P(x)$ is a polynomial.
\end{enumerate}
\end{properties}







\subsection{Poisson summation formula}
% Qingguo put the Poisson summation formula here in this file:  statement and sketch of proof

\begin{theorem}
Let $f \in L^{1}(\mathbb{R})$ and $f$ is continuous. Then we have for almost all $(x, \omega ) \in \mathbb{R} \times \hat{\mathbb{R}}$ that
$$
T \sum_{n \in \mathbb{Z}} f(x+n T) e^{-2 \pi i \omega (x+n T)}=\sum_{n \in \mathbb{Z}} \hat{f}\left(\omega +\frac{n}{T}\right) e^{2 \pi i n x / T}
$$
where both sides converge absolutely.

In addition,  let $\Lambda$ be the lattice in $\mathbb{R}^{d}$ consisting of points with integer coordinates. 
For a function $f$ in $L^{1}\left(\mathbb{R}^{d}\right)$ and $f$ is continuous, we have 
$$
\sum_{\omega  \in \Lambda} f(x+\omega )=\sum_{\nu \in \Lambda} \hat{f}(\omega ) e^{2 \pi i x \cdot \omega }.
$$
where both series converge absolutely and uniformly on $\Lambda$. 
\end{theorem} 

\begin{proof}
We just give a proof of a simple case that $f: \mathbb{R} \rightarrow \mathbb{C}$ is a Schwarz function (see Definition \ref{def:schwarz}).
Let:
$$
F(x)=\sum_{n \in \mathbb{Z}} f(x+n).
$$
Then $F(x)$ is 1-periodic (because of absolute convergence), and has Fourier coefficients:
$$
\begin{aligned}
\hat{F}_{\omega } &=\int_{0}^{1} \sum_{n \in \mathbb{Z}} f(x+n) e^{-2 \pi i \omega x} \mathrm{~d} x \\
&=\sum_{n \in \mathbb{Z}} \int_{0}^{1} f(x+n) e^{-2 \pi i \omega  x} \mathrm{~d} x \quad \text { because } f \text { is Schwarz, so convergence is uniform}\\
&=\sum_{n \in \mathbb{Z}} \int_{n}^{n+1} f(x) e^{-2 \pi i\omega  x} \mathrm{~d} x \\
&=\int_{\mathbb{R}} f(x) e^{-2 \pi i \omega  x} \mathrm{~d} x\\
&=\hat{f}(k)\\
\end{aligned}
$$
 where $\hat{f}$ is the Fourier transform of $f$.
 

Therefore by the definition of the Fourier series of $f:$
$$
F(x) =\sum_{\omega  \in \mathbb{Z}} \hat{f}(k) e^{2\pi i \omega x}.
$$
Choosing $x=0$ in this formula:
$$
\sum_{n \in \mathbb{Z}} f(n)=\sum_{\omega  \in \mathbb{Z}} \hat{f}(\omega )
$$
as required.
\end{proof}






\subsection{A special cut-off function}
Let us first state the following simple result that can be obtained by following a calculation given in Section 3 of \cite{johnson2015saddle}. 
\begin{lemma} Given $\alpha>1$, consider
 \begin{equation}\label{alpha-g}
  g(t) = \begin{cases} 
      e^{-(1-t^2)^{1 - \alpha}} & t\in (-1,1) \\
      0 & \text{otherwise}.
   \end{cases}
 \end{equation}
then there is a constant $c_\alpha$ such that
 \begin{equation}\label{eq_181}
  |\hat{g}(\omega )|\lesssim e^{-c_\alpha|\omega |^{1-\alpha^{-1}}},
 \end{equation}
\end{lemma}
\begin{proof}
Consider the asymptotic behavior of the Fourier transform
$$
F(\omega )=\int_{-\infty}^{\infty} g(t) e^{2\pi i \omega  t} dt=2 \operatorname{Re} \int_{0}^{1} e^{2\pi i \omega  t- (1-t^{2})^{1-\alpha}} dt
$$
for $|\operatorname{Re} \omega | \gg 1.$ (Without loss of generality, we can restrict ourselves to real $\omega  \geq 0$).  
With a change of variable $x=1-t$,
$$
F(\omega )=2 \operatorname{Re} \int_{0}^{1} e^{f(x)} dx
$$
with 
$
f(x)=2\pi i \omega  - 2\pi i \omega   x- (2x-x^2)^{1-\alpha}\approx \tilde f(x)+O\left(x^{2-\alpha}\right)
$
and 
$$
\tilde f(x) = 2\pi i \omega  - 2\pi i \omega    x - (2 x)^{1-\alpha}.
$$
The saddle point is the $x=x_0$ where $f'(x_0)=0$. Since
$
\tilde f'(x)=-2\pi i \omega  + (\alpha-1)2^{1-\alpha} x^{-\alpha},
$
$$
x_{0} \approx \tilde x_0=\left (2^{-\alpha} (\alpha-1) / i \omega \pi \right )^{1 / \alpha} \sim \omega ^{-1 / \alpha}.
$$
Therefore $\tilde f(\tilde x_{0}) \sim \omega ^{(\alpha-1) / \alpha}$ asymptotically. The second derivative is 
$$
\tilde f'' (\tilde x_{0} )=-2^{1-\alpha}  \alpha(\alpha-1) \tilde x_{0}^{-\alpha-1}=-i^{(\alpha+1) / \alpha} 2 A \omega ^{(\alpha+1)/\alpha},
$$
where
$$
A=2\alpha  (\alpha-1)^{-1/\alpha}\pi^{(\alpha+1)/\alpha}.
$$
Now,
\begin{equation}
\begin{split}
\tilde f(x)\approx &\tilde f(\tilde x_0) + {\tilde f''(\tilde x_0)\over 2} (x-\tilde x_0)^2
\\
=&2\pi i \omega  - (\alpha - 1)^{1\over \alpha}(i\omega \pi )^{\alpha -1\over \alpha}  - (\alpha - 1)^{1-\alpha\over \alpha} (i\omega \pi )^{\alpha -1\over \alpha}
\\
&-i^{(\alpha+1) / \alpha} A \omega ^{(\alpha+1)/\alpha}(x- 2^{-1}(\alpha - 1)^{-{1\over \alpha}}(i\omega \pi )^{-{1\over \alpha}} )^2.
\end{split}
\end{equation} 
Choose a contour $x=i^{-1 / \alpha}u$, in which case  
$$
\tilde f(x) \approx \tilde f(\tilde x_{0}) -i^{(\alpha-1) / \alpha} A \omega ^{(\alpha+1) / \alpha}\left(u-u_{0}\right)^{2},
$$
which is a path of descent so we can perform a Gaussian integral. 

Recall that the integral of 
\begin{equation}\label{gaussInt}
\int_{-\infty}^{\infty} e^{-a u^{2}} d u=\sqrt{\pi / a}
\end{equation}
as long as Re$a>0,$ which is true here. Note also that, in the limit as $\omega $ becomes large, the integrand becomes zero except close to $u=\sqrt{1 / 2 \omega },$ so we can neglect the rest of the contour and treat the integral over $u$ as going from $-\infty$ to $\infty$. (Thankfully, the width of the Gaussian $\Delta u \sim \omega ^{-3 / 4}$ goes to zero faster than the location of the maximum $u_{0} \sim \omega ^{-1 / 2},$ so we don't have to worry about the $u=0$ origin). Also note that the change of variables from $x$ to $u$ gives us the Jacobian factor for 
$$dx=i^{-1 / \alpha}d u.$$ 
Thus, when all is said and done, we obtain the exact asymptotic form of the Fourier integral for $\omega  \gg 1$: 
\begin{equation}
\begin{split}
F(\omega ) \approx &2 \operatorname{Re}\int_{0}^{1} e^{\tilde f(\tilde x_0) - i^{(\alpha-1) / \alpha} A \omega ^{(\alpha+1) / \alpha}\left(u-u_{0}\right)^{2}} dx
\\
=&2 \operatorname{Re} e^{\tilde f(\tilde x_0)} i^{-1 / \alpha} \int_{-\infty}^{\infty} e^{- i^{(\alpha-1) \over  \alpha} A \omega ^{(\alpha+1) / \alpha}\left(u-u_{0}\right)^{2}} du
\\
=&2 \operatorname{Re} e^{\tilde f(\tilde x_0)} \pi^{1/2}i^{-1 / \alpha}  i^{(1-\alpha) \over  2\alpha} A^{-1/2} \omega ^{-(\alpha+1) / 2\alpha}\qquad \text{ by \eqref{gaussInt}} 
\\
=&2 \operatorname{Re}\left[\sqrt{\frac{\pi}{(i \omega )^{(\alpha+1) / \alpha} A}} e^{\tilde f(\tilde x_0)}\right]
\\
\approx &2 \operatorname{Re}\left[\sqrt{\frac{\pi}{(i \omega )^{(\alpha+1) / \alpha} A}} e^{ 2\pi i \omega - 2\pi i \omega  \tilde x_{0}- \left[\left(2-\tilde x_{0}\right) \tilde x_{0}\right]^{1-\alpha}}\right]
\end{split}
\end{equation}  
with $x_{0}$ and $A$ given above.  Notice that $ \tilde x_0\sim \omega ^{-1 / \alpha}$. Thus,
$$
|F(\omega ) | \approx  e^{-c_\alpha|\omega |^{1-\alpha^{-1}}}.
$$
\end{proof}
