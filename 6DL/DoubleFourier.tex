\subsection{General activation functions}
Assume that $\sigma$ is a locally Riemann integrable nonzero function 
and $\sigma\in L^1(\mathbb{R})$ and thus the Fourier transform of
 $\sigma$ is well-defined and continuous. 
 Since $\sigma$ is non-zero and 
 \begin{equation}\label{key}
 \hat \sigma(\omega) = \int_{\mathbb{R}} \sigma(t)e^{-2\pi i\omega t}dt,
 \end{equation}
 this implies that $\hat{\sigma}(a)\neq 0$ for
 some $a\neq 0$. Via a change of variables $t = \omega\cdot x + b$ and $dt = db$,
  this means that for all $x$ and $\omega$, we have
 \begin{equation}
 \begin{aligned}
  0\neq \hat{\sigma}(a)&= \int_{\mathbb{R}}\sigma(\omega\cdot x+b)e^{-2\pi ia(\omega\cdot x+b)}db \\
 & = e^{-2\pi ia\omega \cdot x} \int_{\mathbb{R}}\sigma(\omega\cdot x+b)e^{-2\pi iab}db ,
 \end{aligned}
 \end{equation}
 and so
 \begin{equation}
  e^{2\pi ia\omega \cdot x} = \frac{1}{\hat{\sigma}(a)}\int_{\mathbb{R}}\sigma(\omega\cdot x+b)e^{-2\pi iab}db.
 \end{equation}
 Likewise, since the growth condition also implies that $\sigma^{(k)}\in L^1$, we can differentiate the above expression  under the integral with respect to $x$.

 This allows us to write the Fourier mode $e^{2\pi ia\omega \cdot x}$ as an integral of neuron output functions. We substitute this
 into the Fourier representation of $u$
 (note that the assumption we make implies that $\hat{u}\in L^1$ so this
 is rigorously justified for a.e. $x$) to get
 \begin{equation}\label{integral_representation}
 \begin{split}
  u(x) &= \int_{\mathbb{R}^d} e^{2\pi i\omega\cdot x}\hat{u}(\omega)d\omega = 
  \int_{\mathbb{R}^d}\int_\mathbb{R}\frac{1}{ \hat{\sigma}(a)}
  \sigma\left(a^{-1}{\omega}\cdot
    x+b\right)\hat{u}(\omega)e^{-2\pi iab}dbd\omega
\\
&=  \int_{\mathbb{R}^d}\int_\mathbb{R} k(x,\theta) dbd\omega 
\end{split}
 \end{equation}
where $\theta=(\omega, b)$ 
and   
$$
k(x,\theta)= \frac{1}{ \hat{\sigma}(a)}
  \sigma\left(a^{-1}{\omega}\cdot
    x+b\right)\hat{u}(\omega)e^{-2\pi iab}.
$$