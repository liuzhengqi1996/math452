\subsection{Fourier transform of polynomials}
We begin by noting that an activation
function $\sigma$, which satisfies a polynomial growth condition
$|\sigma(x)| \leq C(1 + |x|)^n$ for some constants $C$ and $n$, is a
tempered distribution. As a result, we make this assumption on our
activation functions in the following theorems. We briefly note that
this condition is sufficient, but not necessary (for instance an
integrable function need not satisfy a pointwise polynomial growth
bound) for $\sigma$ to be represent a tempered distribution.

 We begin by studying the convolution of $\sigma$ with a Gaussian mollifier. Let $\eta$ be a Gaussian mollifier
 \begin{equation}
  \eta(x) = \frac{1}{\sqrt{\pi}}e^{-x^2}.
 \end{equation}
Set $\eta_\epsilon=\frac{1}{\epsilon}\eta(\frac{x}{\epsilon})$. Then consider 
\begin{equation}
\label{sigma-epsilon}
\sigma_{\epsilon}(x):=\sigma\ast{\eta_\epsilon}(x)=\int_{\mathbb{R}}\sigma(x-y){\eta_\epsilon}(y)dy
\end{equation}
for a given activation function $\sigma$.
It is clear that $\sigma_{\epsilon}\in C^\infty(\mathbb{R})$. Moreover, by considering the Fourier transform (as a tempered
distribution) we see that
\begin{equation}\label{eq_278}
 \hat{\sigma}_{\epsilon} = \hat{\sigma}\hat{\eta}_{\epsilon} = \hat{\sigma}\eta_{\epsilon^{-1}}.
\end{equation} 


We begin by stating a lemma which characterizes the set of polynomials in terms of their
 Fourier transform.
\begin{lemma}\label{polynomial_lemma} Given a tempered distribution
  $\sigma$,  the following statements are equivalent:
\begin{enumerate}
\item $\sigma$ is a polynomial 
\item $\sigma_\epsilon$ given by \eqref{sigma-epsilon} is a polynomial for any
  $\epsilon>0$. 
\item $\text{\normalfont supp}(\hat{\sigma})\subset \{0\}$. 
\end{enumerate}
\end{lemma}
\begin{proof}
  We begin by proving that (3) and (1) are equivalent.  This follows
  from a characterization of distributions supported at a single point
  (see \cite{strichartz2003guide}, section 6.3). In particular, a
  distribution supported at $0$ must be a finite linear combination of
  Dirac masses and their derivatives.  In particular, if
  $\hat{\sigma}$ is supported at $0$, then
  \begin{equation}
   \hat{\sigma} = \displaystyle\sum_{i=1}^n a_i\delta^{(i)}.
  \end{equation}
  Taking the inverse Fourier transform and noting that the inverse
  Fourier transform of $\delta^{(i)}$ is $c_ix^i$, we see that
  $\sigma$ is a polynomial. This shows that (3) implies (1), for the
  converse we simply take the Fourier transform of a polynomial and
  note that it is a finite linear combination of Dirac masses and
  their derivatives.
  
  Finally, we prove the equivalence of (2) and (3). For this it
  suffices to show that $\hat{\sigma}$ is supported at $0$ iff
  $\hat{\sigma}_\epsilon$ is supported at $0$. This follows from
  equation \ref{eq_278} and the fact that $\eta_{\epsilon^{-1}}$ is
  nowhere vanishing.
\end{proof}

As an application of Lemma \ref{polynomial_lemma}, we give a
simple proof of the result in the next section.   
