\chapter{Kolmogorov–Arnold representation theorem}
\begin{verbatim}
https://en.wikipedia.org/wiki/Kolmogorov–Arnold_representation_theorem
\end{verbatim}
From Wikipedia, the free encyclopedia
In real analysis and approximation theory, the Kolmogorov–Arnold representation theorem (or superposition theorem) states that every multivariate continuous function can be represented as a superposition of continuous functions of two variables. It solved a more general form of Hilbert's thirteenth problem.[1][2]
The works of Andrey Kolmogorov and Vladimir Arnold established that if f is a multivariate continuous function, then f can be written as a finite composition of continuous functions of a single variable and the binary operation of addition.[3]
More specifically
$$
f(x)=\sum_{q=0}^{2n}\Phi_q\left(\sum_{p=1}^n\phi_{q,p}(x_p)\right)
$$
Constructive proofs, and even more specific constructions can be found in <ref>Jürgen Braun and Michael Griebel. "On a constructive proof of Kolmogorov’s superposition theorem", http://citeseerx.ist.psu.edu/viewdoc/download?doi=10.1.1.91.5436&rep=rep1&type=pdf</ref>

In a sense, they showed that the only true multivariate function is the sum, since every other function can be written using [[univariate]] functions and summing.<ref name="dia">[[Persi Diaconis]] and Mehrdad Shahshahani, ''On Linear Functions of Linear Combinations'' (1984) p. 180 ([http://www-stat.stanford.edu/~cgates/PERSI/papers/nonlin_func.pdf link])</ref>

==History==
The Kolmogorov–Arnold representation theorem is closely related to [[Hilbert's 13th problem]]. In his [[Paris]] lecture at the [[International Congress of Mathematicians]] in 1900, [[David Hilbert]] formulated [[Hilbert's problems|23 problems]] which in his opinion were important for the further development of mathematics.<ref>[[David Hilbert]], Mathematical problems, [[Bulletin of the American Mathematical Society]], '''8''' (1902), pp. 461–462.</ref> The 13th of these problems dealt with the solution of general equations of higher degrees. It is known that for algebraic equations of degree 4 the solution can be computed by formulae that only contain radicals and arithmetic operations. For higher orders, [[Galois theory]] shows us that the solutions of algebraic equations cannot be expressed in terms of basic algebraic operations. It follows from the so called [[Tschirnhaus transformation]] that the general algebraic equation <math> x^{n}+a_{n-1}x^{n-1}+\cdot \cdot \cdot +a_{0}=0</math>  can be translated to the form <math> y^{n}+b_{n-4}y^{n-4}+\cdot \cdot \cdot +b_{1}y+1=0</math>. The Tschirnhaus transformation is given by a formula containing only radicals and arithmetic operations and transforms. Therefore, the solution of an algebraic equation of degree <math>n</math>  can be represented as a superposition of functions of two variables if <math>n<7</math> and as a superposition of functions of <math>n-4</math> variables if <math>n\geq 7</math>. For <math>n=7</math>  the solution is a superposition of arithmetic operations, radicals, and the solution of the equation  <math>y^{7}+b_{3}y^{3}+b_{2}y^{2}+b_{1}y+1=0</math>. 

A further simplification with algebraic transformations seems to be impossible which led to Hilbert's conjecture that "A solution of the general equation of degree 7 cannot be represented as a superposition of continuous functions of two variables". This explains the relation of [[Hilbert's thirteenth problem]] to the representation of a higher-dimensional function as superposition of lower-dimensional functions. In this context, it has stimulated many studies in the theory of functions and other related problems by different authors.<ref>Jürgen Braun, On Kolmogorov's Superposition Theorem and Its Applications, SVH Verlag, 2010, 192 pp.</ref>

==Variants of the Kolmogorov–Arnold representation theorem==
A variant of Kolmogorov's theorem that reduces the number of
outer functions <math>\Phi _{q}</math> is due to George Lorentz.<ref>George Lorentz, ''Metric entropy, widths, and superpositions of functions'', The American Mathematical Monthly, 69 (1962), pp. 469–485.</ref> He showed in 1962  that the outer functions <math>\Phi_{q}</math> can be replaced by a single function <math>\Phi</math>. More precisely, Lorentz proved the existence of functions <math>\phi _{q,p}</math>, <math>q=0,1,\ldots,2n,</math> <math>p=1,\ldots,n,</math> such that 

:<math> f(\bold x) = \sum_{q=0}^{2n} \Phi\left(\sum_{p=1}^{n} \phi_{q,p}(x_{p})\right)</math>.

Sprecher <ref>David A. Sprecher, ''On the structure of continuous functions of several variables'', [[Transactions of the American Mathematical Society]], '''115''' (1965), pp. 340–355.</ref> replaced the inner functions <math>\phi_{q,p}</math> by one single inner function with an appropriate shift in its argument. He proved that there exist real values <math>\eta, \lambda_1,\ldots,\lambda_n</math>, a continuous function <math>\Phi\colon \mathbb{R} \rightarrow \mathbb{R}</math>, and a real increasing continuous function <math>\phi\colon [0,1] \rightarrow [0,1]</math> with <math>\phi \in \operatorname{Lip}(\ln 2/\ln (2N+2))</math>, for <math>N \geq n \geq 2</math>, such that

:<math> f(\bold x) = \sum_{q=0}^{2n} \Phi\left(\sum_{p=1}^{n} \lambda_p \phi(x_{p}+\eta q)+q \right)</math>.

Phillip A. Ostrand <ref>Phillip A. Ostrand, ''Dimension of metric spaces and Hilbert's problem 13'', [[Bulletin of the American Mathematical Society]], 71 (1965), pp. 619–622.</ref> generalized the Kolmogorov superposition theorem to compact metric spaces. For <math>p=1,...,m</math> let <math>X_p</math> be compact metric spaces of finite dimension <math>n_p</math> and let <math>n = \sum_{p=1}^{m} n_p</math>. Then there exists continuous functions <math>\phi_{q,p}\colon X_p \rightarrow [0,1], q=0,\ldots,2n, p=1,\ldots,m</math> and continuous functions <math>G_q\colon [0,1] \rightarrow \mathbb{R}, q=0,\ldots,2n</math> such that any continuous function <math>f\colon X_1 \times \dots \times X_m \rightarrow \mathbb{R}</math> is representable in the form 

:<math> f(x_1,\ldots,x_m) = \sum_{q=0}^{2n} G_{q}\left(\sum_{p=1}^{m} \phi_{q,p}(x_{p})\right) </math>.

==Original references==
*[[Andrey Kolmogorov|A. N. Kolmogorov]], "On the representation of continuous functions of several variables by superpositions of continuous functions of a smaller number of variables", ''[[Proceedings of the USSR Academy of Sciences]]'', 108 (1956), pp.&nbsp;179–182; English translation: ''Amer. Math. Soc. Transl.'', 17 (1961), pp.&nbsp;369–373.
*[[Vladimir Arnold|V. I. Arnold]], "On functions of three variables", ''Proceedings of the USSR Academy of Sciences'', 114 (1957), pp.&nbsp;679–681; English translation: ''Amer. Math. Soc. Transl.'', 28 (1963), pp.&nbsp;51–54.

==Further reading==
*S. Ya. Khavinson, ''Best Approximation by Linear Superpositions (Approximate Nomography)'', AMS Translations of Mathematical Monographs (1997)

==References==
{{reflist}}

 
