\section{Two-point boundary problems and finite element discretization}
Denote the functional space
$$
V:=H^1_0(0,1)=\{v:\, [0,1]\rightarrow R,\,\,\mbox{$v$ is continuous and}\, v(0)=v(1)=0\}.
$$
Given any $f: [0,1]\rightarrow R$, consider 
\begin{equation*}
J(v)=\frac12\int_{0}^1|v'|^2dx-\int_{0}^1 fvdx.
\end{equation*}
Find $u\in V$ such that 
\begin{equation}\label{minvar}
\displaystyle u=\argmin_{v\in V} J(v)\quad
\end{equation}
\begin{theorem}
Problem \eqref{minvar} is equivalent to: Find $u\in V$ such that 
\begin{equation}\label{1Dpossion}
\left\{
\begin{aligned}
-u''&= f, \,\, 0<x<1, \\
 u(0)&=u(1)=0.
\end{aligned}
\right.
\end{equation}
\end{theorem}
\begin{proof}
For any $v\in V, t\in R$, let $g(t)=J(u+t v)$. 
Since $u=\argmin_{v\in V} J(v)$ means $g(t)\ge g(0)$. Hence, for any $v\in V$, $0$ is the global minimum of the function $g(t)$. Therefore
$g'(0)=0$ implies 
$$
\int_{0}^1 u'v'dx=\int_0^1fvdx\quad \forall v\in V. 
$$
By integration by parts, which is equivalent to 
$$
\int_{0}^1 (-u''-f)vdx=0\quad \forall v\in V. 
$$
It can be proved that the above identity holds if and only if 
$-u''= f$ for all $x\in (0,1)$. Namely $u$ satisfies \eqref{1Dpossion}. 
\end{proof}

Let $V_h$ be finite element space and $\{\varphi_1, \varphi_2, \cdots \varphi_{n}\}$ be a nodal basis of $V_h$. 
%Let $\{\psi_1,\psi_2,\cdots, \psi_{n}\}$ be a dual basis of $\{\varphi_1, \varphi_2, \cdots \varphi_{n}\}$, namely $(\varphi_i,\psi_j )=\delta_{ij}$.
\begin{equation}
J(v_h)=\frac12\int_0^1|v_h'|^2dx-\int_0^1fv_hdx.
\end{equation}
Let 
$$
\displaystyle v_h=\sum_{i=1}^{n}\nu_i\varphi_i,\quad \nu=(\nu_i)_{i=1}^{n}
$$
then using the convolution notation, we have
$$
J(v_h)=I(\nu)=\frac12\nu^TA\ast\nu-b^T\nu
$$
and 
$$
\nabla I(\nu) =A\ast \nu -b,
$$
where $A=\frac{1}{h}[-1,2, -1]$.

At the same time, let $\displaystyle u_h=\sum_{i=1}^{n}\mu_i\varphi_i,$
\begin{equation}\label{min1d}
\displaystyle u_h=\argmin_{v_h\in V_h} J(v_h)\Leftrightarrow \mu=\argmin_{\nu \in R^n} I(\nu)
\end{equation}
And $u_h$ solves the problem: Find $u_h\in V_h$
\begin{equation}\label{1ddiscrevar}
\frac{d}{dt}J(u_h+t v_h)|_{t=0}=a(u_h,v_h)-\langle f, v_h\rangle=0 \quad \forall v_h\in V_h. 
\end{equation}
where 
$$
a(u_h,v_h)=\int_0^1 u_h'v_h'dx.
$$

\begin{theorem}\label{thm:best-approximation}
Let $V$ be a Hilbert space, $a(\cdot,\cdot)$ be a continuous,
symmetric, bilinear form defining an inner product on $V$. Let
$f(\cdot)$ be a continuous linear form. If $u$ is solution to the
problem~\eqref{minvar} and $u_h$ is a solution
to~\eqref{min1d}, then the following equality holds:
\begin{equation}\label{eq:best-approx}
\|u-u_h\|_a = \inf_{v\in V_h}\|u-v\|_a,
\end{equation}
where $\|u-u_h\|^2_a = a(u-u_h,u-u_h)$. 
\end{theorem}

\begin{proof}
Frist, one sees that
$$
a(u-u_h, v)=0,~~~\forall v\in V_h.
$$
Thus, for any $v\in V_h$ we have
\begin{eqnarray*}
& &\|u-u_h\|_a^2=a(u-u_h, u-u_h)\\[0.2cm]
&=& a(u-u_h, u-v)\\[0.2cm]
&\le& \|u-u_h\|_a\|u-v\|_a.
\end{eqnarray*}
Taking the infimum on both sides of this equation then gives. 
$$
\|u-u_h\|_a\le \inf_{v\in V_h}\|u-v\|_a.
$$
The proof is complete, since 
\[
\|u-u_h\|_a\ge \inf_{v\in V_h}\|u-v\|_a,
\]
by the definition of infimum. 
\end{proof}
Combining Theorem \ref{interp00} and Theorem \ref{thm:best-approximation}, we obtain the error estimate for the finite element method. 
\begin{theorem}
If $u\in H^2(\Omega)$ is solution to the
problem~\eqref{1Dpossion} and $u_h$ is a solution to~\eqref{min1d}, then we have:
\begin{equation}
 \|u-u_h\|+h |u-u_h|_{1}\lc h^2 |u|_2.
   \end{equation}
 \end{theorem}

