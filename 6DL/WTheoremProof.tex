\begin{theorem}
Let $\Omega\subset R^n$ be a  closed and bounded set. Given any continuous function $f(x)$ on $\Omega$, there exists a sequence of polynomials $\{p_n(x)\}$ such 
that 
\begin{equation}
\displaystyle \lim_{n\rightarrow \infty} \max_{x\in \Omega}|f(x)-p_n(x)|=0
\end{equation}
\end{theorem}
\begin{proof}
Let us first give the proof for $d=1$ and $\Omega=[0,1]$. Given $f:[0,1]\rightarrow R$ be a  continuous function. 

Let
\begin{equation}
\tilde f(x)=f(x)-l(x)
\end{equation}
where $l(x)=f(0)+x(f(1)-f(0))$.
Then $\tilde f(0)=\tilde f(1)=0$. Noting that $l(x)$ is a linear function, hence without lose of generality, we can only consider the 
case $f:[0,1]\rightarrow R$ with $f(0)=f(1)=0$. 
Since $f$ is continuous on the closed interval $[0,1]$, then $f$ is uniformly continuous on $[0,1]$.

First we extend $f$ to be zero outside of $[0,1]$ and obtain $f: R\rightarrow R$, then it is obviously that $f$ is still uniformly continuous. 

Next for $0\le x\le 1$, we construct
\begin{equation}
p_n(x)=\int_{-1}^1f(x+t)Q_n(t)dt=\int_{-x}^{1-x}f(x+t)Q_n(t)dt=\int_{0}^{1}f(t)Q_n(t-x)dt
\end{equation} 
where $Q_n(x)=c_n(1-x^2)^n$ and 
\begin{equation}\label{intq}
\int_{-1}^1 Q_n(x) dx=1.
\end{equation} 
Thus $\{p_n(x)\}$ is a sequence of polynomials. 

Since 
\begin{align}
\int_{-1}^1 (1-x^2)^n dx&=2\int_{0}^1 (1-x^2)^n dx=  2\int_{0}^1 (1-x)^n(1+x)^n dx\\ 
&\ge 2\int_{0}^1 (1-x)^n dx=\frac{2}{n+1}> \frac{1}{n}.
\end{align}
Combing with $\int_{-1}^1 Q_n(x) dx=1$, we obtain $c_n< n$ implying that for any $\delta>0$
 \begin{equation}\label{qest}
 0\le Q_n(x)\le n(1-\delta^2)^n \quad (\delta\le |x|\le 1),
 \end{equation}
so that $Q_n\rightarrow 0$ uniformly in $\delta\le |x|\le 1$ as $n\rightarrow \infty$. 

Given any $\epsilon >0$, since $f$ in uniformly continuous, there exists $\delta>0$ such that for any $|y-x|<\delta$, we have 
\begin{equation}\label{fcont}
|f(y)-f(x)|< \frac{\epsilon}{2}.
\end{equation}
Finally, let $M=\max |f(x)|$, using \eqref{fcont}, \eqref{intq} and \eqref{qest}, we have 
\begin{align}
\big| p_n(x)-f(x)\big|&=\big|\int_{-1}^1(f(x+t)-f(t))Q_n(t)dt\big|\le \int_{-1}^1 \big| f(x+t)-f(t)\big| Q_n(t)dt\\
&\le 2M \int_{-1}^{-\delta} Q_n(t)dt+ \frac{\epsilon}{2}\int_{-\delta}^{\delta} Q_n(t)dt+ 2M\int_{\delta}^1 Q_n(t)dt\\
&\le 4M n(1-\delta^2)^n + \frac{\epsilon}{2}< \epsilon
\end{align}
for all large enough $n$, which proves the theorem. 

The above proof generalize the high dimensional case easily.   We
consider the case that
$$
\Omega=[0,1]^d.
$$
By extension and using cut off function,  W.L.O.G.  that we assume
that $f=0$ on the boundary of $\Omega$ and we then extending this
function to be zero outside of $\Omega$.  

Let us consider the special polynomial functions
\begin{equation}
  \label{Qn}
Q_n(x)=c_n\prod_{k=1}^d(1-x_k^2)  
\end{equation}
Similar proof can then be applied. 
\end{proof}






















