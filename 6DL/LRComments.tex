\section{Additional comments}
\paragraph{\textbf{Remarks}}
\begin{enumerate}
	
	\item $k$ disjoint sets $A_1,A_2,\cdots,A_k$ represent $k$ classes. We use $A$ to denote
	\[
	A = A_1\cup A_2\cup \cdots \cup A_k. 
	\]
	Naturally we can define the standard decision mapping $\bm{\chi}:A \rightarrow \{e_1,e_2,\cdots,e_k\}\subset \mathbb{R}^k$  as
	\begin{equation}
	\bm{\chi}(x) = e_i,\ x\in A_i,
	\end{equation}
	which is just like a characteristic function.
	
	\item To solve a classification problem, our goal is to find the standard decision mapping. But usually it's unpractical, because we seem to be better at dealing with continuous functions with continuous outputs. A natural idea is to reach the goal in two steps. We can use a composition of the following two mappings 
	$$\bm{f}:A \rightarrow \mathbb{R}^k;\ x\mapsto y.$$
	$$\bm{\pi}: \mathbb{R}^k \rightarrow \{e_1,e_2,\cdots,e_k\};\ \ y\mapsto e_i,i = \argmax_j~y_j$$
	to obtain the standard decision mapping. Notice that the mapping $\bm{\pi}$ is fixed, so we only need to find a proper mapping $\bm{f}$. 
	
	\item Assume that we want to find a proper $\bm{f}$ in a mapping sapce $\mathscr{H}$. The result of a classification problem depends on which $\mathscr{H}$ to choose and how to find the optimal mapping $\bm{f}$ in $\mathscr{H}$.\\
	
	\item Given a mapping space $\mathscr{H}$, if there exists an $\mathbf{f}\in \mathscr{H}$ such that 
	\begin{equation}
	\bm{\chi} = \bm{\pi} \circ \bm{f}.
	\end{equation}
	we can say that $A_1,A_2,\cdots,A_k$ are $\mathscr{H}$-separable. Setting $\mathscr{H}$ to be the affine mapping space, $i.e.$ $\mathscr{H} = \{\bm{f}: \bm{f}(x) = Wx+b\}$, we obtain the definition of linearly separable.\\
	
	\item A linear model  is a method to find the optimal $W$ and $b$ when we take $\mathscr{H} = \{\bm{f}: \bm{f}(x) = Wx+b\}$.\\
	
	\item Assume that $\bm{\phi}$ is a feature mapping from $A$ to a feature space $\mathcal{H}$. A natural choice of $\mathscr{H}$ is 
	\begin{equation}
	\mathscr{H} = \{\bm{f}: \bm{f}(x) = W\bm{\phi}(x)+b\}
	\end{equation}
	%If we set $\mathcal{H}$ to be a Reproduced Kernel Hilbert Space $w.r.t.$ a kernel function $\bm{k}$ and use SVM to find the optimal $W$ and $b$, we can obtain the methods of kernel SVM.
\end{enumerate}


\subsection{General approach for separation }
A collection of sets $A_i\subset \mathbb{R}^n$, $1\le i\le k$, are said to be separable if there exists a continuous function $h:\mathbb{R}^n\to\mathbb{R}^k$ such that $h(x)=e_i$.

A general approach:
\begin{itemize}
	\item Find a ``nonlinear" function $h_0:\mathbb{R}^n\to\mathbb{R}^m$ such that
\begin{enumerate}
	\item $\tilde{A}_i=h_0(A_i)\subset\mathbb{R}^m$ are linearly separable
	\item $m\ll n$
\end{enumerate}
\item Use a ``linear" classifier $l_0:\mathbb{R}^m\to\mathbb{R}^k$, then $h=l_0\circ h_0:\mathbb{R}^n\to\mathbb{R}^k$ is the classifier.
\end{itemize}

\begin{remark}
	\begin{enumerate}
		\item $h_0$: dimension reduction;
		\item Feature extraction;
		\item $h_0$: deep neural network, special function class;
		\item $x^{l+1}=\sigma(Wx^l+b)$, for CNN, $W,b$ are ``sparse";
		\item ``linear classifier" is fully connected layer.
	\end{enumerate}
	\end{remark}

\subsection{Feature map}

\begin{remark}
The "linear" here means that $wx+b$ is linear with respect to $x$. This condition is not crucial. Actually, we can replace $x$ by $\phi: \mathbb{R}^n\rightarrow \mathbb{R}^m$. To be specific, let  $\tilde x=w\phi(x)+b$ and $\tilde A_i=\phi (A_i)\in \mathbb{R}^m$. The function $\phi$ here is a feature map, 
common choices include
\begin{enumerate}
\item $\phi(x)=x$;
\item finite elements, wavelets;
\item deep neural networks.
\end{enumerate}
\end{remark}

\section{Examples of Linearly Separable Sets}
In this section, we shall give several examples of linearly separable sets.  We will give a detailed description of these sets and list the web pages that contain these sets. 

\subsection{L-MNIST sets}
The original MNIST data set contains the pictures of single digits. The size of each digit picture is 28*28 and the label is the digit from 0 to 9. There are 60000 training data and 10000 test data in all. The accuracy of the state-of-art CNN model applied on the MNIST is higher than 99\%. If we apply standard least square to the original MNIST data, we only get 86\% accuracy. If we apply logistic regression to the original MNIST data, we get 92\% accuracy. This means that the original MNIST is linearly separable. 

This collection of linearly separable L-MNIST data set are derived from the MNIST data basis after applying some appropriate DNN model. The fully connected neural network contains an input layer, a hidden layer and an output layer. The corresponding widths are 784, 500 and 10. We use this DNN to train all the data (70000) in MNIST for 50 epochs with cross entropy as the loss function and Adam as the optimization method. Then the training accuracy reaches 100\%. We save this model and define the output of the hidden layer as the linearly separable L-MNIST data set. 

To verify this data set is indeed linearly separable, we can use a fully connected neural network with input layer (500) and output layer (10) only without activation. So this is a straightforward linear combination instead of  logistic regression. Due to the weight initialization, the immediate classification accuracy may be low. But after training with this DNN for only 3 epochs, the classification accuracy for all the data in MNIST is already 100\%.  As a comparison, if we apply standard least square to the linearly separable MNIST data, we get 98.38\% accuracy.

We have also uploaded the linearly separable MNIST data set to \\
http://multigrid.org/wiki/index.php/deep-neural-network/
\subsection{CIFAR sets}

