\section{General ReLU DNN as linear finite element functions}

Here we will show that any ReLU DNN function can be viewed as linear finite element function. 
Let us first consider a ReLU DNN function with $L$ hidden layers
\begin{equation}\label{key}
f(x) = \theta^{L}\circ \sigma \circ \theta^{L-1} \circ \sigma \cdots \circ \theta^1 \circ \sigma \circ \theta^0(x),
\end{equation}
for any $X \in \mathbb{R}^d$.
Based on the property of ${\rm ReLU}$ activation function, we know that $f(x)$ is a continuous 
piecewise linear function. More precisely, let us consider a closed bounded domain, for example $\Omega = [-1, 1]^d$ and 
\begin{equation}\label{key}
\left. f(x) \right|_{E_i} = \ell_i(x),  \quad \text{ for } i =1,2,\cdots M,
\end{equation}
for any $x\in \Omega$, where $\ell_i(x) = w_i x + b_i$ is a linear function on $\mathbb{R}^d$ and 
$\{E_i\}_{i=1}^M$ are the domains which consist a partition of $\Omega$. 
Furthermore, we know that the boundary of $E_i$ consists of some hyperplanes 
of $d$-dimensional space and $E_i \cap E_j = \emptyset$ or $E_i \cap E_j $ belongs to
hyperplanes of lower dimensional space.

Then we have the following representation theory to represent any ReLU DNN function
by linear finite element function.
\begin{theorem}
Any ReLU DNN function on $[-1,1]^d$ can be represent
by a linear finite element function.
\end{theorem}
The proof of this theorem can be found in~\cite{he2020approximation}.


Here we show a brief description of the main steps in the proof, which will appear in~\cite{he2020approximation}.
\begin{enumerate}
	\item Based on the discussion before, we have a partition of $\Omega$ as
	\begin{equation}\label{key}
	\Omega = \bigcup_{i=1}^M E_i.
	\end{equation}
	\item We can extend all the $d$-dimensional hyperplanes on the boundary of all $E_i$ to achieve
	a fine partition of $\Omega$ noted as
	\begin{equation}\label{key}
	\Omega = \bigcup_{i=1}^{\tilde M} \tilde E_i,
	\end{equation}
	with an important feature that each $\tilde E_i$ is a convex n-dimensional polytope.
	\item For these $\{\tilde E_i\}_{i=1}^{\tilde M}$, we can further partition each $\tilde E_i$
	to be a collection of n-dimensional simples. Finally, we can rewrite the continuous
	piecewise linear function $f(x)$ on a simplicial mesh of $\Omega$, which means that $f(x)$
	will can be written as a linear finite element function.
\end{enumerate}



