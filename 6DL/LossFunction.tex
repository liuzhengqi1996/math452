\section{Loss functions:  nonlinear least square and cross-entropy}
Given a collection of data $D=(x_j,y_j)_{j=1}^N$ with $x_j\in A_{i_j}$
and $y_j=e_{i_j}$ We now discuss how to find $\theta =(W,b)$ such that
\begin{equation}
\label{fitting}
f(x_i,\theta)=\sigma(Wx_j+b)=y_j,\quad 1\le j \le N.
\end{equation}

\subsection{Least square}
The data fitting problem \eqref{fitting} can be solved by find a minimizer of the following loss function
\begin{equation}
\label{L0}
L_0(\theta)={1\over N}\sum_{i=1}^N\|f(x_i,\theta)-y_i\|^2
\end{equation}

Obviously,
$$
\theta^*=(W^*,b^*) \mathrm{satisfies}\ \eqref{fitting} \Leftrightarrow L_0(\theta^*)=0\Leftrightarrow \theta^*\in\arg\min_{\theta}L_0(\theta)
$$

We make the  following two observations on the loss function $L_0$ given by \eqref{L0}:
\begin{enumerate}
	\item It has infinitely many global minimizers.
	\item It is a non-convex loss function.
\end{enumerate}


%\newbreak
\section{KL divergence and cross-entropy}
Cross-entropy minimization is frequently used in optimization and rare-event probability estimation. When comparing a distribution    against a fixed reference distribution, cross-entropy and KL divergence are identical up to an additive constant. See more details in \cite{murphy2012machine,kullback1951information,kullback1997information} and the reference therein.

The KL(Kullback--Leibler) divergence defines a special distance between two discrete probability distributions 
$$
p=\left( \begin{array}{ccc}
p_1\\
\vdots \\
p_k
\end{array} \right),\quad  q=\left( \begin{array}{ccc}
q_1\\
\vdots \\
q_k
\end{array} \right)
$$
with $0\le p_i, q_i\le1$ and $\sum_{i=1}^{k}p_i=\sum_{i=1}^{k}q_i=1$ by
\begin{equation}
\label{KL-divergence}
D_{\rm KL}(q,p)= \sum_{i=1}^k q_i\log \frac{q_i}{p_i}.  
\end{equation}

\begin{lemma}\mbox{}
$D_{\rm KL}(q,p)$ works like a ``distance" without the symmetry:
	\begin{enumerate}
		\item $D_{\rm KL}(q,p)\ge0$;		
		\item $D_{\rm KL}(q,p)=0$ if and only if $p=q$;
	\end{enumerate}
\end{lemma}

\begin{proof}We first note that the elementary inequality
	\begin{equation}
	\log x \le x - 1, \quad\mathrm{for\ any\ }x\ge0,
	\end{equation}
	and the equality holds if and only if $x=1$.
	\begin{equation}
	-D_{\rm KL}(q,p) = - \sum_{i=1}^c q_i\log \frac{q_i}{p_i}   = \sum_{i=1}^k q_i\log \frac{p_i}{q_i} \le \sum_{i=1}^k q_i( \frac{p_i}{q_i}  - 1) = 0.
	\end{equation}
	And the equality holds if and only if 
	\begin{equation}
	\frac{p_i}{q_i} = 1 \quad \forall i = 1:k.
	\end{equation}
\end{proof}

Define cross-entropy for distribution $p$ and $q$ by
\begin{equation}\label{Cross-Entropy}
H(q,p) = - \sum_{i=1}^k q_i \log p_i,
\end{equation}
and the entropy for distribution $q$ by 
\begin{equation}\label{Entropy}
H(q) = - \sum_{i=1}^k q_i \log q_i.
\end{equation}
Note that
\begin{equation}
D_{\rm KL}(q,p)= \sum_{i=1}^k q_i\log \frac{q_i}{p_i} =  \sum_{i=1}^k q_i \log q_i - \sum_{i=1}^k q_i \log p_i
\end{equation}
Thus, 
\begin{equation}\label{KLandEntropy}
H(q,p) = H(q) + D_{\rm KL}(q,p).
\end{equation} 
It follows from the relation \eqref{KLandEntropy} that
\begin{equation}
\label{EntropyandKL}
\mathop{\arg\min}_p D_{\rm KL}(q,p)=\mathop{\arg\min}_p H(q,p).
\end{equation}

The concept of cross-entropy  can be used to define a loss function in machine learning and optimization. 
Let us assume $y_i$ is the true label for $x_i$, for example $y_i = e_{k_i}$ if $x_i \in A_{k_i}$. Consider the predicted distribution 
\begin{equation}\label{key}
\bm p(x; \bm \theta) = \frac{1}{\sum\limits_{i=1}^k e^{w_i x+b_i}}
\begin{pmatrix}
e^{w_1 x+b_1}\\
e^{w_2 x+b_2}\\
\vdots\\
e^{w_k x+b_k}
\end{pmatrix}
= \begin{pmatrix}
p_1(x; \bm\theta) \\
p_2(x; \bm\theta) \\
\vdots \\
p_k(x; \bm\theta)
\end{pmatrix}
\end{equation}
for any data $x \in A$.
By \eqref{EntropyandKL}, the minimization of KL divergence is equivalent to the minimization of the cross-entropy, namely
\begin{equation}
\mathop{\arg\min}_{\theta} \sum_{i=1}^N D_{\rm KL}(y_i,\bm p(x_i; \bm \theta)) = \mathop{\arg\min}_{\theta} \sum_{i=1}^N H(y_i, \bm p(x_i; \bm \theta)).
\end{equation} 
Recall that we have all data $D = \{(x_1,y_1),(x_2,y_2),\cdots, (x_N, y_N)\}$.  Then, it is natural to consider the 
loss function as following:
\begin{equation}
\sum_{j=1}^N H(y_i, \bm p(x_i; \bm \theta)),
\end{equation}
which measures the distance between the real label and predicted one for all data.
In the meantime, we can check that
\begin{equation}
\begin{aligned}
\sum_{j=1}^N H(y_j, \bm p(x_j; \bm \theta))&=-\sum_{j=1}^N y_j  \cdot \log  \bm p(x_j; \bm \theta )\\
&=-\sum_{j=1}^N  \log p_{i_j}(x_i; \bm \theta) \quad (\text{because}~y_j = e_{i_j}~\text{for}~x_j \in A_{i_j})\\
&=-\sum_{i=1}^k \sum_{x\in A_i}  \log p_{i}(x; \bm \theta) \\
&=-\log \prod_{i=1}^k \prod_{x\in A_i}   p_{i}(x; \bm \theta)\\
& = L(\theta)
\end{aligned}
\end{equation}
with $L(\theta)$ defined in \eqref{logistic} as 
\begin{equation}
L(\bm \theta) = - \sum_{i=1}^k \sum_{x\in A_i} \log p_{i}(x;\bm \theta).
\end{equation}

That is to say, the logistic regression loss function defined by likelihood in \eqref{logistic} is exact
the loss function defined by measuring the distance between real label and predicted one via cross-entropy.
We can note 
\begin{equation}\label{key}
\min_{\bm \theta} L_\lambda(\bm \theta) \Leftrightarrow \min_{\bm \theta} \sum_{j=1}^N H(y_i, \bm p(x_i; \bm \theta)) + \lambda R(\|\bm \theta\|) 
\Leftrightarrow \min_{\bm \theta} \sum_{j=1}^N D_{\rm KL}(y_i, \bm p(x_i; \bm \theta)) + \lambda R(\|\bm \theta\|).
\end{equation}



\endinput


\subsection{Convex versus non-convex loss functions}
\subsubsection{Convexity}
The maximum likelihood and maximum posterior look difficult to optimize at first glance. Remarkably though,
the logistic regression has the property that the negative log likelihood is a convex function of the parameters $W,b$.
This means that we can apply a well-developed convergence theory for methods such as gradient descent, accelerated gradient
descent, stochastic gradient descent, etc. Our goal in this section will be to prove this remarkable fact.

The first thing to note is that the negative log-likelihood
$$\displaystyle\sum_{j=1}^n \left(\log(\mathbbm{1}^T\exp(W x_j + b)) - 
 (W x_j + b)_{i_j})\right)$$
is a sum of terms over all of the data points $(x_i,y_i)$. Since a sum of convex functions is convex, it suffices to show that
the term corresponding to a single data point is convex (i.e. the negative log-likelihood of a single data point is a convex
function). So we consider the function
\begin{equation}
 f_{x,i}(W,b) = \log(\mathbbm{1}^T\exp(W x + b)) - 
 (W x + b)_{i}
\end{equation}
We will show that this function is convex for any choice of data $x$ and label $i$. First note that the second term
$$ - (W x + b)_{i}$$
is a linear function, so it suffices to show that the first term
$$\log(\mathbbm{1}^T\exp(W x + b))$$
is convex. To see this, we note that $W x + b$ is a linear function of the parameters $W,b$. Combining
this with the easy fact that a convex function composed with a linear function is convex, we only need to consider the LogSumExp
function $s:\mathbb{R}^k\rightarrow \mathbb{R}$ defined by
\begin{equation}
 s(x) = \log(\exp(x_1) + \cdots + \exp(x_k))
\end{equation}
We can easily verify that this function is convex by calculating its Hessian and noting that $\nabla^2s$ is diagonally dominant.
%\begin{equation}
% D^2s(x) =   \left(\displaystyle\sum_{i=1}^k \exp(x_i)\right)^{-2}\begin{pmatrix}
%    \left(\displaystyle\sum_{i\neq 1} \exp(x_1 + x_i)\right) & -\exp(x_1 + x_2) & -\exp(x_1 + x_3) & \cdots & -\exp(x_1 + x_k) \\
%    -e^{x_1}e^{x_2} & e^{x_2}(e^{x_1} + e^{x_3} + \cdots + e^{x_k}) & -e^{x_2}e^{x_3} & \cdots & -e^{x_2}e^{x_k}
%  \end{pmatrix} 
%\end{equation}

