% Qingguo put the Poisson summation formula here in this file:  statement and sketch of proof

\begin{theorem}
Let $f \in L^{1}(\mathbb{R})$ and $f$ is continuous. Then we have for almost all $(x, \omega ) \in \mathbb{R} \times \hat{\mathbb{R}}$ that
$$
T \sum_{n \in \mathbb{Z}} f(x+n T) e^{-2 \pi i \omega (x+n T)}=\sum_{n \in \mathbb{Z}} \hat{f}\left(\omega +\frac{n}{T}\right) e^{2 \pi i n x / T}
$$
where both sides converge absolutely.

In addition,  let $\Lambda$ be the lattice in $\mathbb{R}^{d}$ consisting of points with integer coordinates. 
For a function $f$ in $L^{1}\left(\mathbb{R}^{d}\right)$ and $f$ is continuous, we have 
$$
\sum_{\omega  \in \Lambda} f(x+\omega )=\sum_{\nu \in \Lambda} \hat{f}(\omega ) e^{2 \pi i x \cdot \omega }.
$$
where both series converge absolutely and uniformly on $\Lambda$. 
\end{theorem} 

\begin{proof}
We just give a proof of a simple case that $f: \mathbb{R} \rightarrow \mathbb{C}$ is a Schwarz function (see Definition \ref{def:schwarz}).
Let:
$$
F(x)=\sum_{n \in \mathbb{Z}} f(x+n).
$$
Then $F(x)$ is 1-periodic (because of absolute convergence), and has Fourier coefficients:
$$
\begin{aligned}
\hat{F}_{\omega } &=\int_{0}^{1} \sum_{n \in \mathbb{Z}} f(x+n) e^{-2 \pi i \omega x} \mathrm{~d} x \\
&=\sum_{n \in \mathbb{Z}} \int_{0}^{1} f(x+n) e^{-2 \pi i \omega  x} \mathrm{~d} x \quad \text { because } f \text { is Schwarz, so convergence is uniform}\\
&=\sum_{n \in \mathbb{Z}} \int_{n}^{n+1} f(x) e^{-2 \pi i\omega  x} \mathrm{~d} x \\
&=\int_{\mathbb{R}} f(x) e^{-2 \pi i \omega  x} \mathrm{~d} x\\
&=\hat{f}(k)\\
\end{aligned}
$$
 where $\hat{f}$ is the Fourier transform of $f$.
 

Therefore by the definition of the Fourier series of $f:$
$$
F(x) =\sum_{\omega  \in \mathbb{Z}} \hat{f}(k) e^{2\pi i \omega x}.
$$
Choosing $x=0$ in this formula:
$$
\sum_{n \in \mathbb{Z}} f(n)=\sum_{\omega  \in \mathbb{Z}} \hat{f}(\omega )
$$
as required.
\end{proof}





