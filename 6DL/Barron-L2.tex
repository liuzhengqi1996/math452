

\subsection{General activation functions}
Assume that $\sigma$ is a locally Riemann integrable nonzero function 
and $\sigma\in L^1(\mathbb{R})$ and thus the Fourier transform of
 $\sigma$ is well-defined and continuous. 
 Since $\sigma$ is non-zero and 
 \begin{equation}\label{key}
 \hat \sigma(\omega) = \int_{\mathbb{R}} \sigma(t)e^{-2\pi i\omega t}dt,
 \end{equation}
 this implies that $\hat{\sigma}(a)\neq 0$ for
 some $a\neq 0$. Via a change of variables $t = \omega\cdot x + b$ and $dt = db$,
  this means that for all $x$ and $\omega$, we have
 \begin{equation}
 \begin{aligned}
  0\neq \hat{\sigma}(a)&= \int_{\mathbb{R}}\sigma(\omega\cdot x+b)e^{-2\pi ia(\omega\cdot x+b)}db \\
 & = e^{-2\pi ia\omega \cdot x} \int_{\mathbb{R}}\sigma(\omega\cdot x+b)e^{-2\pi iab}db ,
 \end{aligned}
 \end{equation}
 and so
 \begin{equation}
  e^{2\pi ia\omega \cdot x} = \frac{1}{\hat{\sigma}(a)}\int_{\mathbb{R}}\sigma(\omega\cdot x+b)e^{-2\pi iab}db.
 \end{equation}
 Likewise, since the growth condition also implies that $\sigma^{(k)}\in L^1$, we can differentiate the above expression  under the integral with respect to $x$.

 This allows us to write the Fourier mode $e^{2\pi ia\omega \cdot x}$ as an integral of neuron output functions. We substitute this
 into the Fourier representation of $u$
 (note that the assumption we make implies that $\hat{u}\in L^1$ so this
 is rigorously justified for a.e. $x$) to get
 \begin{equation}\label{integral_representation}
 \begin{split}
  u(x) &= \int_{\mathbb{R}^d} e^{2\pi i\omega\cdot x}\hat{u}(\omega)d\omega = 
  \int_{\mathbb{R}^d}\int_\mathbb{R}\frac{1}{ \hat{\sigma}(a)}
  \sigma\left(a^{-1}{\omega}\cdot
    x+b\right)\hat{u}(\omega)e^{-2\pi iab}dbd\omega
\\
&=  \int_{\mathbb{R}^d}\int_\mathbb{R} k(x,\theta) dbd\omega 
\end{split}
 \end{equation}
where $\theta=(\omega, b)$ 
and   
$$
k(x,\theta)= \frac{1}{ \hat{\sigma}(a)}
  \sigma\left(a^{-1}{\omega}\cdot
    x+b\right)\hat{u}(\omega)e^{-2\pi iab}.
$$
%\section{Asymptotic convergence estimates for $L^2$ norm}
%\subsection

 
\iffalse
Then we propose the next assumption on $\sigma$
\begin{assumption}\label{assump:sigma}[Siegle \& Xu 2020]
	Let $\sigma \in L^{\infty}(\Omega)$ and there exists $p> 1$  such that
	\begin{equation}\label{key}
	|\sigma(t)| \le (1+|t|)^{-p}.
	\end{equation}
\end{assumption}


Then we have the next important lemma about the estimate of $\rho(\theta)$.
\fi

\begin{theorem}\label{approximation_rate_theoreml2}
 Let $\Omega\subset \mathbb{R}^d$ be a bounded domain. If the activation function $\sigma $ is non-zero and satisfies the polynomial decay condition 
$$
\sigma(t) \le (1+|t|)^{-p}
$$
for some $p>1$, then for any $n \ge 1$, there exist 
	$\theta_i = (\omega_i, b_i) \in \mathbb{R}^{d+1}$ such that
	\begin{equation}\label{key}
	\|u-u_n\|_{L^{2}(\Omega)} \lesssim n^{-\frac{1}{2}}\|u\|_{\mathcal B^1(\Omega)},
	\end{equation}
	where
	\begin{equation}\label{key}
	u_n(x) = \frac{\|\rho\|_{L^1}}{n} \sum_{i=1}^n \beta_i {\sigma}(a^{-1}\omega_i\cdot x + b_i) %{\rm Re} \left( \hat f(\omega_i)e^{iab}\right) 
	\in \dnn(\sigma, n).
	\end{equation}
\end{theorem}

\begin{theorem}\label{approximation_rate_theorem}
 Let $\Omega\subset \mathbb{R}^d$ be a bounded domain. If the activation function $\sigma\in W^{m,\infty}(\mathbb{R})$ is non-zero and satisfies the polynomial decay condition 
 \begin{equation}\label{growth_condition}
  |\sigma^{(k)}(t)| \leq C_p(1 + |t|)^{-p}
 \end{equation}
 for $0\leq k\leq m$ and some $p > 1$, we have
 \begin{equation}
  \inf_{u_n\in \dnn(\sigma,n)}\|u-u_n\|_{H^m(\Omega)} \leq |\Omega|^{\frac{1}{2}}C(p,m,\text{\normalfont diam}(\Omega),\sigma)n^{-\frac{1}{2}}\|u\|_{\mathcal{B}^{m+1}(\Omega)},
 \end{equation}
 for any $u\in \mathcal{B}^{m+1}$.
\end{theorem}
Before we proceed to the proof, we discuss how this bound depends on
the dimension $d$. We first note that $|\Omega|$ may in a sense depend
on the dimension, as the measure may be exponentially large in high
dimensions. However, bounding the $H^m$ error over a larger set is
also proportionally stronger. This can be seen by noting that dividing
by the $|\Omega|^\frac{1}{2}$ factor transforms the left hand side
from the total squared error to the average squared error.

The dimension dependence of this result is a consequence of how the
Barron norm behaves in high dimensions. This issue is discussed in
\cite{barron1993universal}, where the norm $\|\cdot\|_{\mathcal{B}^1}$
is analyzed for a number of different function classes. A particularly
representative result found there is that $H^{\frac{d}{2}+2}\subset
\mathcal{B}^1$. This shows that sufficiently smooth functions have
bounded Barron norm, where the required number of derivatives depends
upon the dimension. It is known that approximating functions with such
a dimension dependent level of smoothness can be done efficiently
\cite{petrushev1998approximation, kainen2007sobolev}. However, the
Barron space $\mathcal{B}^1$ is significantly larger that
$H^{\frac{d}{2}+2}$, in fact we only have $\mathcal{B}^1 \subset H^1$
by lemma \ref{smoothness-lemma}. The precise properties of the Barron
norm in high dimensions are an interesting research direction which
would help explain exactly how shallow neural networks help alleviate
the curse of dimensionality.


Next we consider the proof for Theorem \ref{approximation_rate_theoreml2}. Recall
 \begin{equation} 
u(x) =  \int_{\mathbb{R}^d}\int_\mathbb{R} k(x,\theta) dbd\omega,\quad 
  k(x,\theta)= \frac{1}{ \hat{\sigma}(a)}
  \sigma\left(a^{-1}{\omega}\cdot
    x+b\right)\hat{u}(\omega)e^{-2\pi iab}. 
 \end{equation}
where $\theta=(\omega, b)$. 
 Define 
\begin{equation}\label{key}
h(\omega, b) = \max_{x\in \Omega}|\sigma(a^{-1}\omega\cdot x+b)|, \quad \mbox{ and }\quad \rho(\theta) = h(\omega, b)  |\hat u(\omega)|.
\end{equation}
If $ \rho(\theta) \in L^1(\mathbb{R}^{d+1})$, then $f(x)=\mathbb{E}(k(x,\theta))$. By the Monte Carlo method in Theorem \ref{MC}, it is sufficient to prove $ \rho(\theta) \in L^1(\mathbb{R}^{d+1})$. 
\begin{lemma}
Let $\sigma \in L^{\infty}(\Omega)$ and there exists $p> 1$  such that
	\begin{equation}\label{key}
	|\sigma(t)| \le (1+|t|)^{-p},
	\end{equation}
then we have 
$$
\rho(\theta) \in L^1(\mathbb{R}^{d+1}).
$$
\end{lemma}
\begin{proof}
Note that
\begin{equation}
\label{eq:4}
\begin{aligned}
|k(x,\theta)| &\le \frac{1}{ |\hat \sigma(a)|} \max_{x\in \Omega} |\sigma\left(a^{-1}{\omega}\cdot
x+b\right) | |\hat u(\omega)|  \\
&\le  \frac{1}{ |\hat \sigma(a)|}  h(\omega, b)|\hat u(\omega)|   =  \frac{1}{  |\hat \sigma(a)|} \rho(\theta)
\end{aligned},
\end{equation}
and
	\begin{equation}\label{key}
	\|\rho\|_{L^1(\mathbb{R}^{d+1})} = \int_{\mathbb{R}^{d+1}} |\rho(\theta)|d\theta = \int_{\mathbb{R}^d} \left( \int_{\mathbb{R}} h(\omega, b)db\right) |\hat u(\omega)| d\omega.
	\end{equation}
Note that
\begin{equation}\label{key}
|a^{-1}\omega \cdot x + b| \ge |b| - |a^{-1}\omega \cdot x | \ge  |b| - |a^{-1}||\omega| | x|  \ge |b| - |a^{-1}| |\omega | R,
\end{equation}
where 
\begin{equation}\label{key}
R = \max_{x\in \bar \Omega} |x|,
\end{equation}
as $\Omega$ is bounded.
Thus
\begin{equation}\label{key}
|a^{-1}\omega \cdot x + b| \ge \max(0, |b| - \frac{R}{|a|} |\omega |).
\end{equation}
That is to say
\begin{equation}\label{key}
h(\omega, b) \le (1+  \max(0, |b| - \frac{R}{|a|}|\omega|))^{-p},
\end{equation}
	Then we calculate
	\begin{equation}\label{eq_775}
	\begin{split}
	\int_\mathbb{R} h(\omega,b)db& \le \int_{|b|\leq \frac{R|\omega|}{|a|}} db + 2\int_{b > \frac{R\|\omega\|}{|a|}} \left(1 + b - \frac{R|\omega|}{|a|}\right)^{-p}db \\
	& =~2R|a|^{-1}|\omega| + 2\left[(1-p)^{-1}\left(1 + b - \frac{R|\omega|}{|a|}\right)^{1-p}\right]_{\frac{R|\omega|}{|a|}}^\infty \\
	&=~2R|a|^{-1}|\omega| + \frac{2}{p-1}\leq C_1(p,\text{\normalfont diam}(\Omega),\sigma) (1 + |\omega|).
	\end{split}
	\end{equation}
	Thus, we have
	\begin{equation}\label{key}
		\|\rho\|_{L^1(\mathbb{R}^{d+1})} \le C_1\int_{\mathbb{R}^d} (1+|\omega|)|\hat u(\omega)| d\omega.
	\end{equation}
\end{proof}
Here we denote 
\begin{equation}\label{key}
\|u\|_{\mathcal B^1(\Omega)} = \int_{\mathbb{R}^d} (1+\|\omega\|)|\hat u(\omega)| d\omega,
\end{equation}
namely
\begin{equation}\label{key}
\|\rho\|_{L^1(\mathbb{R}^{d+1})} \lesssim \|u\|_{\mathcal B^1(\Omega)}.
\end{equation}

 
These two theorems include many popular activation functions, such as the rectified linear units \cite{nair2010rectified} and logistic sigmoid activation functions. Below we provide a table listing some well-known activation functions to which this theorem applies.
\begin{center}
\begin{tabular}{ |c|c|c|c|c| } 
 \hline
 Activation Function & $\sigma(x)$ & Maximal $m$ & $n_0$ & $\nu(x)$ \\
 \hline
 Sigmoidal (Logistic) & $(1 + e^{-x})^{-1}$ & $\infty$ & $2$ & $\sigma(x+1) - \sigma(x)$ \\
 \hline

 Arctan & $\arctan(x)$ & $\infty$ & $2$ & $\sigma(x+1) - \sigma(x)$ \\ 
 \hline
 Hyperbolic Tangent & $\tanh(x)$ & $\infty$ & $2$ & $\sigma(x+1) - \sigma(x)$ \\
 \hline
 SoftPlus \cite{glorot2011deep} & $\log(1 + e^x)$ & $\infty$ & $4$ & $\sigma(x+1) + \sigma(x - 1) - 2\sigma(x)$ \\
 \hline
 ReLU\cite{nair2010rectified} & $\max(0,x)$ & $1$ & $4$ & $\sigma(x+1) + \sigma(x - 1) - 2\sigma(x)$ \\
 \hline
  Leaky ReLU\cite{maas2013rectifier} & $\epsilon x + (1-\epsilon)\max(0,x)$ & $1$ & $4$ & $\sigma(x+1) + \sigma(x - 1) - 2\sigma(x)$ \\
 \hline
 $k$-th power of ReLU & $[\max(0,x)]^k$ & $k$ & $k+1$ & $\sum_{i=0}^k(-1)^i\binom{k}{i} \sigma(x - \lfloor k/2 \rfloor + i)$\\
 \hline
\end{tabular}
\end{center}

 
The bound depends on
the dimension $d$. We first note that $|\Omega|$ may in a sense depend
on the dimension, as the measure may be exponentially large in high
dimensions. However, bounding the $H^m$ error over a larger set is
also proportionally stronger. This can be seen by noting that dividing
by the $|\Omega|^\frac{1}{2}$ factor transforms the left hand side
from the total squared error to the average squared error.







\iffalse
\newpage

\subsection{Comparison with linear finite element method}
We can briefly have the next two asymptotic approximation results for deep neural networks and adaptive
linear finite element methods:
\begin{description}
	\item[Neural Network (NN)] 
	\begin{equation}\label{key}
	\inf_{f_n \in \dnn(\sigma,n)} \|f-f_n\|_{L^{2}(\Omega)} \lesssim  n^{-\frac{1}{2}}\|f\|_{\mathcal B^1(\Omega)},
	\end{equation}
	where $nd$ is the number of parameters.
	\item[Finite Element (FE)] 
	\begin{equation}\label{key}
	\inf_{f_n \in V_n} \|f-f_n\|_{L^{2}(\Omega)} \lesssim  n^{-\frac{2}{d}}\|f\|_{\ast},
	\end{equation}
	where
	\begin{equation}\label{key}
	V_n: \text{linear finite element space of}~ n-\text{elements},
	\end{equation}
	and $\|f\|_{\ast}$ is some Besov norm. 
\end{description}
A direct observation for the asymptotic approximation error is that :
\begin{equation}\label{key}
(\frac{n}{d})^{-\frac{1}{2}} << n^{-\frac{2}{d}},
\end{equation}
if $d >> 1$ with respect to the umber of parameters. However, in the future  we will show that 
\begin{equation}\label{key}
{\rm DNN}({\rm ReLU}) = \text{Linear FE},
\end{equation}
or we can say that DNN with ReLU activation function is a different way to parametrize linear 
finite element space.
\fi
