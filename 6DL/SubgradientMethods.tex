The goal of this chapter is to provide a comprehensive treatment of the most relevant
gradient and subgradient methods for convex optimization. 
We intend to cover
the most robust methods with the fewest assumptions first, and then later move to
methods which require more assumptions but have much better convergence rates.
%Since we will be most interested in stochastic methods for machine learning, it
%is the robust methods with few assumptions (but poor convergence rates) which will
%be most relevant.

\section{Convex Lipschitz Functions}
In this section, we consider methods for optimizing convex, lipschitz functions. This is a very large function
class, since we make no assumptions on the objective other than assuming bounded subgradients (which is
equivalent to the lipschitz property). In particular, we assume no smoothness, i.e. the objective need not
be differentiable, and no strong convexity.

Consequently, the convergence results we obtain are rather weak. They require averaging the iterates of our method
(or choosing the best one), i.e. we cannot simply take the last iterate, and the convergence rates obtained are
quite slow. However, these methods are robust. In particular, they obtain the same convergence rate
in the stochastic setting, as we will see.

In later sections we will detail precisely
how stronger assumptions on the objective yield better convergence results. We find it useful to
begin with the weakest assumptions and weakest convergence results first, and then to explain how everything relates to
these.

\subsection{Subgradient Descent}
The simplest version of subgradient descent (with a fixed step size) is the following iteration
\begin{equation}\label{subgradient_descent}
 x_{n+1} = x_n - sg_n
\end{equation}
where $g_n\in \partial f(x_n)$ (the subdifferential of a convex function $f$ to be optimized) and $s$ is the 
step size.

With a fixed step size, this method will not converge for an arbitrary choice of subgradient $g_n$. 
However, if we are given a tolerance beforehand, we can choose
a small enough step size to obtain an appropriate objective error, as the following well-known result shows.

\begin{theorem}\label{original_subgradient}
 Assume that $f$ is convex with bounded subgradient, meaning that $\|g\|_2 \leq M$ for all $g\in \partial f(x)$ (for any point $x$).
 
 Additionally, let $x^*\in \arg\min_x f(x)$ and assume that $\|x_1 - x^*\|_2 \leq R$ (note: here we are implicitly assuming that
 $f$ is bounded below and achieves its minimum). 
 
 Then, setting the step size $s = \frac{R}{M\sqrt{n}}$,
 the iterates of (\ref{subgradient_descent}) satisfy
 \begin{equation}
  f(\bar{x}_n) - f(x^*) \leq \frac{RM}{2\sqrt{n}}
 \end{equation}
 where $\bar{x}_n = \frac{1}{n}\sum_{i = 1}^n x_i$ is the average of the first $n$ iterates, and
 \begin{equation}
  \min_{i=1,...,n} f(x_i) - f(x^*) \leq \frac{RM}{2\sqrt{n}}
 \end{equation}


\end{theorem}
\begin{proof}
 Consider first $\bar{x}_n = \frac{1}{n}\sum_{i = 1}^n x_i$, the average of the iterates encountered so
 far. By convexity and the definition of the subgradient, we have
 \begin{equation}\label{eqn11}
  f(\bar{x}_n) - f(x^*) \leq \frac{1}{n}\displaystyle\sum_{i = 1}^n (f(x_i) - f(x^*))\leq \frac{1}{n}\displaystyle\sum_{i = 1}^n 
  \langle g_i, x_i - x^*\rangle
 \end{equation}
 From iteration (\ref{subgradient_descent}), we see that $g_i = \frac{1}{s}(x_i - x_{i+1})$. Plugging this into equation
 (\ref{eqn11}), we obtain
 \begin{equation}
  f(\bar{x}_n) - f(x^*) \leq \frac{1}{ns} \displaystyle\sum_{i = 1}^n \langle x_i - x_{i+1}, x_i - x^*\rangle
 \end{equation}
 Setting $y_i = x_i - x^*$, this sum becomes
 \begin{equation}
  \displaystyle\sum_{i = 1}^n \langle y_i - y_{i+1}, y_i\rangle = \frac{1}{2}\displaystyle\sum_{i = 1}^n
  \left(\|y_i\|_2^2 - \|y_{i+1}\|_2^2 + \|y_i - y_{i+1}\|_2^2\right)
 \end{equation}
 This sum telescopes, and we get (noting that $y_i - y_{i+1} = x_i - x_{i+1}$)
 \begin{equation}
  f(\bar{x}_n) - f(x^*) \leq \frac{1}{2ns}\left(\|y_1\|_2^2 - \|y_{n+1}\|_2^2 + \displaystyle\sum_{i = 1}^n \|x_i - x_{i+1}\|_2^2\right)
 \end{equation}
 Using the iteration (\ref{subgradient_descent}), we see (reversing our previous substitution) that
 $x_i - x_{i + 1} = sg_i$. Our assumption that the subgradient of $f$ is bounded implies that $\|g_i\|_2^2 \leq M^2$,
 so that
 $$\displaystyle\sum_{i = 1}^n \|x_i - x_{i+1}\|_2^2 = \displaystyle\sum_{i = 1}^n \|sg_i\|_2^2 \leq ns^2M^2
 $$
 and we obtain (dropping the negative $-\|y_{n+1}\|_2^2$)
 \begin{equation}
  f(\bar{x}_n) - f(x^*) \leq \frac{1}{2}\left(\frac{\|x_1 - x^*\|_2^2}{ns} + M^2s\right) \leq \frac{1}{2}\left(\frac{R^2}{ns} + M^2s\right)
 \end{equation}
 This bound is optimized when $s = \frac{R}{M\sqrt{n}}$ and we get
 \begin{equation}
  f(\bar{x}_n) - f(x^*) \leq \frac{RM}{2\sqrt{n}}
 \end{equation}
 For the second part of the theorem, we simply note that the minimum is smaller than the average, i.e. that
 \begin{equation}
  \min_{i = 1,...,n} f(x_i) - f(x^*) \leq \frac{1}{n}\displaystyle\sum_{i = 1}^n (f(x_i) - f(x^*))
 \end{equation}
 and repeat the same argument.

\end{proof}

It is a bit of a nuisance to choose the step size based on the total number of iterations, since it means that
we need to specify the total number of iterations in advance. For various reasons, it may be desirable to have a 
method which we can run for
an a priori unspecified number of steps.

This is possible if we allow ourselves a varying step size, i.e. an iteration
of the form
\begin{equation}\label{subgradient_descent_variable}
 x_{n+1} = x_n - s_ng_n
\end{equation}
As we show below, we can obtain the same convergence rate (up to logarithmic factors) without specifying the
number of iterations in advance.

\begin{theorem}\label{variable_subgradient_descent_thm}
 Assume that $f$ is convex with bounded subgradient, meaning that $\|g\|_2 \leq M$ for all $g\in \partial f(x)$ (for any point $x$).
 
 Additionally, let $x^*\in \arg\min_x f(x)$ and assume that $\|x_1 - x^*\|_2 \leq R$. 
 
 Then, setting the variable step size $s_i = \frac{R}{M\sqrt{i}}$,
 the iterates of (\ref{subgradient_descent_variable}) satisfy
 \begin{equation}
  f(\bar{x}_n) - f(x^*) \leq \frac{DM(\log(n) + 1)}{2\sqrt{n}}
 \end{equation}
 where $\bar{x}_n$ is the step size weighted average of the iterates, i.e.
 \begin{equation}
  \bar{x}_n = \left(\displaystyle\sum_{i = 1}^n s_i\right)^{-1} \displaystyle\sum_{i=1}^n s_ix_i
 \end{equation}
 Additionally, we get
 \begin{equation}
  \min_{i=1,...,n} f(x_i) - f(x^*) \leq \frac{DM(\log(n) + 1)}{2\sqrt{n}}
 \end{equation}

\end{theorem}
\begin{proof}
 We first consider the step size weighted average
 $$\bar{x}_n = \left(\displaystyle\sum_{i = 1}^n s_i\right)^{-1} \displaystyle\sum_{i=1}^n s_ix_i
 $$
 and note that by the convexity of $f$ (Jensen's inequality)
 \begin{equation}
  f(\bar{x}_n) - f(x^*) \leq D_n^{-1} \displaystyle\sum_{i=1}^n s_i(f(x_i) - f(x^*))
 \end{equation}
 where we have written $D_n = \sum_{i = 1}^n s_i$.
 
 The fact that $g_i\in \partial f(x_i)$ means that $f(x_i) - f(x^*) \leq \langle g_i, x_i - x^*\rangle$ so that
 \begin{equation}\label{eq22}
  f(\bar{x}_n) - f(x^*) \leq D_n^{-1} \displaystyle\sum_{i=1}^n s_i\langle g_i, x_i - x^*\rangle
 \end{equation}
 From the iteration (\ref{subgradient_descent_variable}) we have $s_ig_i = x_i - x_{i+1}$. Plugging this into
 (\ref{eq22}) we get
 \begin{equation}
  f(\bar{x}_n) - f(x^*) \leq D_n^{-1} \displaystyle\sum_{i=1}^n \langle x_i - x_{i+1}, x_i - x^*\rangle
 \end{equation}
 The sum on the right hand side of this inequality can now be rewritten in the same way as in the proof of Theorem
 \ref{original_subgradient}. The right hand side of the inequality becomes
 \begin{equation}
  \frac{1}{2D_n}\left(\|x_1 - x^*\|_2^2 - \|x_{n+1} - x^*\|_2^2 + \displaystyle\sum_{i=1}^n\|x_i - x_{i+1}\|_2^2\right)
 \end{equation}
 Dropping the negative term $-\|x_{n+1} - x^*\|_2^2$ we thus see that
 \begin{equation}
  f(\bar{x}_n) - f(x^*) \leq \frac{1}{2D_n}\left(\|x_1 - x^*\|_2^2 + \displaystyle\sum_{i=1}^n\|x_i - x_{i+1}\|_2^2\right)
 \end{equation}
 Using the iteration (\ref{subgradient_descent_variable}), we undo our previous substitution to obtain
 $x_i - x_{i+1} = s_ig_i$. This, combined with the bound on the subgradients of $f$, i.e. $\|g_i\|_2 \leq M$, implies
 \begin{equation}
  f(\bar{x}_n) - f(x^*) \leq \frac{1}{2D_n}\left(\|x_1 - x^*\|_2^2 + M^2\displaystyle\sum_{i=1}^n s_i^2\right)
 \end{equation}
 The assumption that $\|x_1 - x^*\|_2 \leq D$ gives us
 \begin{equation}
  f(\bar{x}_n) - f(x^*) \leq \frac{D^2 + M^2\displaystyle\sum_{i=1}^n s_i^2}{2\displaystyle\sum_{i=1}^n s_i}
 \end{equation}
 This bound is optimized when $s_i = \frac{D}{M\sqrt{i}}$ and we obtain
 \begin{equation}
  f(\bar{x}_n) - f(x^*) \leq \frac{DM\displaystyle\sum_{i=1}^n \frac{1}{i}}{2\displaystyle\sum_{i=1}^n \frac{1}{\sqrt{i}}}
 \end{equation}
 we now use the fact that $\displaystyle\sum_{i=1}^n \frac{1}{i} \leq \log(n) + 1$ and $\displaystyle\sum_{i=1}^n \frac{1}{\sqrt{i}} \geq \sqrt{n}$
 to get
 \begin{equation}
  f(\bar{x}_n) - f(x^*) \leq \frac{DM(\log(n) + 1)}{2\sqrt{n}}
 \end{equation}
 This proves the first result. As in the proof of Theorem \ref{original_subgradient}, we can use the fact that the minimum
 is bounded by the average to get
 \begin{equation}
  \min_{i=1,...,n} f(x_i) - f(x^*) \leq D_n^{-1} \displaystyle\sum_{i=1}^n s_i(f(x_i) - f(x^*))
 \end{equation}
 and so (by the exact same logic)
 \begin{equation}
   \min_{i=1,...,n} f(x_i) - f(x^*) \leq \frac{DM(\log(n) + 1)}{2\sqrt{n}}
 \end{equation}
 which is the second inequality. This completes the proof.

\end{proof}

We note again that these results are fairly weak. Notably, the convergence rates are very slow and we must take an average of
our iterates (or the minimum over the iterates) to obtain them. We will see later how stronger assumptions
on the function $f$ to be optimized result in better convergence results. 

\subsection{Stochastic Subgradient Descent}

A beneficial consequence of the weak assumptions we have made is that these results are very robust. 
Most notably, they remain true
if we replace the subgradient elements $g_n$ in iterations (\ref{subgradient_descent}) and (\ref{subgradient_descent_variable})
by unbiased samples of an element of the subgradient, i.e. we consider
\begin{equation}\label{random_subgradient_descent}
 x_{n+1} = x_n - sg(x_n,\xi_n)
\end{equation}
and
\begin{equation}\label{random_subgradient_descent_variable}
 x_{n+1} = x_n - s_ng(x_n,\xi_n)
\end{equation}
where the $\xi_n$ are independent random variables underlying the sampling process and the samples 
$g(x_n,\xi_n)$ satisfy $\mathbb{E}_{\xi_n}(g(x_n,\xi_n))\in \partial f(x_n)$. We must now assume that
the subgradient samples $g(x_n,\xi_n)$ are bounded. In particular, we have the
following results.
\begin{theorem}\label{random_subgradient_descent_thm}
 Assume that $f$ is convex and its subgradient samples are bounded, 
 meaning that $\|g(x,\xi)\|_2 \leq M$ for all $x$ and $\xi$.
 
 Additionally, let $x^*\in \arg\min_x f(x)$ and assume that $\|x_1 - x^*\|_2 \leq R$. 
 
 Then, setting the step size $s = \frac{R}{M\sqrt{n}}$,
 the iterates of (\ref{random_subgradient_descent}) satisfy
 \begin{equation}
  \mathbb{E}(f(\bar{x}_n) - f(x^*)) \leq \frac{RM}{2\sqrt{n}}
 \end{equation}
 where $\bar{x}_n = \frac{1}{n}\sum_{i = 1}^n x_i$ is the average of the first $n$ iterates, and
 \begin{equation}
  \mathbb{E}\left(\min_{i=1,...,n} f(x_i) - f(x^*)\right) \leq \frac{RM}{2\sqrt{n}}
 \end{equation}
 
\end{theorem}
\begin{proof}
 We consider first the average of the iterates. We have
 \begin{equation}
  \mathbb{E}(f(\bar{x}_n) - f(x^*)) \leq \mathbb{E}\left(\frac{1}{n}\displaystyle\sum_{i = 1}^n (f(x_i) - f(x^*))\right)
 \end{equation}
 by Jensen's inequality. By the linearity of the expectation and the fact that $x_i$ only depends on $\xi_1,...,\xi_{i-1}$
 we have
 \begin{align}
  \mathbb{E}\left(\frac{1}{n}\displaystyle\sum_{i = 1}^n (f(x_i) - f(x^*))\right)& = \frac{1}{n}\displaystyle\sum_{i = 1}^n \mathbb{E}(f(x_i) - f(x^*)) \\
   & = \frac{1}{n}\displaystyle\sum_{i = 1}^n \mathbb{E}_{\xi_1,...,\xi_{i-1}}(f(x_i) - f(x^*))
   \end{align}
 Now we use the assumption that $\mathbb{E}_{\xi_n}(h(x_n,\xi_n))\in \partial f(x_n)$ to get
 $$
  \frac{1}{n}\displaystyle\sum_{i = 1}^n \mathbb{E}_{\xi_1,...,\xi_{i-1}}(f(x_i) - f(x^*)) \leq \frac{1}{n}\displaystyle\sum_{i = 1}^n \mathbb{E}_{\xi_1,...,\xi_{i-1}}(\langle \mathbb{E}_{\xi_i}g(x_i,\xi_i),x_i - x^*\rangle)
 $$
 Since $x_i$ is independent of $\xi_i,\xi_{i+1},...,\xi_n$, we can pull the inner expectation out of the inner product and
 rewrite this as
 \begin{equation}
  \frac{1}{n}\displaystyle\sum_{i = 1}^n \mathbb{E}_{\xi_1,...,\xi_{i-1},\xi_i}(\langle g(x_i,\xi_i),x_i - x^*\rangle) = 
  \mathbb{E}\left(\frac{1}{n}\displaystyle\sum_{i = 1}^n \langle g(x_i,\xi_i),x_i - x^*\rangle\right)
 \end{equation}
 The iteration \ref{random_subgradient_descent} means that $g(x_i,\xi_i) = \frac{1}{s}(x_i - x_{i+1})$ and we obtain
 \begin{equation}
  \mathbb{E}(f(\bar{x}_n) - f(x^*)) \leq \mathbb{E}\left(\frac{1}{ns}\displaystyle\sum_{i = 1}^n \langle x_i - x_{i+1},x_i - x^*\rangle\right)
 \end{equation}
 We can now repeat the argument used in Theorem \ref{original_subgradient} to rewrite the right hand side as
 \begin{equation}
  \frac{1}{2ns}\mathbb{E}\left(\|x_1 - x^*\|^2_2 - \|x_{n+1} - x^*\|^2_2 + \displaystyle\sum_{i = 1}^n \|x_i - x_{i+1}\|_2^2\right)
 \end{equation}
 Now $\|x_i - x_{i+1}\|_2^2 = \|sg(x_i,\xi_i)\|_2^2 \leq s^2M^2$ and $\|x_1 - x^*\|_2^2 \leq R^2$ so we get
 \begin{equation}
  \mathbb{E}(f(\bar{x}_n) - f(x^*)) \leq \frac{1}{2}\left(\frac{R^2}{ns} + M^2s\right)
 \end{equation}
 Optimizing over $s$ now yields the same result as in Theorem \ref{original_subgradient}.

 For the second part, we again use the fact that the minimum is bounded by the average
 \begin{equation}
  \mathbb{E}\left(\min_{i=1,...,n} f(x_i) - f(x^*)\right) \leq \mathbb{E}\left(\frac{1}{n}\displaystyle\sum_{i = 1}^n (f(x_i) - f(x^*))\right)
 \end{equation}
 and the rest of the proof is the same as before.
 
\end{proof}

\begin{theorem}
 Assume that $f$ is convex and its subgradient samples are bounded, 
 meaning that $\|h(x,\xi)\|_2 \leq M$ for all $x$ and $\xi$.
 
 Additionally, let $x^*\in \arg\min_x f(x)$ and assume that $\|x_1 - x^*\|_2 \leq R$. 
 
 Then, setting the step size $s_i = \frac{R}{M\sqrt{i}}$,
 the iterates of (\ref{random_subgradient_descent_variable}) satisfy
 \begin{equation}
  \mathbb{E}(f(\bar{x}_n) - f(x^*)) \leq \frac{DM(\log(n) + 1)}{2\sqrt{n}}
 \end{equation}
 where $\bar{x}_n$ is the step size weighted average of the iterates, i.e.
 \begin{equation}
  \bar{x}_n = \left(\displaystyle\sum_{i = 1}^n s_i\right)^{-1} \displaystyle\sum_{i=1}^n s_ix_i
 \end{equation}
 Additionally, we get
 \begin{equation}
  \mathbb{E}\left(\min_{i=1,...,n} f(x_i) - f(x^*)\right) \leq \frac{DM(\log(n) + 1)}{2\sqrt{n}}
 \end{equation}
 
\end{theorem}
\begin{proof}
 The proof is a straightforward combination of the arguments in Theorems \ref{random_subgradient_descent_thm} and
 \ref{variable_subgradient_descent_thm}.
\end{proof}

\subsection{Forward-Backward Subgradient Descent}
Finally, we cover a very useful and interesting variation on subgradient descent, which we call
forward-backward subgradient descent. 

The relevant setting is an optimization problem whose objective can be
split as a sum of two objectives
\begin{equation}\label{split_problem}
 f(x) = f_1(x) + f_2(x)
\end{equation}
where $f_1$ is a function for which we can easily find elements in the subgradient and $f_2$ is a function for
which we can efficiently solve the `backward' or `proximal' problem
\begin{equation}
 p_{\lambda f_2}(x) = \arg\min_{y} \frac{1}{2}\|y - x\|_2^2 + \lambda f_2(y)
\end{equation}
Notice that optimality condition that the solution $y$ satisfies is
$$0\in \partial_y\left(\frac{1}{2}\|y - x\|_2^2 + \lambda f_2(y)\right) = y - x + \lambda\partial f_2(y)
$$
In other words, $y = x - \lambda g$ for some $g\in \partial f_2(y)$. Thus solving the proximal problem
essentially corresponds to a step of backward Euler for the equation $\dot{x} \in \partial f_2(x)$ in the
same sense that a (forward) subgradient step corresponds to a step of forward Euler for the same equation.

Two very common choices for $f_2$ are the indicator function of a convex set $A$
\begin{equation}
 f_2(x) = i_A(x) = 
 \begin{cases}
                    0 &~ \text{if $x\in A$} \\
		    +\infty &~ \text{if $x\notin A$}
 \end{cases}
\end{equation}
and the l$1$-norm $f_2(x) = \|x\|_1$. In the former case, the proximal map is just a projection onto the set $A$ and
the forward-backward method simply recovers projected subgradient descent. In the latter case, the
proximal map is known as soft-thresholding and can be given in closed form.

The method we will analyze for optimizing (\ref{split_problem}) is the `forward-backward' splitting method
\begin{equation}\label{subgradient_forward_backward}
 x_{n+\frac{1}{2}} = x_n - sg_n,~x_{n+1} = p_{sf_2}(x_{n+\frac{1}{2}})
\end{equation}
where $g_n\in \partial f_1(x_n)$. This is just a forward step for $f_1$ followed by a backward step for
$f_2$.

To aid in our analysis of this algorithm we introduce the Bregman distance, defined as follows
\begin{definition}
 Let $f$ be a l.s.c convex function, $x, y\in \mathbb{R}^n$, and $g\in \partial f(x)$ (which implies that
 $f(x) < \infty$). The $f$-Bregman distance from $x$ to $y$ with respect to $g$ is
 \begin{equation}
  B_f(y,x;g) = f(y) - f(x) - \langle g, y - x\rangle
 \end{equation}

\end{definition}\label{bregman_distance}
The Bregman distance is simply the difference between $f(y)$ and the linear approximation
$f(x) + \langle g, y - x\rangle$. The convexity of $f$ means that this quantity is always positive.

One can check that if $f = \frac{1}{2}\|x\|_2^2$, then the Bregman distance corresponds to $\frac{1}{2}\|x - y\|_2^2$,
for instance. In general, symmetry is not guaranteed though.

We will need the following lemma regarding the Bregman distance
\begin{lemma}\label{bregman_equation_1}
 Let $x,y,z\in \mathbb{R}^n$, $g_x\in \partial f(x)$, $g_y\in \partial f(y)$. Then
 \begin{equation}
  \langle g_x - g_y, x - z\rangle = B_f(z,x;g_x) - B_f(z,y;g_y) + B_f(x,y;g_y)
 \end{equation}
 
\end{lemma}

\begin{proof}
 From the definition \ref{bregman_distance}, we see that
  \begin{align}
   &B_f(z,x;g_x) = f(z) - f(x) - \langle g_x, z - x\rangle \\
   &-B_f(z,y;g_y) = -f(z) + f(y) + \langle g_y, z - y\rangle \\
   & B_f(x,y;g_y) = f(x) - f(y) - \langle g_y, x - y\rangle
  \end{align}

  Adding these quantities, we obtain
  \begin{equation}
   \langle g_y, z - y\rangle - \langle g_y, x - y\rangle - \langle g_x, z - x\rangle = \langle g_x - g_y, x - z\rangle
  \end{equation}
  as desired.
\end{proof}

We will now prove the following convergence result for the forward-backward iteration (\ref{subgradient_forward_backward}). 
%has similar convergence
%properties as subgradient descent (\ref{subgradient_descent}).

\begin{theorem}\label{forward_backward_subgradient_thm}
 Assume that $f_1$ is convex with bounded subgradient, meaning that $\|g\|_2 \leq M$ for all $g\in \partial f_1(x)$ (for any point $x$).
 
 Set $x_1 \in \arg\min_xf_2(x)$, let $x^*\in \arg\min_x f(x)$, and assume that $\|x_1 - x^*\|_2 \leq R$. 
 
 Then, setting the step size $s = \frac{R}{M\sqrt{n}}$,
 the iterates of (\ref{subgradient_forward_backward}) satisfy
 \begin{equation}
  f(\bar{x}_n) - f(x^*) \leq \frac{RM}{2\sqrt{n}} + \frac{f_2(x^*) - \min_x f_2(x)}{n}
 \end{equation}
 where $\bar{x}_n = \frac{1}{n}\sum_{i = 1}^n x_i$ is the average of the first $n$ iterates, and
 \begin{equation}
  \min_{i=1,...,n} f(x_i) - f(x^*) \leq \frac{RM}{2\sqrt{n}} + \frac{f_2(x^*) - \min_x f_2(x)}{n}
 \end{equation}

\end{theorem}
The whole point of this theorem is that the convergence rate only depends on the Lipschitz constant
of $f_1$ and not on the Lipschitz constant of $f_2$ (in fact, we don't need $f_2$ to be Lipschitz at all).
%In practice this method is very useful if $f_1$ has a small Lipschitz constant and $f_2$ isn't Lipschitz
%of has a very large Lipshchitz constant.

\begin{proof}
 We consider first the average $\bar{x}_n = \frac{1}{n}\sum_{i = 1}^n x_i$ of the first $n$ iterates.
 We have by Jensen's inequality that
 \begin{equation}
  f(\bar{x}_n) - f(x^*) \leq \frac{1}{n}\displaystyle\sum_{i = 1}^n (f(x_i) - f(x^*))
 \end{equation}
 The forward iteration gives us 
 $$\frac{1}{s}(x_i - x_{i + \frac{1}{2}}) \in \partial f_1(x_i)$$ 
 and the backward iteration means that 
 $$\frac{1}{s}(x_{i - \frac{1}{2}} - x_i)\in\partial f_2(x_i)$$ 
 (for $i = 1$ we set $x_{\frac{1}{2}} = x_1$ and
 use that $x_1\in \arg\min f_2$ so that $0\in \partial f_2(x_1)$).
 
 This implies that
 \begin{equation}\label{eq373}
  \frac{1}{n}\displaystyle\sum_{i = 1}^n (f(x_i) - f(x^*)) \leq \frac{1}{ns}\displaystyle\sum_{i = 1}^n \langle x_{i - \frac{1}{2}} - x_{i + \frac{1}{2}}, x_i - x^*\rangle
 \end{equation}
 We rewrite the right-hand side as
 \begin{equation}\label{eq377}
  \frac{1}{ns}\displaystyle\sum_{i = 1}^n \langle x_i - x_{i + 1}, x_i - x^*\rangle + \frac{1}{n} \displaystyle\sum_{i = 1}^n\langle h_i - h_{i+1}, x_i - x^*\rangle
 \end{equation}
 where $h_i = \frac{1}{s}(x_{i - \frac{1}{2}} - x_i)\in\partial f_2(x_i)$.
 
 The first sum in equation (\ref{eq377}) can be rewritten in exactly the same way as in the proof of Theorem \ref{original_subgradient}
 to obtain
 \begin{equation}\label{eq384}
  \frac{1}{2ns}\left(\|x_1 - x^*\|_2^2 - \|x_{n+1} - x^*\|_2^2 + \displaystyle\sum_{i = 1}^n \|x_i - x_{i+1}\|_2^2\right)
 \end{equation}

 We now consider the second sum in equation (\ref{eq377}). For this we use Lemma \ref{bregman_equation_1} to rewrite
 $\langle h_i - h_{i+1}, x_i - x^*\rangle$ as
 \begin{equation}
  D_{f_2}(x^*,x_i;h_i) - D_{f_2}(x^*,x_{i+1};h_{i+1}) + D_{f_2}(x_i,x_{i+1};h_{i+1})
 \end{equation}
 and note that the relevant sum now telescopes to yield
 \begin{equation}\label{eq394}
  \frac{1}{n}\left(D_{f_2}(x^*,x_1;h_1) - D_{f_2}(x^*,x_{n+1};h_{n+1}) + 
  \displaystyle\sum_{i = 1}^n D_{f_2}(x_i,x_{i+1};h_{i+1})\right)
 \end{equation}
 We continue by considering the sums in equations (\ref{eq384}) and (\ref{eq394}) together
 \begin{equation}\label{eq402}
  \frac{1}{ns}\left(\displaystyle\sum_{i = 1}^n \frac{1}{2}\|x_i - x_{i+1}\|_2^2 + sD_{f_2}(x_i,x_{i+1};h_{i+1})\right)
 \end{equation}
 and consider the term
 $$\frac{1}{2}\|x_i - x_{i+1}\|_2^2 + sD_{f_2}(x_i,x_{i+1};h_{i+1})
 $$
 Recalling the definition of the Bregman distance and the fact that $x_i - x_{i+1} = sg_i + sh_{i+1}$ (recall that
 $g_i\in\partial f_1(x_i)$ is the subgradient in the forward step), this term becomes
 $$\frac{1}{2}\|sg_i + sh_{i+1}\|_2^2 - \langle sh_{i+1}, sg_i + sh_{i+1}\rangle + s(f_2(x_i) - f_2(x_{i+1})) 
 $$
 which is equal to
 \begin{equation}
  \frac{1}{2}(\|sg_i\|_2^2 - \|sh_{i + 1}\|_2^2) + s(f_2(x_i) - f_2(x_{i+1})) 
 \end{equation}
 Plugging this into equation (\ref{eq402}) we get
 \begin{equation}
  \frac{1}{n}(f_2(x_1) - f_2(x_{n+1})) + \frac{1}{2ns}\left(\displaystyle\sum_{i = 1}^n\|sg_i\|_2^2 - \|sh_{i + 1}\|_2^2\right)
 \end{equation}
 Plugging all of this into equation (\ref{eq373}), noting that $f_2(x_1) - f_2(x_{n+1}) \leq 0$ (since $x_1\in \arg\min f_2$),
 dropping all of the negative norm squared terms, and recalling that $D_{f_2}(x^*,x_{n+1};h_{n+1}) \geq 0$, we obtain
 \begin{equation}
  \frac{1}{n}\displaystyle\sum_{i = 1}^n (f(x_i) - f(x^*)) \leq \frac{1}{2ns}\left(\|x_1 - x^*\|_2^2 + \displaystyle\sum_{i = 1}^n\|sg_i\|_2^2\right)
  + \frac{1}{n}D_{f_2}(x^*,x_1;h_1)
 \end{equation}
 The first term in this sum is the same as the bound in the proof of Theorem \ref{original_subgradient}, and by choosing $s = \frac{R}{M\sqrt{n}}$
 we obtain as before
 \begin{equation}
  \frac{1}{2ns}\left(\|x_1 - x^*\|_2^2 + \displaystyle\sum_{i = 1}^n\|sg_i\|_2^2\right) \leq \frac{RM}{2\sqrt{n}}
 \end{equation}
 Finally, we note that since we set $x_{\frac{1}{2}} = x_1$, we have $h_1 = 0$ and so
 $$D_{f_2}(x^*,x_1;h_1) = f_2(x^*) - f_2(x_1) = f_2(x^*) - \min_x f_2(x)
 $$
 This then gives
 \begin{equation}
  f(\bar{x}_n) - f(x^*) \leq \frac{RM}{2\sqrt{n}} + \frac{f_2(x^*) - \min_x f_2(x)}{n}
 \end{equation}
 as desired.

 As usual, we deal with the minimum by noting that the minimum is bounded by the average.
\end{proof}

We can also obtain a version of this method with a variable step size, for which the number of iterations
need not be specified in advance, we leave this as an exercise.

In addition, we obtain the same result in the stochastic setting, where the subgradient elements $g_n$ are replaced by an unbiased 
sample, i.e. where we consider the iteration
\begin{equation}\label{forward_backward_stochastic_subgradient}
 x_{n+\frac{1}{2}} = x_n - sg(x_n,\xi_n),~x_{n+1} = p_{sf_2}(x_{n+\frac{1}{2}})
\end{equation}
where $\mathbb{E}_{\xi_n}(g(x_n,\xi_n))\in \partial f_1(x_n)$.
\begin{theorem}
 Assume that $f_1$ is convex with bounded subgradient samples, meaning that $\|g(x,\xi)\|_2 \leq M$ for all $x$ and $\xi$.
 
 Set $x_1 \in \arg\min_xf_2(x)$, let $x^*\in \arg\min_x f(x)$, and assume that $\|x_1 - x^*\|_2 \leq R$. 
 
 Then, setting the step size $s = \frac{R}{M\sqrt{n}}$,
 the iterates of (\ref{forward_backward_stochastic_subgradient}) satisfy
 \begin{equation}
  \mathbb{E}(f(\bar{x}_n) - f(x^*)) \leq \frac{RM}{2\sqrt{n}} + \frac{f_2(x^*) - \min_x f_2(x)}{n}
 \end{equation}
 where $\bar{x}_n = \frac{1}{n}\sum_{i = 1}^n x_i$ is the average of the first $n$ iterates, and
 \begin{equation}
  \mathbb{E}(\min_{i=1,...,n} f(x_i) - f(x^*)) \leq \frac{RM}{2\sqrt{n}} + \frac{f_2(x^*) - \min_x f_2(x)}{n}
 \end{equation}
\end{theorem}
\begin{proof}
  The proof is a straightforward combination of the arguments in Theorems \ref{random_subgradient_descent_thm} and
 \ref{forward_backward_subgradient_thm}.
\end{proof}

\subsection{Dual Averaging}
In this section, we introduce Nesterov's method of dual averaging. In order to best explain where this method comes from,
we first start with a different perspective on subgradient descent. Notice that the subgradient descent iteration
(\ref{subgradient_descent}) can also be written as
\begin{equation}\label{subgradient_descent_argmin}
 x_{n+1} = \arg\min_x \left(f(x_n) + \langle g_n, x - x_n\rangle + \frac{1}{2s}\|x - x_n\|_2^2\right)
\end{equation}
Notice that $f(x_n) + \langle g_n, x - x_n\rangle$ is a lower bound on the objective $f(x)$, and 
$\frac{1}{2s}\|x - x_n\|_2^2$ can be thought of as a regularization term which prevents $x_{n+1}$
from moving too far away from $x_n$.

So for subgradient descent, the next iterate can be obtained by minimizing a regularized lower bound on the objective.
In fact, many optimization methods can be thought of as minimizing regularized approximations to the objective in 
every step.

In equation (\ref{subgradient_descent_argmin}), the lower bound $f(x_n) + \langle g_n, x - x_n\rangle$ only depends on
the previous point $x_n$ and the previous gradient $g_n$. The idea behind dual averaging is to replace this
lower bound by its average over all previous iterates
\begin{equation}
 \frac{1}{n}\displaystyle\sum_{i = 1}^n (f(x_i) + \langle g_i, x - x_i\rangle)
\end{equation}
Since each of these terms is a lower bound on $f$, so is their average. We now need to add a regularization term
to obtain the next iterate $x_{n+1}$. For the purposes of this exposition, we choose $\frac{1}{2}\|x\|^2_2$ as a regularizer,
although any strongly convex function can be used. This leads to the following subproblem for $x_{n+1}$
\begin{equation}
 x_{n+1} = \arg\min_x\left(\frac{1}{n}\displaystyle\sum_{i = 1}^n (f(x_i) + \langle g_i, x - x_i\rangle) + \frac{\gamma_n}{2}\|x\|_2^2\right)
\end{equation}
where $g_i\in \partial f(x_i)$ and the sequence $\gamma_n > 0$ must be chosen appropriately to make the method converge. 

In order to simplify notation, we drop terms not depending on $x$, multiply by $n$, and define $\alpha_{n+1} = n\gamma_n$,
to rewrite the above problem as
\begin{equation}\label{SDA_iteration}
 x_{n+1} = \arg\min_x\left(\displaystyle\sum_{i = 1}^n \langle g_i, x\rangle + \frac{\alpha_{n+1}}{2}\|x\|_2^2\right)
\end{equation}
(again $g_i\in \partial f(x_i)$) whose solution is given by
\begin{equation}
 x_{n+1} = -\frac{1}{\alpha_{n+1}}\displaystyle\sum_{i = 1}^n g_i
\end{equation}
Additionally, we set $x_1 = 0$ and $\alpha_1 > 0$ so that
\begin{equation}
 x_1 = \arg\min_x \frac{\alpha_1}{2}\|x\|_2^2
\end{equation}


We begin by using the definition of $x_{n+1}$ as the minimizer in (\ref{SDA_iteration}) to obtain two simple lemmas which will
be crucial in our convergence analysis.
\begin{lemma}\label{lemma558}
 For any $z$ and any $n \geq 1$, we have
 \begin{equation}
  \displaystyle\sum_{i = 1}^{n-1} \langle g_i, x_n - z\rangle \leq \frac{\alpha_n}{2}(\|z\|_2^2 - \|x_n\|_2^2)
 \end{equation}

\end{lemma}
\begin{proof}
 Since $x_{n+1}$ is the minimizer in (\ref{SDA_iteration}), we have
 \begin{equation}
  \displaystyle\sum_{i = 1}^n \langle g_i, x_{n+1}\rangle + \frac{\alpha_{n+1}}{2}\|x_{n+1}\|_2^2 \leq \displaystyle\sum_{i = 1}^n \langle g_i, z\rangle + \frac{\alpha_{n+1}}{2}\|z\|_2^2
 \end{equation}
 Rearranging this and shifting the index yields the desired inequality (we can also easily verify the conclusion for $n = 1$).
\end{proof}

\begin{lemma}\label{lemma573}
 Let $x_n$ be defined by (\ref{SDA_iteration}). Then we have
 \begin{equation}
  \displaystyle\sum_{i = 1}^n \langle g_i, x_n - x_{n+1}\rangle \leq \frac{\alpha_{n}}{2}(\|x_{n+1}\|_2^2 - \|x_n\|_2^2) + \frac{1}{2\alpha_{n}} \|g_n\|_2^2
 \end{equation}

\end{lemma}
\begin{proof}
 Define $f_n(x) = \sum_{i = 1}^{n-1} \langle g_i, x\rangle + \frac{\alpha_{n}}{2}\|x\|_2^2$, so that
 $$x_n = \arg\min_x f_n(x)
 $$
 and $f_n(x)$ is strongly convex with convexity parameter $\alpha_{n}$.
 
 This means that $f^\prime_n(x) = f_n(x) + \langle g_n, x\rangle$ is also strongly convex with the same convexity parameter $\alpha_{n}$.
 
 Note that since $x_n = \arg\min_x f_n(x)$, we have $0\in \partial f_n(x_n)$ and so $g_n\in \partial f^\prime_n(x_n)$. Using the strong
 convexity, we thus obtain
 \begin{equation}
  f_n^\prime(x_{n+1}) \geq f_n^\prime(x_n) + \langle g_n, x_{n+1} - x_n\rangle + \frac{\alpha_{n}}{2}\|x_{n+1} - x_n\|_2^2
 \end{equation}
 so that (rearranging and using Cauchy-Schwartz)
 \begin{align}
  f_n^\prime(x_n) - f_n^\prime(x_{n+1}) &\leq \|g_n\|_2\|x_{n+1} - x_n\|_2 - \frac{\alpha_{n}}{2}\|x_{n+1} - x_n\|_2^2 \\
  &\leq \frac{1}{2\alpha_{n}} \|g_n\|_2^2
 \end{align}
 This means that
 \begin{equation}
  f_n^\prime(x_n) \leq f_n^\prime(x_{n+1}) + \frac{1}{2\alpha_{n}} \|g_n\|_2^2
 \end{equation}
 We now recall the definition of $f^\prime_n(x)$ to see that
 \begin{equation}
  \displaystyle\sum_{i = 1}^{n} \langle g_i, x_n\rangle + \frac{\alpha_{n}}{2}\|x_n\|_2^2 \leq 
  \displaystyle\sum_{i = 1}^{n} \langle g_i, x_{n+1}\rangle + \frac{\alpha_{n}}{2}\|x_{n+1}\|_2^2 + \frac{1}{2\alpha_{n}} \|g_n\|_2^2
 \end{equation}
 Rearranging this, we get
 \begin{equation}
  \displaystyle\sum_{i = 1}^{n} \langle g_i, x_n - x_{n+1}\rangle \leq \frac{\alpha_{n}}{2}(\|x_{n+1}\|_2^2 - \|x_n\|_2^2) + \frac{1}{2\alpha_{n}} \|g_n\|_2^2
 \end{equation}
 as desired.

\end{proof}

We now prove convergence of the dual averaging method.
\begin{theorem}\label{SDA_convergence_thm}
 Assume that $f$ is convex with bounded subgradient (i.e. Lipschitz), meaning that $\|g\|_2 \leq M$ for all $g\in \partial f(x)$ (for any point $x$).
 
 Additionally, let $x^*\in \arg\min_x f(x)$.
 
 Then if $x_n$ is given by the iteration (\ref{SDA_iteration}) with $\alpha_n = c\sqrt{n}$, we have
 \begin{equation}
  f(\bar{x}_n) - f(x^*) \leq \frac{1}{\sqrt{n}}\left(\frac{c}{2}\|x^*\|_2^2 + \frac{M^2}{c}\right)
 \end{equation}
 where $\bar{x}_n$ is the average of the first $n$ iterates. Additionally, we have
 \begin{equation}
  \min_{i = 1,...,n} (f(x_i) - f(x^*)) \leq \frac{1}{\sqrt{n}}\left(\frac{c}{2}\|x^*\|_2^2 + \frac{M^2}{c}\right)
 \end{equation}
\end{theorem}

\begin{proof}
 As before, we begin by noting that (by convexity)
 \begin{equation}
  f(\bar{x}_n) - f(x^*) \leq \frac{1}{n}\displaystyle\sum_{i = 1}^n (f(x_i) - f(x^*))
 \end{equation}
 and
 \begin{equation}
  \min_{i = 1,...,n} (f(x_i) - f(x^*)) \leq \frac{1}{n}\displaystyle\sum_{i = 1}^n (f(x_i) - f(x^*))
 \end{equation}
 Thus it suffices to bound the `average regret.' We begin by using the subdifferential property to get
 \begin{equation}
  \frac{1}{n}\displaystyle\sum_{i = 1}^n (f(x_i) - f(x^*)) \leq 
  \frac{1}{n}\displaystyle\sum_{i = 1}^n \langle g_i, x_i - x^*\rangle
 \end{equation}
 The trick is now to rewrite this sum as
 \begin{equation}\label{eq646}
  \displaystyle\sum_{i = 1}^n \langle g_i, x_i - x^*\rangle = 
  \displaystyle\sum_{i = 1}^n \langle g_i, x_{n+1} - x^*\rangle + \displaystyle\sum_{i = 1}^n \displaystyle\sum_{j = 1}^i\langle g_j, x_i - x_{i+1}\rangle
 \end{equation}
 and to use Lemmas \ref{lemma573} and \ref{lemma558} to bound
 \begin{equation}
  \displaystyle\sum_{i = 1}^n \langle g_i, x_{n+1} - x^*\rangle \leq \frac{\alpha_{n+1}}{2}(\|x^*\|_2^2 - \|x_{n+1}\|_2^2)
 \end{equation}
 and
 \begin{equation}
  \displaystyle\sum_{j = 1}^i\langle g_j, x_i - x_{i+1}\rangle \leq \frac{\alpha_{i}}{2}(\|x_{i+1}\|_2^2 - \|x_i\|_2^2) + \frac{1}{2\alpha_{i}} \|g_i\|_2^2
 \end{equation}
 Utilizing our bound on the subgradients $\|g_i\|_2\leq M$, plugging this into equation (\ref{eq646}), and rearranging
 the (almost) telescoping sum, we get
 \begin{align}
  \displaystyle\sum_{i = 1}^n \langle g_i, x_i - x^*\rangle &\leq \frac{\alpha_{n+1}}{2}\|x^*\|_2^2 - \frac{\alpha_1}{2}\|x_1\|_2^2 \\
  &+ \frac{1}{2}\displaystyle\sum_{i = 2}^{n+1}(\alpha_{i - 1} - \alpha_{i}) \|x_i\|_2^2 + \frac{M^2}{2}\displaystyle\sum_{i = 1}^n\frac{1}{\alpha_{i}} 
 \end{align}
 The above inequality is true for any choice of $\alpha_1,...,\alpha_{n+1}$ as long as the $x_i$ are given by the iteration
 (\ref{SDA_iteration}). Since the left hand side of the inequality only depends upon $x_1,...,x_n$, we are free to choose
 $\alpha_{n+1} = \alpha_n$ (without changing the left hand size), and we get
 \begin{align}
  \displaystyle\sum_{i = 1}^n \langle g_i, x_i - x^*\rangle &\leq \frac{\alpha_{n}}{2}\|x^*\|_2^2 - \frac{\alpha_1}{2}\|x_1\|_2^2 \\
  &+ \frac{1}{2}\displaystyle\sum_{i = 2}^{n}(\alpha_{i - 1} - \alpha_{i}) \|x_i\|_2^2 + \frac{M^2}{2}\displaystyle\sum_{i = 1}^n\frac{1}{\alpha_{i}} 
 \end{align}
 Now we note that if $\alpha_n$ is an increasing sequence (which it is for our particular choice $\alpha_n = c\sqrt{n}$), we can use 
 $\alpha_{i-1} - \alpha_i \leq 0$ to get
 \begin{equation}
  \displaystyle\sum_{i = 1}^n \langle g_i, x_i - x^*\rangle \leq\frac{\alpha_{n}}{2}\|x^*\|_2^2 + \frac{M^2}{2}\displaystyle\sum_{i = 1}^n\frac{1}{\alpha_{i}} 
 \end{equation}
 Plugging in our choice of $\alpha_n = c\sqrt{n}$ and noting that $\sum_{i = 1}^n \frac{1}{\sqrt{i}} \leq 2\sqrt{n}$ gives
 \begin{equation}
  \displaystyle\sum_{i = 1}^n \langle g_i, x_i - x^*\rangle \leq\left(\frac{c}{2}\|x^*\|_2^2 + \frac{M^2}{c}\right)\sqrt{n}
 \end{equation}
 Dividing this by $n$ completes the proof.

\end{proof}

So the dual averaging method achieves the same asymptotic convergence rate as subgradient descent. A straightforward modification
of the above argument can be used to obtain the convergence of a stochastic version of the dual average algorithm. As before,
we need to replace the Lipschitz assumption on $f$ by the assumption that our subgradient samples are bounded. We state the
relevant result without proof.

\begin{theorem}
 Assume that $f$ is convex and assume that $x_n$ are given by the iteration (\ref{SDA_iteration}) where $g_i$ is replaced by
 an unbiased sample $g(x_i,\xi_i)$ of an element in the subgradient 
 (i.e. $\xi_i$ is a random variable and $\mathbb{E}_{\xi_i}g(x_i,\xi_i) \in \partial f(x_i)$). Assume additionally that the 
 subgradient samples are bounded, i.e. that $\|g(x,\xi)\|_2 \leq M$.
 
 Let $x^*\in \arg\min_x f(x)$.
 
 Then with $\alpha_n = c\sqrt{n}$, we have
 \begin{equation}
  \mathbb{E}(f(\bar{x}_n) - f(x^*)) \leq \frac{1}{\sqrt{n}}\left(\frac{c}{2}\|x^*\|_2^2 + \frac{M^2}{c}\right)
 \end{equation}
 where $\bar{x}_n$ is the average of the first $n$ iterates. Additionally, we have
 \begin{equation}
  \mathbb{E}\left(\min_{i = 1,...,n} (f(x_i) - f(x^*))\right) \leq \frac{1}{\sqrt{n}}\left(\frac{c}{2}\|x^*\|_2^2 + \frac{M^2}{c}\right)
 \end{equation}
\end{theorem}

\subsection{Regularized Dual Averaging}

In this subsection, we introduce a version of the dual averaging method which is designed for solving composite
optimization problems of the form
\begin{equation}
 f(x) = f_1(x) + f_2(x)
\end{equation}
where $f_1(x)$ is a convex function whose subgradient can be readily calculated and $f_2(x)$ is a convex function
for which a proximal (or backward) step
\begin{equation}
 p_{\lambda f_2}(x) = \arg\min_y \frac{1}{2}\|y - x\|_2^2 + \lambda f_2(y)
\end{equation}
can efficiently be calculated. The proximal step and functions $f_2$ for which it can efficiently be evaluated were
discussed in more detail in the section on forward backward subgradient descent.

We begin the discussion by noting that forward-backward subgradient descent (\ref{subgradient_forward_backward}) can be written
as
\begin{equation}
 x_{n+1} = \arg\min_x \left(f_1(x_n) + \langle g_n, x - x_n\rangle + f_2(x) + \frac{1}{2s}\|x - x_n\|_2^2\right)
\end{equation}
where $g_n\in \partial f(x_n)$. We note that $f_1(x_n) + \langle g_n, x - x_n\rangle + f_2(x)$ is a lower bound on
$f(x)$ and thus this method can be thought of as minimizing a regularized lower bound on the objective.

The regularized dual average method replaces the above lower bound by an average over all of the previous iterates
\begin{equation}
 \frac{1}{n}\displaystyle\sum_{i = 1}^n (f_1(x_n) + \langle g_n, x - x_n\rangle + f_2(x))
\end{equation}
and as for the dual averaging method, we add $\frac{1}{2}\|x\|_2^2$ as a regularizer (although any strongly convex
function will do) to obtain the following method
\begin{equation}
 x_{n+1} = \arg\min_x\left(\frac{1}{n}\displaystyle\sum_{i = 1}^n (f_1(x_n) + \langle g_n, x - x_n\rangle + f_2(x)) + \frac{\gamma_n}{2}\|x\|_2^2\right)
\end{equation}
In order to simplify notation, we rewrite this in the same way we rewrote the dual averaging method to get
\begin{equation}\label{RDA_iteration}
 x_{n+1} = \arg\min_x \left(\displaystyle\sum_{i = 1}^n \langle g_n, x\rangle\right) + nf_2(x) + \frac{\alpha_{n+1}}{2}\|x\|_2^2
\end{equation}
where $\alpha_{n+1} = n\gamma_n$. More explicitly, we can split this subproblem into two steps to get
\begin{equation}
 x_{n+\frac{1}{2}} = -\frac{1}{\alpha_{n+1}}\displaystyle\sum_{i = 1}^n g_n,~x_{n+1} = p_{\lambda_nf_2}(x_{n+\frac{1}{2}})
\end{equation}
where $\lambda_n = \frac{n}{\alpha_{n+1}}$. Also, we set $x_1 = 0$ and $\alpha_1 > 0$ so that
\begin{equation}
 x_1 = \arg\min_x \frac{\alpha_1}{2}\|x\|_2^2
\end{equation}

The convergence analysis of this method is very similar to that of the dual averaging method. We begin by using the
definition of $x_{n+1}$ as the minimizer in (\ref{RDA_iteration}) to obtain the following two lemmas.

\begin{lemma}\label{lemma751}
 For any $z$ and any $n \geq 1$, we have
 \begin{equation}
  \displaystyle\sum_{i = 1}^{n-1} \langle g_i, x_n - z\rangle \leq \frac{\alpha_n}{2}(\|z\|_2^2 - \|x_n\|_2^2) + (n-1)(f_2(z) - f_2(x_n))
 \end{equation}

\end{lemma}
\begin{proof}
 Since $x_{n+1}$ is the minimizer in (\ref{RDA_iteration}), we have
 \begin{align}
  &\displaystyle\sum_{i = 1}^n \langle g_i, x_{n+1}\rangle + \frac{\alpha_{n+1}}{2}\|x_{n+1}\|_2^2 + nf_2(x_{n+1}) \\
  &\leq \displaystyle\sum_{i = 1}^n \langle g_i, z\rangle + \frac{\alpha_{n+1}}{2}\|z\|_2^2+ nf_2(z)
 \end{align}
 Rearranging this and shifting the index yields the desired inequality (we can also easily verify the conclusion for $n = 1$).
\end{proof}

\begin{lemma}\label{lemma767}
 Let $x_n$ be defined by (\ref{SDA_iteration}). Then we have
 \begin{align}
  \displaystyle\sum_{i = 1}^n \langle g_i, x_n - x_{n+1}\rangle &\leq \frac{\alpha_{n}}{2}(\|x_{n+1}\|_2^2 - \|x_n\|_2^2) \\
  &+ (n-1)(f_2(x_{n+1}) - f_2(x_n)) + \frac{1}{2\alpha_{n}} \|g_n\|_2^2
 \end{align}

\end{lemma}
\begin{proof}
 The proof is exactly the same as the proof of Lemma \ref{lemma573}, except that we set
 \begin{equation}
  f_n(x) = \left(\displaystyle\sum_{i = 1}^{n-1} \langle g_n, x\rangle\right) + (n-1)f_2(x) + \frac{\alpha_{n}}{2}\|x\|_2^2
 \end{equation}

\end{proof}

We now prove the convergence of the regularized dual averaging method (\ref{RDA_iteration}).
\begin{theorem}
 Assume that $f_1$ is convex with bounded subgradient (i.e. Lipschitz), meaning that $\|g\|_2 \leq M$ for all $g\in \partial f_1(x)$ (for any point $x$).
 
 Let $x^*\in \arg\min_x f(x)$.
 
 Then if $x_n$ is given by the iteration (\ref{RDA_iteration}) with $\alpha_n = c\sqrt{n}$, we have
 \begin{equation}
  f(\bar{x}_n) - f(x^*) \leq \frac{1}{\sqrt{n}}\left(\frac{c}{2}\|x^*\|_2^2 + \frac{M^2}{c}\right) + \frac{f_2(x_1) - \inf_x f_2(x)}{n}
 \end{equation}
 where $\bar{x}_n$ is the average of the first $n$ iterates. Additionally, we have
 \begin{equation}
  \min_{i = 1,...,n} (f(x_i) - f(x^*)) \leq \frac{1}{\sqrt{n}}\left(\frac{c}{2}\|x^*\|_2^2 + \frac{M^2}{c}\right) + \frac{f_2(x_1) - \inf_x f_2(x)}{n}
 \end{equation}
\end{theorem}
\begin{proof}
 The proof is very similar to the proof of Theorem \ref{SDA_convergence_thm}. We begin by noting that (by convexity)
 \begin{equation}
  f(\bar{x}_n) - f(x^*) \leq \frac{1}{n}\displaystyle\sum_{i = 1}^n (f(x_i) - f(x^*))
 \end{equation}
 and
 \begin{equation}
  \min_{i = 1,...,n} (f(x_i) - f(x^*)) \leq \frac{1}{n}\displaystyle\sum_{i = 1}^n (f(x_i) - f(x^*))
 \end{equation}
 Thus it suffices to bound the `average regret.' We begin by decomposing $f = f_1 + f_2$ and using the subdifferential property to get
 \begin{equation}\label{eq808}
  \frac{1}{n}\displaystyle\sum_{i = 1}^n (f(x_i) - f(x^*)) \leq 
  \frac{1}{n}\displaystyle\sum_{i = 1}^n \langle g_i, x_i - x^*\rangle + \frac{1}{n}\displaystyle\sum_{i = 1}^n (f_2(x_i) - f_2(x^*))
 \end{equation}
We now proceed to rewrite and bound the first sum above as in the proof of Theorem \ref{SDA_convergence_thm}. We get
\begin{equation}
 \displaystyle\sum_{i = 1}^n \langle g_i, x_i - x^*\rangle = 
  \displaystyle\sum_{i = 1}^n \langle g_i, x_{n+1} - x^*\rangle + \displaystyle\sum_{i = 1}^n \displaystyle\sum_{j = 1}^i\langle g_j, x_i - x_{i+1}\rangle
\end{equation}
and use Lemmas \ref{lemma751} and \ref{lemma767} to bound
\begin{equation}
 \displaystyle\sum_{i = 1}^n \langle g_i, x_{n+1} - x^*\rangle \leq \frac{\alpha_{n+1}}{2}(\|x^*\|_2^2 - \|x_{n+1}\|_2^2) + n(f_2(x^*) - f_2(x_{n+1}))
\end{equation}
and
 \begin{align}
  \displaystyle\sum_{j = 1}^i\langle g_j, x_i - x_{i+1}\rangle &\leq \frac{\alpha_{i}}{2}(\|x_{i+1}\|_2^2 - \|x_i\|_2^2) \\
  &+ (n-1)(f_2(x_{i+1}) - f_2(x_i)) + \frac{1}{2\alpha_{i}} \|g_i\|_2^2
 \end{align}
 Utilizing our bound on the subgradients $\|g_i\|_2\leq M$, plugging this into equation (\ref{eq646}), and rearranging
 the (almost) telescoping sums, we get
 \begin{align}
  \displaystyle\sum_{i = 1}^n \langle g_i, x_i - x^*\rangle &\leq \frac{\alpha_{n+1}}{2}\|x^*\|_2^2 - \frac{\alpha_1}{2}\|x_1\|_2^2 + nf_2(x^*) - \displaystyle\sum_{i = 2}^{n+1} f_2(x_i) \\
  &+ \frac{1}{2}\displaystyle\sum_{i = 2}^{n+1}(\alpha_{i - 1} - \alpha_{i}) \|x_i\|_2^2 + \frac{M^2}{2}\displaystyle\sum_{i = 1}^n\frac{1}{\alpha_{i}} 
 \end{align}
 Plugging this into equation (\ref{eq808}) we obtain
 \begin{align}
  \displaystyle\sum_{i = 1}^n (f(x_i) - f(x^*)) &\leq \frac{\alpha_{n+1}}{2}\|x^*\|_2^2 - \frac{\alpha_1}{2}\|x_1\|_2^2 + (f_2(x_1) - f_2(x_{n+1})) \\
  &+ \frac{1}{2}\displaystyle\sum_{i = 2}^{n+1}(\alpha_{i - 1} - \alpha_{i}) \|x_i\|_2^2 + \frac{M^2}{2}\displaystyle\sum_{i = 1}^n\frac{1}{\alpha_{i}} 
 \end{align}
 which gives the bound
 \begin{align}
  \displaystyle\sum_{i = 1}^n (f(x_i) - f(x^*)) &\leq \frac{\alpha_{n+1}}{2}\|x^*\|_2^2 - \frac{\alpha_1}{2}\|x_1\|_2^2 + (f_2(x_1) - \inf_x f_2(x)) \\
  &+ \frac{1}{2}\displaystyle\sum_{i = 2}^{n+1}(\alpha_{i - 1} - \alpha_{i}) \|x_i\|_2^2 + \frac{M^2}{2}\displaystyle\sum_{i = 1}^n\frac{1}{\alpha_{i}} 
 \end{align}
 As before, the above inequality is true for any choice of $\alpha_1,...,\alpha_{n+1}$ as long as the $x_i$ are given by the iteration
 (\ref{RDA_iteration}). Since the left hand side of the inequality only depends upon $x_1,...,x_n$, we are free to choose
 $\alpha_{n+1} = \alpha_n$ (without changing the left hand size), and we get
 \begin{align}
  \displaystyle\sum_{i = 1}^n (f(x_i) - f(x^*)) &\leq \frac{\alpha_{n}}{2}\|x^*\|_2^2 - \frac{\alpha_1}{2}\|x_1\|_2^2 + (f_2(x_1) - \inf_x f_2(x)) \\
  &+ \frac{1}{2}\displaystyle\sum_{i = 2}^{n}(\alpha_{i - 1} - \alpha_{i}) \|x_i\|_2^2 + \frac{M^2}{2}\displaystyle\sum_{i = 1}^n\frac{1}{\alpha_{i}} 
 \end{align}
 Now we note that if $\alpha_n$ is an increasing sequence (which it is for our particular choice $\alpha_n = c\sqrt{n}$), we can use 
 $\alpha_{i-1} - \alpha_i \leq 0$ to get
 \begin{equation}
  \displaystyle\sum_{i = 1}^n \langle g_i, x_i - x^*\rangle \leq\frac{\alpha_{n}}{2}\|x^*\|_2^2 + \frac{M^2}{2}\displaystyle\sum_{i = 1}^n\frac{1}{\alpha_{i}} + (f_2(x_1) - \inf_x f_2(x)) 
 \end{equation}
 Plugging in our choice of $\alpha_n = c\sqrt{n}$ and noting that $\sum_{i = 1}^n \frac{1}{\sqrt{i}} \leq 2\sqrt{n}$ gives
 \begin{equation}
  \displaystyle\sum_{i = 1}^n \langle g_i, x_i - x^*\rangle \leq\left(\frac{c}{2}\|x^*\|_2^2 + \frac{M^2}{c}\right)\sqrt{n} + (f_2(x_1) - \inf_x f_2(x))
 \end{equation}
 Dividing this by $n$ completes the proof.
 \end{proof}
 
 A straightforward modification of the above argument provides a proof that the stochastic version of the above algorithm
 converges. We state the relevant theorem but omit the proof.
 \begin{theorem}
 Assume that $f$ is convex and assume that $x_n$ are given by the iteration (\ref{RDA_iteration}) where $g_i$ is replaced by
 an unbiased sample $g(x_i,\xi_i)$ of an element in the subgradient 
 (i.e. $\xi_i$ is a random variable and $\mathbb{E}_{\xi_i}g(x_i,\xi_i) \in \partial f(x_i)$). Assume additionally that the 
 subgradient samples are bounded, i.e. that $\|g(x,\xi)\|_2 \leq M$.
 
 Let $x^*\in \arg\min_x f(x)$.
 
 Then with $\alpha_n = c\sqrt{n}$, we have
 \begin{equation}
  \mathbb{E}(f(\bar{x}_n) - f(x^*)) \leq \frac{1}{\sqrt{n}}\left(\frac{c}{2}\|x^*\|_2^2 + \frac{M^2}{c}\right) + \frac{f_2(x_1) - \inf_x f_2(x)}{n}
 \end{equation}
 where $\bar{x}_n$ is the average of the first $n$ iterates. Additionally, we have
 \begin{equation}
  \mathbb{E}\left(\min_{i = 1,...,n} (f(x_i) - f(x^*))\right) \leq \frac{1}{\sqrt{n}}\left(\frac{c}{2}\|x^*\|_2^2 + \frac{M^2}{c}\right) + \frac{f_2(x_1) - \inf_x f_2(x)}{n}
 \end{equation}
 \end{theorem}

 
%-------------------------------------------------------------------------------------------------------------------

\section{Smooth Convex Functions}
In this section, we consider methods for optimizing $L$-smooth convex functions, i.e. differentiable convex functions
whose gradients are Lipschitz with constant $L$
\begin{equation}
 \|\nabla f(x) - \nabla f(y)\|_2 \leq L\|x - y\|_2
\end{equation}
It is a simple exercise to show that such a function satisfies a quadratic upper bound
\begin{equation}
 f(x)\leq f(y) + \langle\nabla f(x), y - x\rangle + \frac{L}{2}\|y - x\|_2^2
\end{equation}

We will see that with this stronger assumption on the objective, we remove the need to average the iterates in 
addition to obtaining a significantly better convergence rate in the deterministic setting.
\subsection{Gradient Descent}
In this subsection, we analyze the well-known gradient descent method for smooth convex functions
\begin{equation}\label{gradient_descent_smooth}
 x_{n+1} = x_n - s\nabla f(x_n)
\end{equation}
We also find it instructive to consider the behavior of the continuous gradient descent dynamics
\begin{equation}\label{gradient_descent_dynamics_smooth}
 \dot{x}(t) = -\nabla f(x(t))
\end{equation}

Using a Lyapunov argument, we can show that the continuous dynamics (\ref{gradient_descent_dynamics_smooth}) obtains
an (objective) convergence rate of $O(\frac{1}{t})$, as the following theorem shows.

\begin{theorem}\label{smooth_gradient_dynamics_thm} 
 Assume that $f$ is convex and smooth and let $x^*\in \arg\min_x f(x)$. 
 Then the dynamics (\ref{gradient_descent_dynamics_smooth}) satisfies
 \begin{equation}
  f(x(t)) - f(x^*) \leq \frac{1}{2t}\|x(0) - x^*\|_2^2
 \end{equation}

\end{theorem}
\begin{proof}
 Consider the following Lyapunov function
 \begin{equation}
  L(t) = t(f(x(t)) - f(x^*)) + \frac{1}{2}\|x(t) - x^*\|_2^2
 \end{equation}
 We calculate the derivative of $L$
 \begin{equation}
  L^\prime(t) = (f(x(t)) - f(x^*)) + t\langle\nabla f(x(t)), \dot{x}(t)\rangle + \langle\dot{x}(t), (x(t) - x^*)\rangle
 \end{equation}
 substituting $\dot{x}(t) = -\nabla f(x(t))$ in the above equation, we see
 \begin{equation}
  L^\prime(t) = (f(x(t)) - f(x^*)) - \langle\nabla f(x(t)), x(t) - x^*\rangle - t\|\nabla f(x(t))\|_2^2
 \end{equation}
 The convexity of $f$ implies that $(f(x(t)) - f(x^*)) - \langle\nabla f(x(t)), x(t) - x^*\rangle \leq 0$ so we get
 \begin{equation}
  L^\prime(t) \leq 0
 \end{equation}
 This means that $L(t) \leq L(0)$, so we get
 \begin{equation}
  t(f(x(t)) - f(x^*)) \leq L(t) \leq L(0) = \frac{1}{2}\|x(0) - x^*\|_2^2
 \end{equation}
 as desired.

\end{proof}

Interestingly, we can adapt the above Lyapunov argument to the discrete case as long as our step size is small enough.
This provides an intuitive and simple derivation of the following convergence result.
\begin{theorem}\label{gradient_descent_convergence_smooth}
 Let $f$ be an $L$-smooth, convex function and let $x^*\in\arg\min_x f(x)$. 
 Then the gradient descent iteration (\ref{gradient_descent_smooth}) with
 step size $s = \frac{1}{L}$ satisfies
 \begin{equation}
  f(x_n) - f(x^*) \leq \frac{L}{2n}\|x_0 - x^*\|_2^2
 \end{equation}

\end{theorem}
 Note that the convergence rate is significantly improved over subgradient descent and we also
 don't need to average the iterates, we can simply take the last one. This is the improvement that
 smoothness buys us.
 \begin{proof}
  Consider the discrete Lyapunov function
  \begin{equation}
   L(n) = n(f(x_n) - f(x^*)) + \frac{L}{2}\|x_n - x^*\|_2^2
  \end{equation}
  We calculate the change in $L$ as follows
  \begin{align}
   L(n+1) - L(n) &= f(x_n) - f(x^*) + (n+1)(f(x_{n+1}) - f(x_n))\\
   & + L\langle(x_{n+1} - x_n), x_n - x^*\rangle + \frac{L}{2}\|x_{n+1} - x_n\|_2^2
  \end{align}
 Substituting $x_{n+1} - x_n = -s\nabla f(x_n) = \frac{-1}{L}\nabla f(x_n)$ into the above equation
 we obtain
  \begin{align}
   L(n+1) - L(n) &= f(x_n) - f(x^*) + (n+1)(f(x_{n+1}) - f(x_n))\\
   & - \langle\nabla f(x_n), x_n - x^*\rangle + \frac{1}{2L}\|\nabla f(x_n)\|_2^2
  \end{align}
  The smoothness allows us to bound $f(x_{n+1}) - f(x_n)$, i.e. show that we attain a sufficient decrease as follows.
  By substituting $x_{n+1} - x_n = \frac{-1}{L}\nabla f(x_n)$ into
  \begin{equation}
   f(x_{n+1}) - f(x_n) \leq \langle\nabla f(x_n), x_{n+1} - x_n\rangle + \frac{L}{2}\|x_{n+1} - x_n\|_2^2
  \end{equation}
  we get
  \begin{equation}
   f(x_{n+1}) - f(x_n) \leq \frac{-1}{2L}\|\nabla f(x_n)\|_2^2
  \end{equation}
  Combining this with the convexity of $f$, which implies that
  \begin{equation}
   f(x_n) - f(x^*) \leq \langle\nabla f(x_n), x_n - x^*\rangle
  \end{equation}
  we get
  \begin{equation}
   L(n+1) - L(n) \leq \frac{-n}{2L}\|\nabla f(x_n)\|_2^2 \leq 0
  \end{equation}
  This means that $L(n+1)\leq L(n)$ and thus $L(n) \leq L(0)$. So we have
  \begin{equation}
   n(f(x_n) - f(x^*)) \leq L(n) \leq L(0) = \frac{L}{2}\|x_0 - x^*\|_2^2
  \end{equation}
  as desired.

 \end{proof}
 
 To conclude this subsection, we address the practical issue that the smoothness parameter $L$ may not be known a priori.
 It turns out that we can achieve the same convergence rate by using a variable step size, which potentially decreases 
 in each iteration, to ensure that the Armijo condition
 \begin{equation}\label{armijo_condition}
  f(x_n - s_n\nabla f(x_n)) \leq f(x_n) - \frac{s_n^2}{2}\|\nabla f(x_n)\|_2^2
 \end{equation}
 is satisfied. In a practical algorithm, we would start with an initial step size $s_0$ and decrease it appropriately whenever
 (\ref{armijo_condition}) isn't satified. If $f$ is $L$-smooth, we never need to take $s_n$ smaller than $\frac{1}{L}$.
 
 We have the following convergence result in this case.
 \begin{theorem}
  Suppose that $f(x)$ is convex and differentiable. Let the iterates $x_n$ be given by
  \begin{equation}
   x_{n+1} = x_n - s_n\nabla f(x_n)
  \end{equation}
  where $s_n \leq s_{n-1}$ is chosen so that (\ref{armijo_condition}) holds. Then
  \begin{equation}
   f(x_n) - f(x^*) \leq \frac{1}{2ns_n}\|x_0 - x^*\|_2^2
  \end{equation}
 \end{theorem}
 The proof is very similar to the proof of Theorem \ref{gradient_descent_convergence_smooth} and we omit it. The relevant
 (discrete) Lyapunov function is
 \begin{equation}
  L(n) = ns_n(f(x_n) - f(x^*)) + \frac{1}{2}\|x_n - x^*\|_2^2
 \end{equation}

 
\subsection{Adding Momentum}
In this subsection, we introduce the idea of `adding momentum' to the gradient descent method (\ref{gradient_descent_smooth}).
We will explain what adding momentum means and which effect it has by consider the corresponding 
continuous dynamics.

The physical intuition behind `adding momentum' is to consider a particle (of unit mass) trapped in a potential given
by the objective $f(x)$. Such a particle would follow the Newtonian dynamics with acceleration given by $-\nabla f$
\begin{equation}
 \dot{x} = v,~\dot{v} = - \nabla f(x)
\end{equation}
This motion preserves the total energy, or Hamiltonian, given by
\begin{equation}\label{hamiltonian}
 H(x,v) = \frac{1}{2}\|v\|_2^2 + (f(x) - f(x^*))
\end{equation}
where $x^*\in \arg\min_x f(x)$.

Consequently, unless the particle starts at $x^*$ with $v = 0$, the dynamics will continually oscillate around the
minimizer $x^*$. Of course, we want out particle to settle at the minimizer $x^*$. To affect this, we add `friction'
to the dynamics to get the following damped Hamiltonion dynamics
\begin{equation}\label{momentum_descent_dynamics_smooth}
 \dot{x} = v,~\dot{v} = -\alpha v - \nabla f(x)
\end{equation}
which can also be written as
\begin{equation}
 \ddot{x}(t) + \alpha \dot{x}(t) + \nabla f(x(t)) = 0
\end{equation}
where $\alpha$ is the damping rate, which determines how much friction there is.

Under this damped Hamiltonian dynamics, the Hamiltonian (\ref{hamiltonian}) decays, as the following lemma shows.
\begin{lemma}\label{hamiltonian_decay}
 Suppose that $x$ and $v$ satisfy the dynamics (\ref{momentum_descent_dynamics_smooth}). Then
 \begin{equation}
  \frac{d}{dt}H(x(t),v(t)) = -\alpha\|v(t)\|_2^2
 \end{equation}
 where $H(x,v)$ is the Hamiltonian in (\ref{hamiltonian}).

\end{lemma}
\begin{proof}
 We calculate
 \begin{align}
  \frac{d}{dt}H(x(t),v(t)) &= \left\langle\frac{d}{dx}H(x(t),v(t)), \dot{x}(t)\right\rangle + 
  \left\langle\frac{d}{dv}H(x(t),v(t)), \dot{v}(t)\right\rangle \\
   &= \langle \nabla f(x(t)), v(t)\rangle + \langle v(t), -\alpha v(t) - \nabla f(x(t))\rangle \\
   &= -\alpha\|v(t)\|_2^2
 \end{align}
  as desired.
\end{proof}

It is not a priori clear why following the dynamics (\ref{momentum_descent_dynamics_smooth}) is better than following the
gradient flow dynamics (\ref{gradient_descent_dynamics_smooth}). We will see in later sections on accelerated gradient
descent for convex and strongly convex functions why the dynamics (\ref{momentum_descent_dynamics_smooth}) leads to
very useful methods. 

For now, we begin by considering the behavior of the damped Hamiltonian dynamics for smooth, convex
objectives. Using a Lyapunov argument, we obtain the following convergence result.
\begin{theorem}\label{continuous_hamiltonian_convergence_smooth}
 Suppose that $x(t)$ and $y(t)$ satisfy the damped Hamiltonian dynamics (\ref{momentum_descent_dynamics_smooth}) with $\alpha > 0$ and
 let $H(x,v)$ be the Hamiltonian in (\ref{hamiltonian}). Suppose also that $v(0) = 0$. Then if $f$ is convex and smooth, we have
 \begin{equation}\label{hamiltonian_decay_bound_smooth}
  H(x(t),v(t)) \leq \left(t + \frac{1}{2\alpha}\right)^{-1}\left[\frac{3}{2\alpha}(f(x(0)) - f(x^*)) + \frac{\alpha}{2}\left\|x(0) - x^*\right\|_2^2\right]
 \end{equation}

\end{theorem}
The proof of this theorem will also provide a result for $v(0) \neq 0$, but it is messier and we don't write it out explicitly.
\begin{proof}
 Consider the Lyapunov function
 $$
  L(t) = \left(t + \frac{1}{2\alpha}\right)H(x(t),v(t)) + \frac{1}{\alpha}(f(x(t)) - f(x^*)) + \frac{\alpha}{2}\|(x - x^*) + \frac{1}{\alpha}v\|_2^2
 $$
 We calculate the derivative of $L(t)$ as follows. First we note that
 $$
  \frac{d}{dt}\left[\left(t + \frac{1}{2\alpha}\right)H(x(t),v(t))\right] = H(x(t),v(t)) + \left(t + \frac{1}{2\alpha}\right)\frac{d}{dt}H(x(t),v(t))
 $$
 Using Lemma \ref{hamiltonian_decay} we get
 \begin{equation}\label{eq700}
  \frac{d}{dt}\left[\left(t + \frac{1}{2\alpha}\right)H(x(t),v(t))\right] = H(x(t),v(t)) - \alpha\left(t + \frac{1}{2\alpha}\right)\|v(t)\|_2^2
 \end{equation}
 We also have
 \begin{equation}\label{eq704}
  \frac{d}{dt}\left(\frac{1}{\alpha}(f(x(t)) - f(x^*))\right) = \frac{1}{\alpha}\langle\nabla f(x(t)), \dot{x}(t)\rangle = \frac{1}{\alpha}\langle\nabla f(x(t)), v(t)\rangle
 \end{equation}
 and
 $$
  \frac{d}{dt}\left(\frac{\alpha}{2}\|(x - x^*) + \frac{1}{\alpha}v\|_2^2\right) = \alpha\left\langle \dot{x}(t) + \frac{1}{\alpha}\dot{v}(t), (x - x^*) + \frac{1}{\alpha}v(t)\right\rangle
 $$
 Pluggin the dynamics (\ref{momentum_descent_dynamics_smooth}) into this, we see that $\dot{x}(t) + \frac{1}{\alpha}\dot{v}(t) = -\frac{1}{\alpha}\nabla f(x(t))$, and so
 \begin{equation}\label{eq712}
  \frac{d}{dt}\left(\frac{\alpha}{2}\|(x - x^*) + \frac{1}{\alpha}v\|_2^2\right) = -\left\langle\nabla f(x(t)), x(t) - x^* + \frac{1}{\alpha}v(t)\right\rangle
 \end{equation}
 Adding equations (\ref{eq700}), (\ref{eq704}), and (\ref{eq712}), and noting that the convexity of $f$ implies
 \begin{align}
 H(x(t),v(t)) &= \frac{1}{2}\|v(t)\|_2^2 + (f(x(t)) - f(x^*)) \\
 &\leq \frac{1}{2}\|v(t)\|_2^2 + \langle\nabla f(x(t)), x(t) - x^*\rangle
 \end{align}
 we get
 \begin{equation}
  L^\prime(t) \leq -\alpha t\|v(t)\|_2^2 \leq 0
 \end{equation}
 This means that $L(t) \leq L(0)$ and so (using the assumption that $v(0) = 0$) we get
 \begin{align}
  &\left(t + \frac{1}{2\alpha}\right)H(x(t),v(t)) \leq L(t)\leq L(0) \\
   &= \frac{3}{2\alpha}(f(x(0)) - f(x^*)) + \frac{\alpha}{2}\left\|x(0) - x^*\right\|_2^2
 \end{align}
 as desired.

\end{proof}

So the damped Hamiltonian dynamics (\ref{momentum_descent_dynamics_smooth}) obtains an $O(\frac{1}{t})$ convergence rate
similar to the gradient flow dynamics (\ref{gradient_descent_dynamics_smooth}). We can actually relate these two dynamics more 
precisely as follows. We consider the following form of equation (\ref{momentum_descent_dynamics_smooth})
$$ \ddot{x}(t) + \alpha \dot{x}(t) + \nabla f(x(t)) = 0
$$
We first rescale time, setting $t^\prime = t / \alpha$ this becomes
$$ \frac{1}{\alpha^2}\ddot{x}(t^\prime) + \dot{x}(t^\prime) + \nabla f(x(t^\prime)) = 0
$$
Now sending $\alpha\rightarrow \infty$ recovers the gradient flow dynamics
$$\dot{x}(t^\prime) + \nabla f(x(t^\prime)) = 0
$$
So one way of thinking about the relationship between the gradient flow and damped Hamiltonian dynamics is that the
gradient flow arises as the overdamped and time rescaled limit of the Hamiltonian dynamics. 

We can use this perspective to
relate Theorems \ref{continuous_hamiltonian_convergence_smooth} and \ref{smooth_gradient_dynamics_thm}. In particular,
if we substitute $t = \alpha t^\prime$ into the bound
$$H(x(t),v(t)) \leq \left(t + \frac{1}{2\alpha}\right)^{-1}\left[\frac{3}{2\alpha}(f(x(0)) - f(x^*)) + \frac{\alpha}{2}\left\|x(0) - x^*\right\|_2^2\right]
$$
and take a limit as $\alpha\rightarrow \infty$, we recover the bound from Theorem \ref{smooth_gradient_dynamics_thm}
$$f(x(t^\prime)) - f(x^*) \leq H(x(t^\prime),v(t^\prime)) \leq \frac{1}{2t^\prime}\|x(0) - x^*\|_2^2
$$

To conclude this subsection, we consider the problem of discretizing the dynamics (\ref{momentum_descent_dynamics_smooth}).
We saw that the simple gradient descent discretization (\ref{gradient_descent_smooth}) of the gradient flow achieved
the same convergence rate
(as long as the step size was sufficiently small, see Theorem \ref{gradient_descent_convergence_smooth}). 
We wish to find an analogous discretization
of (\ref{momentum_descent_dynamics_smooth}).

There are many discrete schemes that correspond to the Hamiltonian dynamics (\ref{momentum_descent_dynamics_smooth}). However,
one of the most robust schemes for the purpose of convex optimization is
\begin{equation}\label{momentum_gradient_descent_smooth}
 x_{n+1} = y_n - s\nabla f(y_n),~y_{n+1} = x_{n+1} + s\alpha (x_{n+1} - x_n),~x_0 = y_0
\end{equation}

\section{Strongly Convex Functions}
In this section we strengthen our assumptions on the objective function $f$ and examine how this impacts the convergence rate we
can obtain. 

In particular, we assume that the function $f$ is $\alpha$-strongly convex. This means that 
$f - \frac{\alpha}{2}\|x\|_2^2$ is convex and is equivalent to the bound
\begin{equation}
 f(y) \geq f(x) + \langle g, y - x\rangle + \frac{\alpha}{2}\|y-x\|_2^2
\end{equation}


\section{Dual Averaging}
In this section, we introduce Nesterov's method of dual averaging. In order to best explain where this method comes from,
we first start with a different perspective on subgradient descent. We note that the subgradient descent iteration
(\ref{subgradient_descent_variable}) (with $x_1 = 0$) can also be written as
\begin{equation}\label{minimizing_subgradient_descent}
 x_{n+1} = \arg\min_x \left(\displaystyle\sum_{i = 1}^n \langle s_ig_i, x\rangle + \frac{1}{2}\|x\|_2^2\right)
\end{equation}

This can easily be proven by induction on $n$ and we leave this as an exercise. The above minimization problem
can be rewritten as
\begin{equation}
 x_{n+1} = \arg\min_x \left(\displaystyle\sum_{i = 1}^ns_i(f(x_i) + \langle g_i, x - x_i\rangle) + \frac{1}{2}\|x\|_2^2\right)
\end{equation}
and we note that the terms $f(x_i) + \langle g_i, x - x_i\rangle$ are lower bounds on the objective $f$. Thus, the next iterate
can be obtained by optimizing an average of these lower bounds (with weights $s_i$) plus a regularization term which is a multiple
of $\|x\|_2^2$.

In the previous section we set $s_i = \frac{1}{\sqrt{i}}$. This means that we are giving more weight to the bounds coming from
earlier iterates, which is odd. The starting point for Nesterov's dual averaging method is to ask whether we can change the
weights $s_i$. It turns out that we can, but in order to do this we must weight the regularization term differently as well.

We thus modify (\ref{minimizing_subgradient_descent}) and consider methods of the form
\begin{equation}\label{generalized_dual_averaging}
 x_{n+1} = \arg\min_x \left(\displaystyle\sum_{i = 1}^n \langle s_ig_i, x\rangle + \frac{\alpha_{n+1}}{2}\|x\|_2^2\right)
\end{equation}
where the sequences $s_i,\alpha_i > 0$ are parameters (and $g_i\in \partial f(x_i)$ as before).

We obtain the following convergence result for the dual averaging method as a consequence of the definition (\ref{generalized_dual_averaging}).
\begin{theorem}\label{dual_averaging_theorem}
 Assume that $f$ is convex and Lipschitz with constant $M$, i.e. $\|g\|_2\leq M$ for $g\in \partial f(x)$. Let 
 $x^*\in \arg\min_x f(x)$.
 
 Then if the sequence $x_i$ is given by (\ref{generalized_dual_averaging}) (note in particular that this means that
 $x_1 = 0$) with $\alpha_i \leq \alpha_{i+1}$, we have
 \begin{equation}
  f(\bar{x}_n) - f(x^*) \leq \frac{1}{2}\left(\displaystyle\sum_{i = 1}^n s_i\right)^{-1} \left(\alpha_n\|x^*\|_2^2 + M\displaystyle\sum_{i = 1}^n \frac{s_i^2}{\alpha_i}\right)
 \end{equation}
 where $\bar{x}_n = (\sum_{i = 1}^n s_i)^{-1}\sum_{i = 1}^n s_ix_i$ is a weighted average of the iterates with weights $s_i$.
 We also have
 \begin{equation}
  \min_{i = 1,...,n} f(x_i) - f(x^*) \leq \frac{1}{2}\left(\displaystyle\sum_{i = 1}^n s_i\right)^{-1} \left(\alpha_n\|x^*\|_2^2 + M\displaystyle\sum_{i = 1}^n \frac{s_i^2}{\alpha_i}\right)
 \end{equation}

\end{theorem}
\begin{proof}
 We begin by noting that by convexity we have
 \begin{equation}
  f(\bar{x}_n) - f(x^*) \leq \left(\displaystyle\sum_{i = 1}^n s_i\right)^{-1}\displaystyle\sum_{i = 1}^n s_i(f(x_i) - f(x^*))
 \end{equation}
 and since the minimum is smaller than the average we have
 \begin{equation}
  \min_{i = 1,...,n} f(x_i) - f(x^*) \leq \left(\displaystyle\sum_{i = 1}^n s_i\right)^{-1}\displaystyle\sum_{i = 1}^n s_i(f(x_i) - f(x^*))
 \end{equation}
 Thus it suffices to prove that
 \begin{equation}
  \displaystyle\sum_{i = 1}^n s_i(f(x_i) - f(x^*)) \leq \frac{1}{2}\left(\alpha_n\|x^*\|_2^2 + M\displaystyle\sum_{i = 1}^n \frac{s_i^2}{\alpha_i}\right)
 \end{equation}
 For this we first use the subgradient property to get
 \begin{equation}\label{eq_dual_averaging_59}
  \displaystyle\sum_{i = 1}^n s_i(f(x_i) - f(x^*)) \leq \displaystyle\sum_{i = 1}^n \langle s_ig_i, x_i - x^*\rangle
 \end{equation}
 and rewrite this sum as follows
 \begin{equation}\label{eq_dual_averaging_63}
  \displaystyle\sum_{i = 1}^n \langle s_ig_i, x_i - x^*\rangle = \displaystyle\sum_{i = 1}^n \langle s_ig_i, x_n - x^*\rangle
  + \displaystyle\sum_{i = 1}^{n-1} \displaystyle\sum_{j = 1}^i\langle s_jg_j, x_i - x_{i+1}\rangle 
 \end{equation}
 We proceed by bounding the term
 \begin{equation}
  \sum_{j = 1}^i\langle s_jg_j, x_i - z\rangle
 \end{equation}
 Recall from (\ref{generalized_dual_averaging}) that $x_i = \arg\min_x f_i(x)$ where
 \begin{equation}
  f_i(x) = \displaystyle\sum_{j = 1}^{i-1} \langle s_jg_j, x\rangle + \frac{\alpha_{i}}{2}\|x\|_2^2
 \end{equation}
 This means that $0\in\partial f_i(x_i)$ and thus, defining 
 $$f_i^\prime(x) = f_i(x) + \langle s_ig_i, x\rangle$$ 
 we see that $s_ig_i\in \partial f_i^\prime(x_i)$. Moreover, $f_i^\prime$ is strongly convex with convexity parameter
 $\alpha_i$, which implies that
 $$f_i^\prime(x_i) \leq f_i^\prime(z) + \frac{1}{2\alpha_i}\|s_ig_i\|_2^2
 $$
 Using the definition of $f_i^\prime$ and rearranging, this gives
 \begin{equation}
  \sum_{j = 1}^i\langle s_jg_j, x_i - z\rangle \leq \frac{\alpha_i}{2}(\|z\|_2^2 - \|x_i\|_2^2) + \frac{s_i^2}{2\alpha_i}\|g_i\|_2^2
 \end{equation}
 Pluggin this into (\ref{eq_dual_averaging_63}) with $z = x_{i+1}$ and $z = x^*$, we obtain (recalling that $x_1 = 0$ and
 $\|g_i\|_2 \leq M$)
 \begin{equation}
  \displaystyle\sum_{i = 1}^n \langle s_ig_i, x_i - x^*\rangle \leq \frac{\alpha_n}{2}\|x^*\|_2^2 
  + \frac{1}{2}\displaystyle\sum_{i = 2}^n (\alpha_{i-1} - \alpha_i)\|x_i\|_2^2 + \frac{M}{2}\displaystyle\sum_{i = 2}^n \frac{s_i^2}{\alpha_i}
 \end{equation}
 Now we use the assumption that $\alpha_i$ is an non-decreasing sequence, i.e. $$\alpha_{i-1} - \alpha_i \leq 0$$ to see that
 the middle sum above is negative. Combining this with (\ref{eq_dual_averaging_59}) we see that
 \begin{equation}
  \displaystyle\sum_{i = 1}^n s_i(f(x_i) - f(x^*)) \leq \frac{1}{2}\left(\alpha_n\|x^*\|_2^2 + M\displaystyle\sum_{i = 1}^n \frac{s_i^2}{\alpha_i}\right)
 \end{equation}
 which completes the proof.

\end{proof}

We conclude this section by noting several consequences of this theorem. First, the choice $s_i = \frac{1}{\sqrt{i}}$
and $\alpha_i = 1$ recovers subgradient descent. For this choice of $s_i$ and $\alpha_i$
$$\displaystyle\sum_{i = 1}^n s_i = O(\sqrt{n})
$$
and
$$\displaystyle\sum_{i = 1}^n \frac{s_i^2}{\alpha_i} = O(\log(n))
$$
and we obtain the bound from Theorem \ref{variable_subgradient_descent_thm}. 

We can remove the logarithmic factor by
shifting the weights $s_i$ to the more recent iterates. For example, if we set $s_i = 1$ and $\alpha_i = \sqrt{i}$,
we recover the dual averaging method originally proposed by Nesterov, for which we have
$$\displaystyle\sum_{i = 1}^n s_i = n
$$
and
$$\displaystyle\sum_{i = 1}^n \frac{s_i^2}{\alpha_i} = O(\sqrt{n})
$$
which removes the logarithmic factor we had for subgradient descent. 

Other choices are also reasonable, for example we could
set $s_n = n^c$ and $\alpha_n = n^{c + (1/2)}$ for any $c \geq -(1/2)$. As long as $c > -(1/2)$ we obtain the same convergence
rate up to a constant, if $c = (1/2)$, then we recover subgradient descent which introduces a logarithmic factor.

Next, we address the issue of scaling. If we scale the sequences $s_i$ and $\alpha_i$ by the same (positive) factor,
then both the bound in Theorem \ref{dual_averaging_theorem} and the method (\ref{generalized_dual_averaging}) don't change.
However, scaling one sequence and not the other has the effect of changing the `step size' of the method, and 
changes the relative importance of the initial distance to the optimum $\|x^*\|_2^2$ and the Lipschitz constant $M$ in the
our bound.

Finally, we address the issue of setting the initial iterate to $0$. This is done solely out of notational convenience. In particular,
if we change the regularization term $\|x\|_2^2$ to $\|x - x_1\|_2^2$, our analysis is unchanged. In fact, we can replace the 
$2$-norm regularizer by any strongly convex function, as the reader can easily verify. This is an important point which we
will use in a later section to extend these methods to the Banach space setting.

\section{Regularized Dual Averaging}
In this section, we generalize the dual averaging method (\ref{generalized_dual_averaging}) to deal
with composite optimization problems, i.e. problems of the form
\begin{equation}\label{composite_optimization}
\arg\min_x \left[f(x) = F(x) + G(x)\right] 
\end{equation}
where $F$ is a convex, Lipschitz function and $G$ is a convex function for which we can efficiently solve the proximal,
or backward step
\begin{equation}\label{prox_regularlized_dual_averaging}
 P_{\lambda G}(x) = \arg\min_{y} \frac{1}{2}\|y - x\|_2^2 + \lambda G(y)
\end{equation}
Two very common choices for $G$ are the indicator function of a convex set $A$
\begin{equation}
 G(x) = i_A(x) = 
 \begin{cases}
                    0 &~ \text{if $x\in A$} \\
		    +\infty &~ \text{if $x\notin A$}
 \end{cases}
\end{equation}
and the l$1$-norm $G(x) = \|x\|_1$. In the former case, the proximal map is just a projection onto the set $A$ and
we obtain methods for solving constrained optimization problems. In the latter case, the
proximal map is called soft-thresholding and can be given in closed form.

It is important to note that the methods we will derive for the composite problem (\ref{composite_optimization})
have convergence rates which only depend upon the Lipschitz constant of $F$. This is where the drastic improvement
lies, since often $G$ will not be Lipschitz at all, for instance $G = i_A$, or the Lipschitz constant of $G$ 
will be very large, for instance $G = \|\cdot\|_1$.

To derive the regularized dual averaging methods, we begin by noting that if $g_i\in \partial F(x_i)$, then
$$ F(x_i) + \langle g_i, x - x_i\rangle + G(x)
$$
is a lower bound for $f(x)$. Substituting this lower bound into the dual averaging iteration (\ref{generalized_dual_averaging}),
we arrive at the following iteration
\begin{equation}\label{naive_regularized_dual_averaging}
 x_{n+1} = \arg\min_x \left(\displaystyle\sum_{i = 1}^n \langle s_ig_i, x\rangle + \frac{\alpha_{n+1}}{2}\|x\|_2^2 + \gamma_{n+1} G(x)\right)
\end{equation}
where $\gamma_{n + 1} = \sum_{i = 1}^n s_i$.

Before we proceed, it is useful to write out the iteration (\ref{naive_regularized_dual_averaging}) in two steps by completing the
square
\begin{equation}
 x_{n+\frac{1}{2}} = -\frac{1}{\alpha_{n+1}}\displaystyle\sum_{i = 1}^n s_ig_i,~x_{n+1} = P_{(\gamma_{n+1}/\alpha_{n+1})G}(x_{n+\frac{1}{2}})
\end{equation}
We note that the second, i.e. proximal, step above means that
$$h_{n+1}:=\frac{\alpha_{n+1}}{\gamma_{n+1}}(x_{n+\frac{1}{2}} - x_{n+1})\in \partial G(x_{n+1})
$$
So in the process of computing the minimizer in (\ref{naive_regularized_dual_averaging}) we obtain an element
of $\partial G(x_{n+1})$ as a byproduct. We can use this element to lower bound $f$ in a different way, namely by
$$ F(x_i) + \langle g_i, x - x_i\rangle + f_i(G(x_i) + \langle h_i, x - x_i\rangle) + (1 - f_i)G(x)
$$
where $0 \leq f_i$. Here we have replaced a fraction $f_i$ of the bound $G(x)$ by its linear lower bound
based on $h_i\in \partial G(x_i)$. Writing $t_i = s_if_i$, this leads to the following method
\begin{equation}\label{argmin_formulation_generalized_RDA}
 x_{n+1} = \arg\min_x \left(\displaystyle\sum_{i = 1}^n \langle s_ig_i + t_ih_i, x\rangle + \frac{\alpha_{n+1}}{2}\|x\|_2^2 + \gamma_{n+1} G(x)\right)
\end{equation}
i.e.
\begin{align}\label{first_attempt_iteration}
 x_{n+\frac{1}{2}} = -\frac{1}{\alpha_{n+1}}\displaystyle\sum_{i = 1}^n &s_ig_i + t_ih_i = \frac{\alpha_n}{\alpha_{n+1}}x_{n-\frac{1}{2}} - \frac{s_ng_n + t_nh_n}{\alpha_{n+1}}\\
 x_{n+1} &= P_{(\gamma_{n+1}/\alpha_{n+1})G}(x_{n+\frac{1}{2}})
\end{align}
where $\gamma_{n + 1} = \sum_{i = 1}^n s_i-t_i$, $h_i = \frac{\alpha_{i}}{\gamma_{i}}(x_{i-\frac{1}{2}} - x_{i})$, and
$g_i\in \partial f(x_i)$. Additionally, we must have $t_1 = 0$ since we don't have an element in the subdifferential
of $G$ at $x_1$ (as $\gamma_1 = 0$).

By utilizing the definition of $h_i$ in (\ref{first_attempt_iteration}), we can rewrite this iteration in the following simpler way
\begin{align}\label{generalized_RDA_iteration}
 &x_{n+\frac{1}{2}} = \frac{\alpha_n}{\alpha_{n+1}}\left(\frac{t_n}{\gamma_n}x_n + \left(1 - \frac{t_n}{\gamma_n}\right)x_{n-\frac{1}{2}}\right) - \frac{s_n}{\alpha_{n+1}}g_n \\
 &x_{n+1} = \arg\min_x \left(\frac{1}{2}\|x - x_{n+\frac{1}{2}}\|_2^2 + \frac{\gamma_{n+1}}{\alpha_{n+1}}G(x)\right)
\end{align}

We have the following convergence result for the generalized RDA method.
\begin{theorem}
 Assume that $F$ is convex and Lipschitz with parameter $M$ and let $x^*\in \arg\min_x f(x)$. 
 Then if $x_n$ is given by the iteration (\ref{generalized_RDA_iteration})
 with $t_i\leq \gamma_i$, $\alpha_i\leq \alpha_{i+1}$, and $s_{i+1}\leq s_i$, then we have
 \begin{align}
  &f(\bar{x}_n) - f(x^*) \leq \\
  &\left(\displaystyle\sum_{i = 1}^n s_i\right)^{-1} \left(\alpha_n\|x^*\|_2^2 + M^2\displaystyle\sum_{i = 1}^n \frac{s_i^2}{\alpha_i} + s_1(G(x_1) - \inf_x G(x))\right)
 \end{align}
 where $\bar{x}_n = (\sum_{i = 1}^n s_i)^{-1}\sum_{i = 1}^n s_ix_i$ is a weighted average of the iterates with weights $s_i$.
 We also have
 \begin{align}
  &\min_{i = 1,...,n} f(x_i) - f(x^*) \leq \\
  &\left(\displaystyle\sum_{i = 1}^n s_i\right)^{-1}\left(\alpha_n\|x^*\|_2^2 + M^2\displaystyle\sum_{i = 1}^n \frac{s_i^2}{\alpha_i} + s_1(G(x_1) - \inf_x G(x))\right)
 \end{align}
\end{theorem}

\begin{proof}
  We begin by noting that by convexity we have
 \begin{equation}
  f(\bar{x}_n) - f(x^*) \leq \left(\displaystyle\sum_{i = 1}^n s_i\right)^{-1}\displaystyle\sum_{i = 1}^n s_i(f(x_i) - f(x^*))
 \end{equation}
 and since the minimum is smaller than the average we have
 \begin{equation}
  \min_{i = 1,...,n} f(x_i) - f(x^*) \leq \left(\displaystyle\sum_{i = 1}^n s_i\right)^{-1}\displaystyle\sum_{i = 1}^n s_i(f(x_i) - f(x^*))
 \end{equation}
 Thus it suffices to prove that
 \begin{equation}
 \displaystyle\sum_{i = 1}^n s_i(f(x_i) - f(x^*)) \leq 
 \frac{1}{2}\left(\alpha_n\|x^*\|_2^2 + M^2\displaystyle\sum_{i = 1}^n \frac{s_i^2}{\alpha_i}\right)+ s_1(G(x_1) - \inf_x G(x))
 \end{equation}
 For this we first use the subgradient property to get
 \begin{align}\label{eq_rda_100}
  \displaystyle\sum_{i = 1}^n s_i(f(x_i) - f(x^*)) &\leq \displaystyle\sum_{i = 1}^n \langle s_ig_i + t_ih_i, x_i - x^*\rangle \\
  &+ \displaystyle\sum_{i = 1}^n (s_i - t_i)(G(x_i) - G(x^*))
 \end{align}
 We now rewrite the first sum above as
 \begin{align}\label{eq_rda_105}
  \displaystyle\sum_{i = 1}^n \langle s_ig_i + t_ih_i, x_i - x^*\rangle &= \displaystyle\sum_{i = 1}^n \langle s_ig_i + t_ih_i, x_n - x^*\rangle \\
  &+ \displaystyle\sum_{i = 1}^{n-1} \displaystyle\sum_{j = 1}^i\langle s_jg_j + t_jh_j, x_i - x_{i+1}\rangle 
 \end{align}
 We proceed by bounding the term
 \begin{equation}
  \sum_{j = 1}^i\langle s_jg_j + t_jh_j, x_i - z\rangle
 \end{equation}
 Recall from (\ref{argmin_formulation_generalized_RDA}) that $x_i = \arg\min_x f_i(x)$ where
 \begin{equation}
  f_i(x) = \displaystyle\sum_{j = 1}^{i-1} \langle s_jg_j + t_jh_j, x\rangle + \frac{\alpha_{i}}{2}\|x\|_2^2 + \gamma_i G(x_i)
 \end{equation}
 The definition of $h_i\in \partial G(x_i)$ means that
 \begin{equation}
  0 = \alpha_i x_i + \gamma_i h_i + \displaystyle\sum_{j = 1}^{i-1} s_jg_j + t_jh_j
 \end{equation}
 This means that $0\in \partial f_i^\prime(x_i)$ if we set 
 $$f_i^\prime(x) = f_i(x) + \langle t_ih_i, x\rangle - t_iG(x)
 $$
 Moreover, $f_i^\prime$ is a strongly convex function with convexity parameter $\alpha_i$ as long as $t_i\leq \gamma_i$.
 Now, define $f^{\prime\prime}_i(x) = f_i^\prime(x) + \langle s_ig_i, x\rangle$. Then we have
 $s_ig_i\in \partial f^{\prime\prime}_i(x_i)$ and the strong convexity of $f^{\prime\prime}_i$ implies that
 \begin{equation}
  f^{\prime\prime}_i(x_i) \leq f^{\prime\prime}_i(z) + \frac{1}{2\alpha_i}\|s_ig_i\|_2^2
 \end{equation}
 Substituting in the definition of $f^{\prime\prime}_i$ and rearranging, we get
 \begin{align}
  \sum_{j = 1}^i\langle s_jg_j + t_jh_j, x_i - z\rangle &\leq \frac{\alpha_i}{2}(\|z\|_2^2 - \|x_i\|_2^2) + \frac{s_i^2}{2\alpha_i}\|g_i\|_2^2 \\
  &+ (\gamma_i - t_i)(G(z) - G(x_i))
 \end{align}
 Plugging this bound into (\ref{eq_rda_105}) with $z = x_{i+1}$ and $z = x^*$, recalling that 
 $\gamma_n = \sum_{i = 1}^{n-1} s_i - t_i$, that $x_1 = 0$, $\gamma_1 = t_1 = 0$, and that
 $\|g_i\|_2\leq M$, and rearranging terms, we obtain
 \begin{align}
  \displaystyle\sum_{i = 1}^n \langle s_ig_i + t_ih_i, x_i - x^*\rangle& \leq \frac{\alpha_n}{2}\|x^*\|_2^2 
  + \frac{1}{2}\displaystyle\sum_{i = 2}^n (\alpha_{i-1} - \alpha_i)\|x_i\|_2^2 \\
  &+ \frac{M^2}{2}\displaystyle\sum_{i = 1}^n \frac{s_i^2}{\alpha_i}
  + \displaystyle\sum_{i = 2}^n (t_i - s_{i-1})G(x_i)
 \end{align}
 Plugging this bound into (\ref{eq_rda_100}) we obtain
 \begin{align}
  \displaystyle\sum_{i = 1}^n s_i(f(x_i) - f(x^*)) &\leq \frac{\alpha_n}{2}\|x^*\|_2^2 
  + \frac{M^2}{2}\displaystyle\sum_{i = 1}^n \frac{s_i^2}{\alpha_i} \\
  &+ \frac{1}{2}\displaystyle\sum_{i = 2}^n (\alpha_{i-1} - \alpha_i)\|x_i\|_2^2 \\
  &+ s_1 G(x_1) - \displaystyle\sum_{i = 2}^n (s_{i-1} - s_i)G(x_i) - s_nG(x^*)
 \end{align}
 Finally, we use the assumption that $\alpha_{i-1} - \alpha_i \leq 0$ and that $s_{i-1} - s_i \geq 0$ to conclude that
 $$\frac{1}{2}\displaystyle\sum_{i = 2}^n (\alpha_{i-1} - \alpha_i)\|x_i\|_2^2 \leq 0
 $$
 and
 $$s_1 G(x_1) - \displaystyle\sum_{i = 2}^n (s_{i-1} - s_i)G(x_i) - s_nG(x^*) \leq s_1(G(x_1) - \inf_x G(x))
 $$
 Finally, this yields
 \begin{equation}
  \displaystyle\sum_{i = 1}^n s_i(f(x_i) - f(x^*)) \leq \frac{\alpha_n}{2}\|x^*\|_2^2 
  + \frac{M^2}{2}\displaystyle\sum_{i = 1}^n \frac{s_i^2}{\alpha_i}
  + s_1(G(x_1) - \inf_x G(x))
 \end{equation}
 which completes the proof.

\end{proof}

We begin by noting that the regularizer $\|x\|_2^2$ can be replaced by any strongly convex function. In particular,
if we replace it by $\|x - x_1\|_2^2$, we are free to choose $x_1$ arbitrarily. This allows us to remove the
annoying term $s_1(G(x_1) - \inf_x G(x))$ from the bound in the above theorem, by simply choosing $x_1$ to be a minimizer
of $G$.

We would also like to note two special cases of the above method. If we set $t_i = \frac{1}{\sqrt{i-1}}$ for $i > 1$,
$s_i = \frac{1}{\sqrt{i}}$, and $\alpha_i = 1$ (note that this means $\gamma_i = t_i$), 
then we recover forward-backward subgradient descent.

Also, if we set $t_i = 0$, $s_i = 1$, and $\alpha_i = \sqrt{n}$ (note that this means $\gamma_i = i$), we recover the
original version of RDA.
